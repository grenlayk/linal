\documentclass[a4paper]{article}
\usepackage{header}
\newcommand{\Mat}{\operatorname{Mat}}
\newcommand{\M}{\operatorname{M}}
\allowdisplaybreaks

%% Графика
\usepackage{graphicx}       
\graphicspath{{images/}}            
\usepackage{tikz}  
\usetikzlibrary{patterns}                 
\usepackage{pgfplots}              
\usepackage{circuitikz}
\usepackage{bm}
            

\usetikzlibrary{quotes,angles}
\usetikzlibrary{positioning,intersections}
\usetikzlibrary{through}

\textwidth=19.0cm \oddsidemargin=-1.3cm
\textheight=25cm \topmargin=-2.5cm

\newtheorem{task}{Задание}
\newtheorem*{task*}{Задание}

\theoremstyle{remark}

\newtheorem{remark}{Замечание}
\newtheorem*{remark*}{Замечание}
\newtheorem{commentarium}{Комментарий}
\newtheorem*{commentarium*}{Комментарий}

\usepackage{tikz-cd}

\newenvironment{sysmatrix}[1]
{
    \left(\begin{array}{@{}#1@{}}
}
{\end{array}\right)}
\newcommand{\smt}[2]{\begin{sysmatrix}{#1} #2\end{sysmatrix}}
\newcommand{\eq}[1]{\begin{cases} #1 \end{cases}}

\newcommand{\elon}[3]{%
  \ensuremath{\text{Э}_1(#1,\; #2,\; #3)}%
}
\newcommand{\eltw}[2]{%
  \ensuremath{\text{Э}_2(#1,\; #2)}%
}
\newcommand{\elth}[2]{%
  \ensuremath{\text{Э}_3(#1,\; #2)}%
}
 
\newcommand{\eqon}[3]{%
  \ensuremath{\overset{\text{Э}_1(#1,\; #2,\; #3)}{=}}%
}
\newcommand{\eqtw}[2]{%
  \ensuremath{\overset{\text{Э}_2(#1,\; #2)}{=}}%
}
\newcommand{\eqth}[2]{%
  \ensuremath{\overset{\text{Э}_3(#1,\; #2)}{=}}%
}

\newcommand{\arron}[3]{%
  \ensuremath{\xrightarrow{\text{Э}_1(#1,\; #2,\; #3)}}%
}
\newcommand{\arrtw}[2]{%
  \ensuremath{\xrightarrow{\text{Э}_2(#1,\; #2)}}%
}
\newcommand{\arrth}[2]{%
  \ensuremath{\xrightarrow{\text{Э}_3(#1,\; #2)}}%
}

\makeatletter
\renewcommand*\env@matrix[1][*\c@MaxMatrixCols c]{%
  \hskip -\arraycolsep
  \let\@ifnextchar\new@ifnextchar
  \array{#1}}
\makeatother

% tasks here https://www.dropbox.com/s/8yhwwpf3dvft1yd/LA_19-20_Stream2_Test2.pdf?dl=0
\title{Весенняя контрольная работа учебного года 2019/2020\\Вариант \textnumero 1}
\author{	
  % ОТКОММЕНТИРУЙ СЕБЯ
  Алиса Вернигор       \\ \href{https://t.me/allisyonok}{Telegram} \and
	Сергей Лоптев        \\ \href{https://t.me/beast_sl}{Telegram} \and
	Оля Козлова        \\ \href{https://t.me/grenlayk}{Telegram}
}

\date{}

\begin{document}
	\maketitle
    \section*{Задача \textnumero 1}
        % текст задачи
        Рассмотрим \dots
        \begin{proof}[Решение.]
                
        \end{proof}	 
    
    \section*{Задача \textnumero 2}
        % текст задачи
        Пусть \dots
        \begin{proof}[Решение.]

        \end{proof}
    
    \section*{Задача \textnumero 3}
        % текст задачи
        Даны \dots
        \begin{proof}[Решение.] \ 
            
        \end{proof}
      
    \section*{Задача \textnumero 4}
        % текст задачи
        В евклидовом \dots
        \begin{proof}[Решение.]
            
        \end{proof}
    
    \section*{Задача \textnumero 5}
        В пространстве $V = \mathbb{R}[x]_{\leqslant 2}$, снабжённом структурой евклидова пространства
        относительно некоторого скалярного произведения, объём параллелепипеда, натянутого на векторы
        $2 - x + 3x^2, 3 + x + x^2, 1 + 3x - 3x^2$, равен 4. Найдите объём параллелепипеда, натянутого
        на векторы $1 + x + x^2, 3 - 2x + 3x^2, -1 + 2x - 2x^2$. 
         \begin{proof}[Решение.] \ 
          \begin{itemize}
              \item Пусть $v_1 = 2 - x + 3x^2, v_2 = 3 + x + x^2, v_3 = 1 + 3x - 3x^2$. \\
              И пусть $v_1' = 1 + x + x^2, v_2' = 3 - 2x + 3x^2, v_3' = -1 + 2x - 2x^2$.
              \item Формула объёма $k$-мерного (в нашем случе 3-мерного) параллелепипеда (материал
              лекций): $VolP(v_1, v_2, v_3) = \sqrt{detG(v_1, v_2, v_3)}$, где $G(v_1, v_2, v_3)$ --
              матрица Грамма системы векторов $v_1, v_2, v_3$.
              \item Нам известно, что $\sqrt{detG(v_1, v_2, v_3)} = VolP(v_1, v_2, v_3) = 4$ (из
              условий задачи) $\Rightarrow detG(v_1, v_2, v_3) = 16$. 
              \item Хотим найти: $VolP(v_1', v_2', v_3') = \sqrt{detG(v_1', v_2', v_3')} \Rightarrow$
              хотим найти $detG(v_1', v_2', v_3')$.
              \item Для краткости обозначим $G := G(v_1, v_2, v_3)$, $G' := G(v_1', v_2', v_3')$.
              \item Утверждение: Пусть $u_1, \ldots, u_k$ -- некоторая система векторов 
              евклидова пространства $E$, и пусть $G$ -- её матрица Грамма. Предположим, что
              векторы $u_1', \ldots, u_l' \in E$ выражаются через $u_1, \ldots, u_k$; тогда
              можно записать $(u_1', \ldots, u_l') = (u_1, \ldots, u_k)C$ для некоторой матрицы
              $C \in Mat_{k \times l}(\mathbb{R})$. Тогда матрица Грамма системы векторов
              $(u_1', \ldots, u_l')$ равна $C^TGC$. \\
              
              Докажем: $\displaystyle (u_i', u_j') = (\sum_{p = 1}^k c_{pi}v_p, \sum_{q = 1}^k c_{qj}v_q) =
              \sum_{p = 1}^k \sum_{q = 1}^k c_{pi} (v_p, v_q) c_{qj} =
               \sum_{p = 1}^k \sum_{q = 1}^k [C^T]_{ip} G_{pq} c_{qj} \Rightarrow
                G(u_1', \ldots, u_l') = C^TGC$.
              \item Хочу отдельно обратить ваше внимание на то, что ни первая ни вторая система не
              обязаны быть линейно независимыми. То есть, мы буквально никаких условий на векторы
              системы не накладываем (кроме, конечно, выразимости одной через другую). 
              \item Докажем же, что все векторы $v_1', v_2', v_3'$ выражаются через 
              $v_1, v_2, v_3$.
              \item Уравнение: $(v_1', v_2', v_3') = (v_1, v_2, v_3)C$ -- обычное матричное уравнение,
              где неизвестной является матрица $C$. Решим это методом Гаусса. 
              \item Координаты векторы обеих систем я буду записывать в базисе $(1, x, x^2)$.
              \item Решаем: \\
              $\smt{rrr|rrr}{
                2 & 3 & 1 & 1 & 3 &-1\\
                -1 & 1 & 3  & 1& -2 & 2\\
                3 & 1 & -3 & 1& 3& -2
              }
             \rightsquigarrow 
             \smt{rrr|rrr}{
              1 & -1 & -3  & -1& 2 & -2\\
              2 & 3 & 1 & 1 & 3 &-1\\
              3 & 1 & -3 & 1& 3& -2
            }
            \rightsquigarrow 
            \smt{rrr|rrr}{
              1 & -1 & -3  & -1& 2 & -2\\
             0 & 5 & 7 & 3 & -1 &3\\
             0 & 4 & 6 & 4& -3& 4
           }
           \rightsquigarrow  \\ 
           \rightsquigarrow
           \smt{rrr|rrr}{
             1 & -1 & -3  & -1& 2 & -2\\
            0 & 1 & 1 & -1 & 2 &-1\\
            0 & 4 & 6 & 4& -3& 4
          }
          \rightsquigarrow 
          \smt{rrr|rrr}{
            1 & -1 & -3  & -1& 2 & -2\\
           0 & 1 & 1 & -1 & 2 &-1\\
           0 & 0 & 2 & 8& -11& 8
         }
         \rightsquigarrow 
         \smt{rrr|rrr}{
           1 & -1 & -3  & -1& 2 & -2\\
          0 & 1 & 1 & -1 & 2 &-1\\
          0 & 0 & 1 & 4& -\dfrac{11}{2}& 4}
          \rightsquigarrow \\     \rightsquigarrow
          \smt{rrr|rrr}{
            1 & -1 & 0  & 11& -\dfrac{29}{2} & 10\\
           0 & 1 & 0 & -5 & \dfrac{15}{2} &-5\\
           0 & 0 & 1 & 4& -\dfrac{11}{2}& 4}
             \rightsquigarrow 
             \smt{rrr|rrr}{
               1 & 0 & 0  & 6& -7 & 5\\
              0 & 1 & 0 & -5 &  \dfrac{15}{2} &-5\\
              0 & 0 & 1 & 4& -\dfrac{11}{2}& 4}
                $
            \item Заметим, что решение существует. Матрица справа от черты и есть наша
            искомая матрица $C$.
            \item $G' = C^TGC \Rightarrow$  $
            detG' = det(C^TGC) = $ (по свойству определителя) $= det(C^T)det(G)det(C) =$ 
            (по свойству определителя) $ = det(C)^2det(G)$.
            \item $detG$ мы нашли в самом начале решения, осталось найти $det(C)$ (с вашего позволения,
            упущу счёт). $detC = \dfrac{5}{2}$.
            \item Итого: $detG' = \dfrac{25}{4} \cdot 16 = 100 \Rightarrow
            VolP(v_1','v_2','v_3') = \sqrt{100} = 10$. 
          \end{itemize}
        \end{proof}

    \section*{Задача \textnumero 6}
        % текст задачи
        Прямая \dots
        \begin{proof}[Решение.]
            
        \end{proof}

\end{document}