\documentclass[a4paper]{article}
\usepackage{header}
\newcommand{\Mat}{\operatorname{Mat}}
\newcommand{\M}{\operatorname{M}}
\allowdisplaybreaks

%% Графика
\usepackage{graphicx}       
\graphicspath{{images/}}            
\usepackage{tikz}  
\usetikzlibrary{patterns}                 
\usepackage{pgfplots}              
\usepackage{circuitikz}
\usepackage{bm}
            

\usetikzlibrary{quotes,angles}
\usetikzlibrary{positioning,intersections}
\usetikzlibrary{through}

\textwidth=19.0cm \oddsidemargin=-1.3cm
\textheight=25cm \topmargin=-2.5cm

\newtheorem{task}{Задание}
\newtheorem*{task*}{Задание}

\theoremstyle{remark}

\newtheorem{remark}{Замечание}
\newtheorem*{remark*}{Замечание}
\newtheorem{commentarium}{Комментарий}
\newtheorem*{commentarium*}{Комментарий}

\usepackage{tikz-cd}

\newenvironment{sysmatrix}[1]
{
    \left(\begin{array}{@{}#1@{}}
}
{\end{array}\right)}
\newcommand{\smt}[2]{\begin{sysmatrix}{#1} #2\end{sysmatrix}}
\newcommand{\eq}[1]{\begin{cases} #1 \end{cases}}

\newcommand{\elon}[3]{%
  \ensuremath{\text{Э}_1(#1,\; #2,\; #3)}%
}
\newcommand{\eltw}[2]{%
  \ensuremath{\text{Э}_2(#1,\; #2)}%
}
\newcommand{\elth}[2]{%
  \ensuremath{\text{Э}_3(#1,\; #2)}%
}
 
\newcommand{\eqon}[3]{%
  \ensuremath{\overset{\text{Э}_1(#1,\; #2,\; #3)}{=}}%
}
\newcommand{\eqtw}[2]{%
  \ensuremath{\overset{\text{Э}_2(#1,\; #2)}{=}}%
}
\newcommand{\eqth}[2]{%
  \ensuremath{\overset{\text{Э}_3(#1,\; #2)}{=}}%
}

\newcommand{\arron}[3]{%
  \ensuremath{\xrightarrow{\text{Э}_1(#1,\; #2,\; #3)}}%
}
\newcommand{\arrtw}[2]{%
  \ensuremath{\xrightarrow{\text{Э}_2(#1,\; #2)}}%
}
\newcommand{\arrth}[2]{%
  \ensuremath{\xrightarrow{\text{Э}_3(#1,\; #2)}}%
}

\makeatletter
\renewcommand*\env@matrix[1][*\c@MaxMatrixCols c]{%
  \hskip -\arraycolsep
  \let\@ifnextchar\new@ifnextchar
  \array{#1}}
\makeatother

% tasks here https://www.dropbox.com/s/8yhwwpf3dvft1yd/LA_19-20_Stream2_Test2.pdf?dl=0
\title{Весенняя контрольная работа учебного года 2019/2020\\Вариант \textnumero 1}
\author{	
  % ОТКОММЕНТИРУЙ СЕБЯ
  Алиса Вернигор       \\ \href{https://t.me/allisyonok}{Telegram} \and
	Сергей Лоптев        \\ \href{https://t.me/beast_sl}{Telegram} \and
	Оля Козлова        \\ \href{https://t.me/grenlayk}{Telegram}
}

\date{}

\begin{document}
	\maketitle
    \section*{Задача \textnumero 1}
        Рассмотрим линейное отображение $\varphi: \RR^4 \to \RR^3$, $v \mapsto Av$, где
        \begin{equation*}
          A = \begin{pmatrix}
            3 & -1 & 3 & 7 \\
            2 & -1 & 1 & 4 \\
            -1 & 3 & 7 & 3
          \end{pmatrix}.
        \end{equation*}
        Найдите базис $\mathbbm{e}$ пространства $\RR^4$ и базис $\mathbbm{f}$ пространства $\RR^3$, в которых $\varphi$ имеет диагональный вид с единицами и нулями на диагонали, и выпишите этот вид.
        \begin{proof}[Решение.]
        Будем пользоваться стандартным алгоритмом. Найдём сначала базис $\ker \varphi$~--- это ФСР ОСЛУ $Ax = 0$. Приводим $A$ к улучшенному ступенчатому виду:
        \begin{multline*}
          \begin{pmatrix}
            3 & -1 & 3 & 7 \\
            2 & -1 & 1 & 4 \\
            -1 & 3 & 7 & 3
          \end{pmatrix} \overset{\elon{2}{3}{2}}{\arron{1}{3}{3}} \begin{pmatrix}
            0 & 8 & 24 & 16 \\
            0 & 5 & 15 & 10 \\
            -1 & 3 & 7 & 3 
          \end{pmatrix} \arron{1}{2}{-8/5} \begin{pmatrix}
            0 & 0 & 0 & 0 \\
            0 & 5 & 15 & 10 \\
            -1 & 3 & 7 & 3 
          \end{pmatrix} \overset{\elth{2}{1/5}}{\arrtw{1}{3}} \begin{pmatrix}
            -1 & 3 & 7 & 3 \\
            0 & 1 & 3 & 2 \\
            0 & 0 & 0 & 0
          \end{pmatrix} \\ \arron{1}{2}{-3} \begin{pmatrix}
            -1 & 0 & -2 & -3 \\
            0 & 1 & 3 & 2 \\
            0 & 0 & 0 & 0
          \end{pmatrix} \arrth{1}{-1} \begin{pmatrix}
            1 & 0 & 2 & 3 \\
            0 & 1 & 3 & 2 \\
            0 & 0 & 0 & 0
          \end{pmatrix}
        \end{multline*}
        Тогда искомая ФСР~--- векторы
        \begin{equation*}
          e_3 = \begin{pmatrix}
            -2 \\
            -3 \\
            1 \\
            0
          \end{pmatrix}, \quad e_4 = \begin{pmatrix}
            -3 \\
            -2 \\
            0 \\ 
            1
          \end{pmatrix}
        \end{equation*}
        Очевидно, что эту систему векторов можно дополнить до базиса $\RR^4$ векторами 
        \begin{equation*}
          e_1 = \begin{pmatrix}
            1 \\
            0 \\
            0 \\
            0
          \end{pmatrix}, \quad e_2 = \begin{pmatrix}
            0 \\
            1 \\
            0 \\ 
            0
          \end{pmatrix}
        \end{equation*}
        Теперь заметим, что (по определению матрицы линейного отображения) $Ae_1 = A^{(1)},\, Ae_2 = A^{(2)}$. Тогда скажем, что $f_1 = A^{(1)},\, f_2 = A^{(2)}$~--- это будет базисом $\Im \varphi$. Осталось дополнить $f_1,\, f_2$ до базиса $\RR^3$:
        \begin{equation*}
          \begin{pmatrix}
            3 & 2 & -1 \\
            -1 & -1 & 3
          \end{pmatrix} \arron{1}{2}{3}
          \begin{pmatrix}
            0 & -1 & 8 \\
            -1 & -1 & 3
          \end{pmatrix} \arrtw{1}{2}
          \begin{pmatrix}
            -1 & -1 & 3 \\ 
            0 & -1 & 8
          \end{pmatrix}
        \end{equation*}
        Видим, что вектор $
          f_3 = \begin{pmatrix}
            0 \\
            0 \\
            1  
          \end{pmatrix}$ дополняет $f_1,\, f_2$ до базиса $\RR^3$. Тогда по доказанному на лекции матрица линейного отображения $\varphi$ в базисах $\mathbbm{e},\, \mathbbm{f}$ принимает вид:
          \begin{equation*}
            A(\varphi,\, \mathbbm{e},\, \mathbbm{f}) = \begin{pmatrix}
              1 & 0 & 0 & 0\\
              0 & 1 & 0 & 0 \\
              0 & 0 & 0 & 0
            \end{pmatrix}.
          \end{equation*}
        \end{proof}	 
    
    \section*{Задача \textnumero 2}
        % текст задачи
        Пусть \dots
        \begin{proof}[Решение.]

        \end{proof}
    
    \section*{Задача \textnumero 3}
        % текст задачи
        Даны две квадратичные формы 
        \[ 
            Q_1(x_1, x_2, x_3) = 5x_1^2 - x_2^2 - 2x_1x_2 - 4x_1x_3, \
            Q_2(x_1, x_2, x_3) = (x_1 - x_2 - x_3)^2. 
        \]
         Определите нормальный вид квадратичной формы $Q_1 + aQ_2$ в зависимости 
         от значения параметра $a$.
        \begin{proof}[Решение.] \ 
            Первое, что хочется сделать --- просто подставить, 
            раскрыть все скобки и применить Якоби. Делать я так конечно же не буду. Получилось бы долго, 
            с кучей арифметики и частных случаев. 

            1. Вообще в таких задачах с параметром нам обычно хочется, чтобы параметр 
            встречался как можно реже. Сейчас он встречается всего 1 раз, давайте так и оставим --- сделаем 
            следующую замену: 
            \[
                \eq{
                    y_1 = x_1\\ 
                    y_2 = x_2 \\ 
                    y_3 = x_1 - x_2 - x_3
                } 
                \Rightarrow
                \eq{
                    x_1 = y_1\\ 
                    x_2 = y_2 \\ 
                    x_3 = y_1 - y_2 - y_3
                }
            \]
            Именно так мне захотелось сделать, потому что данная нам квадратичная форма $Q_1 + aQ_2$ 
            уж очень напоминает вид, полученный в процессе метода Лагранжа --- 
            несколько переменных собраны внутри квадрата ($Q_2$), а переменной $x_3$ в квадрате (в явном виде) нет.
            
            2. Введу обозначение: $Q = Q_1 + aQ_2$, тогда 
            \begin{multline*}
                Q(y_1, y_2, y_3) = 
                5y_1^2 - y_2^2 - 2y_1y_2 - 4y_1(y_1 - y_2 - y_3) + ay_3^2 = \\ =
                5y_1^2 - y_2^2 - 2y_1y_2 - 4y_1^2 + 4y_1y_2 + 4y_1y_3 + ay_3^2 = 
                y_1^2 - y_2^2 + 2y_1y_2 + 4y_1y_3 + ay_3^2
            \end{multline*}

            3. А вот теперь можно и метод Якоби применить. 
            \[
                B(Q,\bm{e}) = 
                \begin{pmatrix}
                    1 & 1 & 2\\
                    1 & -1 & 0 \\
                    2 & 0 & a
                \end{pmatrix} 
                \Rightarrow 
                \eq{
                    \delta_1 = 1\\
                    \delta_2 = -1 - 1 = -2 \\
                    \delta_3 = -a + 0  + 0 - (-4) - 0 - a = -2a + 4
                } 
                \Rightarrow 
                \eq{
                    \delta_1 = 1\\
                    \frac{\delta_2}{\delta_1} = -2 \\
                    \frac{\delta_3}{\delta_2} = a - 2
                }
            \]
            Откуда получаем \textbf{ответ}: 
            \[
                Q(z_1, z_2, z_3) = \eq{
                    z_1^2 - z_2^2 - z_3^2, \ a < 2\\
                    z_1^2 - z_2^2, \ a = 2\\
                    z_1^2 - z_2^2 + z_3^2, \ a > 2\\}
            \] 
            (z --- координаты вектора в базисе, в котором $Q$ 
            принимает нормальный вид)

        \end{proof}
      
    \section*{Задача \textnumero 4}
        В евклидовом пространстве $\RR^4$ даны два подпространства $U = \langle u_1,\, u_2 \rangle$ и $W = \langle w_1,\, w_2 \rangle$, где $u_1 = (2, -1, 2, -1),\, u_2 = (3, -3, 1, 1), w_1 = (1, 2, -1, 2),\, w_2 = (1, -3, 3, -1)$. Найдите вектор $v \in \RR^4$, для которого $\pr_U v = (9, -12, -1, 8)$ и $\ort_W v = (1, -8, -7, 4)$.
        \begin{proof}[Решение.]
        Для начала найдём базис $U^\bot$~--- это ФСР ОСЛУ $\begin{pmatrix}
          u_1 \\
          u_2
        \end{pmatrix}x = 0$. Применяем Гаусса:
        \begin{multline*}
          \begin{pmatrix}
            2 & -1 & 2 & -1 \\
            3 & -3 & 1 & 1
          \end{pmatrix} \arrth{2}{2} 
          \begin{pmatrix}
            2 & -1 & 2 & -1 \\
            6 & -6 & 2 & 2
          \end{pmatrix} \arron{2}{1}{-3}
          \begin{pmatrix}
            2 & -1 & 2 & -1 \\
            0 & -3 & -4 & 5
          \end{pmatrix} \arrth{1}{3}
          \begin{pmatrix}
            6 & -3 & 6 & -3 \\
            0 & -3 & -4 & 5
          \end{pmatrix} \\ \arron{1}{2}{-1}
          \begin{pmatrix}
            6 & 0 & 10 & -8 \\
            0 & -3 & -4 & 5
          \end{pmatrix} \overset{\elth{2}{-1/3}}{\arrth{1}{1/6}}
          \begin{pmatrix}
            1 & 0 & 5/3 & -4/3 \\
            0 & 1 & 4/3 & -5/3
          \end{pmatrix}
        \end{multline*}
        Тогда ФСР~--- это векторы
        \begin{equation*}
          u_1^\bot = \begin{pmatrix}
            -5 \\
            -4 \\
            3 \\
            0
          \end{pmatrix}, \quad u_2^\bot = \begin{pmatrix}
            4 \\
            5 \\
            0 \\
            3
          \end{pmatrix}.
        \end{equation*}
        Тогда знаем, что $\exists a,\, b,\, c,\, d \in \RR$, такие что $\begin{dcases}
          v = \pr_U v + au_1^\bot + bu_2^\bot \\
          v = \ort_W v + cw_1 + dw_2
        \end{dcases}$.
        Вычтем второе уравнение из первого и получим, что $\ort_W v - \pr_U v = au_1^\bot + bu_2^\bot - cw_1 - dw_2$. Посчитаем $\ort_W v - \pr_U v$:
        \begin{equation*}
          \begin{pmatrix}
            1 & -8 & -7 & 4
          \end{pmatrix} - \begin{pmatrix}
            9 & -12 & -1 & 8
          \end{pmatrix} = \begin{pmatrix}
            -8 & 4 & -6 & -4
          \end{pmatrix}.
        \end{equation*}
        Теперь решаем СЛУ:
        \begin{multline*}
          \smt{rrrr|r} {
            -5 & 4 & -1 & -1 & -8 \\ 
            -4 & 5 & -2 & 3 & 4 \\ 
            3 & 0 & 1 & -3 & -6 \\ 
            0 & 3 & -2 & 1 & -4 \\ 
          } \overset{\elth{2}{3}}{\arrth{1}{3}} 
          \smt{rrrr|r} {
            -15 & 12 & -3 & -3 & -24 \\ 
            -12 & 15 & -6 & 9 & 12 \\ 
            3 & 0 & 1 & -3 & -6 \\ 
            0 & 3 & -2 & 1 & -4 \\ 
          } \overset{\elon{2}{3}{4}}{\arron{1}{3}{5}}
          \smt{rrrr|r} {
            0 & 12 & 2 & -18 & -54 \\ 
            0 & 15 & -2 & -3 & -12 \\ 
            3 & 0 & 1 & -3 & -6 \\ 
            0 & 3 & -2 & 1 & -4 \\ 
          } \\ \overset{\elon{1}{4}{-4}}{\arron{2}{4}{-5}}
          \smt{rrrr|r} {
            0 & 0 & 10 & -22 & -38 \\ 
            0 & 0 & 8 & -8 & 8 \\ 
            3 & 0 & 1 & -3 & -6 \\ 
            0 & 3 & -2 & 1 & -4 \\ 
          } \arrth{2}{1/8}
          \smt{rrrr|r} {
            0 & 0 & 10 & -22 & -38 \\ 
            0 & 0 & 1 & -1 & 1 \\ 
            3 & 0 & 1 & -3 & -6 \\ 
            0 & 3 & -2 & 1 & -4 \\ 
          } \overset{\elon{4}{2}{2}}{\overset{\elon{3}{2}{-1}}{\arron{1}{2}{-10}}}
          \smt{rrrr|r} {
            0 & 0 & 0 & -12 & -48 \\ 
            0 & 0 & 1 & -1 & 1 \\ 
            3 & 0 & 0 & -2 & -7 \\ 
            0 & 3 & 0 & -1 & -2 \\ 
          } \\ \overset{\eltw{3}{4}}{\arrth{1}{-1/12}}
          \smt{rrrr|r} {
            0 & 0 & 0 & 1 & 4 \\ 
            0 & 0 & 1 & -1 & 1 \\ 
            0 & 3 & 0 & -1 & -2 \\ 
            3 & 0 & 0 & -2 & -7 \\ 
          } \overset{\elon{4}{1}{2}}{\overset{\elon{3}{1}{1}}{\arron{2}{1}{1}}}
          \smt{rrrr|r} {
            0 & 0 & 0 & 1 & 4 \\ 
            0 & 0 & 1 & 0 & 5 \\ 
            0 & 3 & 0 & 0 & 2 \\ 
            3 & 0 & 0 & 0 & 1 \\ 
          } \overset{\eltw{1}{4}}{\arrtw{2}{3}}
          \smt{rrrr|r} {
            3 & 0 & 0 & 0 & 1 \\ 
            0 & 3 & 0 & 0 & 2 \\ 
            0 & 0 & 1 & 0 & 5 \\ 
            0 & 0 & 0 & 1 & 4 \\ 
          }
        \end{multline*}
        Отсюда уже видно, что решение уравнения: $
          \begin{dcases}
            a = 1/3 \\
            b = 2/3 \\
            -c = 5 \\
            -d = 4
          \end{dcases}$. Осталось только найти $v$. $c$ и $d$ получились целые, поэтому найдём по формуле $v = \ort_W v - cw_1 - dw_2$:
          \begin{equation*}
            \bm{v} = \begin{pmatrix}
              1 & -8 & -7 & 4
            \end{pmatrix} + 5\begin{pmatrix}
              1 & 2 & -1 & 2
            \end{pmatrix} + 4\begin{pmatrix}
              1 & -3 & 3 & -1
            \end{pmatrix} = \begin{pmatrix}
              \bm{10} & \bm{-10} & \bm{0} & \bm{10}
            \end{pmatrix}.
          \end{equation*}
        \end{proof}
    
    \section*{Задача \textnumero 5}
        В пространстве \dots
        \begin{proof}[Решение.] \ 

        \end{proof}

    \section*{Задача \textnumero 6}
        % текст задачи
        Прямая $l \subset \mathbb{R}^3$ проходит через точку $P = (2, 0, 1)$, 
        пересекает прямую $l_1 = \{4x - 3z = 1, x + y = -2\}$ и перпендикулярна прямой 
        $l_2 = \{x + 2y - 3z = 5, y - z = 2\}$. Найдите расстояние между прямыми $l$ и $l_2$.
        \begin{proof}[Решение.]
            В задачах на геометрию бывает полезно порисовать картинки --- 
            попробовать представить, что вообще происходит в задаче. Однако здесь их не будет, 
            потому что мне лень рисовать красивые картинки --- но мы и так справимся, честно!
            \begin{enumerate}
                \item Для начала зададим прямую $l_2$ с помощью параметрического уравнения: 
                \[
                    \smt{rrr|r} {
                        1 & 2 & -3 & 5 \\
                        0 & 1 & -1 & 2 \\
                    } \arron{1}{2}{-2}
                    \smt{rrr|r} {
                        1 & 0 & -1 & 1 \\
                        0 & 1 & -1 & 2 \\
                    }
                    \Rightarrow 
                    \eq{
                        x = t + 1\\
                        y = t + 2 \\
                        z = t + 0
                    }
                    \Rightarrow 
                    \eq {
                        v_2 = (1, 2, 0) \\
                        a_2 = (1, 1, 1)
                    }
                \]
                \item Так как  $l \perp l_2$, то $l \subset S$, где $S$ -- плоскость, 
                перпендикулярная $l_2$. Найдем уравнение $S$ по нормали $a_2$ и точке P (через
                 которую эта плоскость тоже должна проходить):
                \[
                    1(x - 2) + 1(y - 0) + 1(z - 1) = 0 \Rightarrow
                    x + y + z = 3
                \]
                \item Так как $l \subset S$ и $l \cap l_1$, то $l$ проходит через точку 
                пересечения плоскости $S$ и прямой $l_1$. Найдем эту точку (обозначу ее $q$) -- объединим 
                уравнения прямой и плоскости в одну систему:
                \[
                    \eq{
                        4x - 3z = 1\\
                        x + y = -2 \\
                        x + y + z = 3
                    }
                \]
                \begin{multline*}
                    \smt{rrr|r} {
                        4 & 0 & -3 & 1 \\
                        1 & 1 & 0 & -2 \\
                        1 & 1 & 1 & 3 \\
                    } \arrtw{1}{2}
                    \smt{rrr|r} {
                        1 & 1 & 0 & -2 \\
                        4 & 0 & -3 & 1 \\
                        1 & 1 & 1 & 3 \\
                    } \overset{\elon{3}{1}{-1}}{\arron{2}{1}{-4}}
                    \smt{rrr|r} {
                        1 & 1 & 0 & -2 \\
                        0 & -4 & -3 & 9 \\
                        0 & 0 & 1 & 5 \\
                    } \arron{2}{3}{3} \\
                    \smt{rrr|r} {
                        1 & 1 & 0 & -2 \\
                        0 & -4 & 0 & 24 \\
                        0 & 0 & 1 & 5 \\
                    } \arrth{2}{-4}
                    \smt{rrr|r} {
                        1 & 1 & 0 & -2 \\
                        0 & 1 & 0 & -6 \\
                        0 & 0 & 1 & 5 \\
                    } \arron{1}{2}{-}
                    \smt{rrr|r} {
                        1 & 0 & 0 & 4 \\
                        0 & 1 & 0 & -6 \\
                        0 & 0 & 1 & 5 \\
                    }
                \end{multline*}
                $\Rightarrow q = (4, -6, 5)$
                \item Найдем параметрическое уравнение $l$ по точке $P$ и $q$:
                \[
                    l \colon \eq{
                        v = P = (2, 0, 1)\\
                        a = q - P = (4, -6, 5) -  (2, 0, 1) = (2, -6, 4)
                    }
                \]
                \item Осталось только найти расстояние между прямыми $l$ и $l_2$, 
                а как это сделать  мы знаем по формуле с лекции: 
                \[
                    \rho(l, l_2) = \frac{\| (v_2 - v, a_2, a)\|}{\| [a_2, a]\|}  
                \]
                Аккуратно все считаем: 
                \begin{itemize} 
                    \item $v_2 - v = (1, 2, 0) - (2, 0, 1) = (-1, 2, -1)$
                    \item Сначала сделаю несколько элементарных преобразований, а потом воспользуюсь 
                    формулой определителя 3 порядка. 

                    $(v_2 - v, a_2, a) = 
                    \begin{vmatrix}
                        -1 & 2 & -1\\
                        1 & 1 & 1\\
                        2 & -6 & 4
                    \end{vmatrix} = 
                    2
                    \begin{vmatrix}
                        -1 & 2 & -1\\
                        1 & 1 & 1\\
                        1 & -3 & 2
                    \end{vmatrix} = 
                    2
                    \begin{vmatrix}
                        0 & 3 & 0\\
                        1 & 1 & 1\\
                        1 & -3 & 2
                    \end{vmatrix} =
                    2
                    \begin{vmatrix}
                        0 & 3 & 0\\
                        1 & 1 & 1\\
                        1 & 0 & 2
                    \end{vmatrix}
                    = 2 \cdot (3 - 6) = 2 \cdot (-3) = -6$
                    \item $[a_2, a] = 
                    \begin{vmatrix}
                        e_1 & e_2 & e_3\\
                        1 & 1 & 1\\
                        2 & -6 & 4
                    \end{vmatrix} = 
                    \begin{vmatrix}
                        1 & 1\\
                        -6 & 4
                    \end{vmatrix}e_1 - 
                    \begin{vmatrix}
                        1 & 1\\
                        2 & 4
                    \end{vmatrix}e_2 +
                    \begin{vmatrix}
                        1 & 1\\
                        2 & -6
                    \end{vmatrix}e_3 = 
                    10e_1 - 2e_2 + (-8)e_3 = (10, -2, -8)$
                    \item $\| [a_2, a]\| = \| (10, -2, -8)\| = \sqrt{10^2 + (-2)^2 + (-8)^2} = 
                    \sqrt{168} = 2\sqrt{42}$
                \end{itemize}
                Вот мы наконец посчитали все что нам надо, найдем \textbf{ответ}: 
                \[
                    \rho(l, l_2) = \frac{\| -6\|}{2\sqrt{42}} = \frac{6}{2\sqrt{42}} = \frac{3}{\sqrt{42}}
                \]
            \end{enumerate}
        \end{proof}

\end{document}
