\documentclass[a4paper]{article}
\usepackage{header}
\newcommand{\Mat}{\operatorname{Mat}}
\newcommand{\M}{\operatorname{M}}
\allowdisplaybreaks

%% Графика
\usepackage{graphicx}       
\graphicspath{{images/}}            
\usepackage{tikz}  
\usetikzlibrary{patterns}                 
\usepackage{pgfplots}              
\usepackage{circuitikz}
\usepackage{bm}
            

\usetikzlibrary{quotes,angles}
\usetikzlibrary{positioning,intersections}
\usetikzlibrary{through}

\textwidth=19.0cm \oddsidemargin=-1.3cm
\textheight=25cm \topmargin=-2.5cm

\newtheorem{task}{Задание}
\newtheorem*{task*}{Задание}

\theoremstyle{remark}

\newtheorem{remark}{Замечание}
\newtheorem*{remark*}{Замечание}
\newtheorem{commentarium}{Комментарий}
\newtheorem*{commentarium*}{Комментарий}

\usepackage{tikz-cd}

\newenvironment{sysmatrix}[1]
{
    \left(\begin{array}{@{}#1@{}}
}
{\end{array}\right)}
\newcommand{\smt}[2]{\begin{sysmatrix}{#1} #2\end{sysmatrix}}
\newcommand{\eq}[1]{\begin{cases} #1 \end{cases}}

\newcommand{\elon}[3]{%
  \ensuremath{\text{Э}_1(#1,\; #2,\; #3)}%
}
\newcommand{\eltw}[2]{%
  \ensuremath{\text{Э}_2(#1,\; #2)}%
}
\newcommand{\elth}[2]{%
  \ensuremath{\text{Э}_3(#1,\; #2)}%
}
 
\newcommand{\eqon}[3]{%
  \ensuremath{\overset{\text{Э}_1(#1,\; #2,\; #3)}{=}}%
}
\newcommand{\eqtw}[2]{%
  \ensuremath{\overset{\text{Э}_2(#1,\; #2)}{=}}%
}
\newcommand{\eqth}[2]{%
  \ensuremath{\overset{\text{Э}_3(#1,\; #2)}{=}}%
}

\newcommand{\arron}[3]{%
  \ensuremath{\xrightarrow{\text{Э}_1(#1,\; #2,\; #3)}}%
}
\newcommand{\arrtw}[2]{%
  \ensuremath{\xrightarrow{\text{Э}_2(#1,\; #2)}}%
}
\newcommand{\arrth}[2]{%
  \ensuremath{\xrightarrow{\text{Э}_3(#1,\; #2)}}%
}

\makeatletter
\renewcommand*\env@matrix[1][*\c@MaxMatrixCols c]{%
  \hskip -\arraycolsep
  \let\@ifnextchar\new@ifnextchar
  \array{#1}}
\makeatother

% tasks here https://www.dropbox.com/s/8yhwwpf3dvft1yd/LA_19-20_Stream2_Test2.pdf?dl=0
\title{Весенняя контрольная работа учебного года 2019/2020\\Вариант \textnumero 1}
\author{	
  % ОТКОММЕНТИРУЙ СЕБЯ
  Алиса Вернигор       \\ \href{https://t.me/allisyonok}{Telegram} \and
	Сергей Лоптев        \\ \href{https://t.me/beast_sl}{Telegram} \and
	Оля Козлова        \\ \href{https://t.me/grenlayk}{Telegram}
}

\date{}

\begin{document}
	\maketitle
    \section*{Задача \textnumero 1}
        % текст задачи
        Рассмотрим \dots
        \begin{proof}[Решение.]
                
        \end{proof}	 
    
    \section*{Задача \textnumero 2}
        % текст задачи
        Пусть \dots
        \begin{proof}[Решение.]

        \end{proof}
    
    \section*{Задача \textnumero 3}
        % текст задачи
        Даны две квадратичные формы 
        \[ 
            Q_1(x_1, x_2, x_3) = 5x_1^2 - x_2^2 - 2x_1x_2 - 4x_1x_3, \
            Q_2(x_1, x_2, x_3) = (x_1 - x_2 - x_3)^2. 
        \]
         Определите нормальный вид квадратичной формы $Q_1 + aQ_2$ в зависимости 
         от значения параметра $a$.
        \begin{proof}[Решение.] \ 
            Первое, что хочется сделать --- просто подставить, 
            раскрыть все скобки и применить Якоби. Делать я так конечно же не буду. Получилось бы долго, 
            с кучей арифметики и частных случаев. 

            Вообще в таких задачах с параметром нам обычно хочется, чтобы параметр 
            встречался как можно реже. Сейчас он встречается всего 1 раз, давайте так и оставим --- сделаем 
            следующую замену: 
            \[
                \eq{
                    y_1 = x_1\\ 
                    y_2 = x_2 \\ 
                    y_3 = x_1 - x_2 - x_3
                } 
                \Rightarrow
                \eq{
                    x_1 = y_1\\ 
                    x_2 = y_2 \\ 
                    x_3 = y_1 - y_2 - y_3
                }
            \]
            Именно так мне захотелось сделать, потому что данная нам квадратичная форма $Q_1 + aQ_2$ 
            уж очень напоминает вид, полученный в процессе метода Лагранжа --- 
            несколько переменных собраны внутри квадрата ($Q_2$), а переменной $x_3$ в квадрате (в явном виде) нет.
            
            Введу обозначение: $Q = Q_1 + aQ_2$, тогда 
            \begin{multline*}
                Q(y_1, y_2, y_3) = 
                5y_1^2 - y_2^2 - 2y_1y_2 - 4y_1(y_1 - y_2 - y_3) + ay_3^2 = \\ =
                5y_1^2 - y_2^2 - 2y_1y_2 - 4y_1^2 + 4y_1y_2 + 4y_1y_3 + ay_3^2 = 
                y_1^2 - y_2^2 + 2y_1y_2 + 4y_1y_3 + ay_3^2
            \end{multline*}

            А вот теперь можно и метод Якоби применить. 
            \[
                B(Q,\bm{e}) = 
                \begin{pmatrix}
                    1 & 1 & 2\\
                    1 & -1 & 0 \\
                    2 & 0 & a
                \end{pmatrix} 
                \Rightarrow 
                \eq{
                    \delta_1 = 1\\
                    \delta_2 = -1 - 1 = -2 \\
                    \delta_3 = -a + 0  + 0 - (-4) - 0 - a = -2a + 4
                } 
                \Rightarrow 
                \eq{
                    \delta_1 = 1\\
                    \frac{\delta_2}{\delta_1} = -2 \\
                    \frac{\delta_3}{\delta_2} = a - 2
                }
            \]
            Откуда получаем \textbf{ответ}: 
            \[
                Q(z_1, z_2, z_3) = \eq{
                    z_1^2 - z_2^2 - z_3^2, \ a < 2\\
                    z_1^2 - z_2^2, \ a = 2\\
                    z_1^2 - z_2^2 + z_3^2, \ a > 2\\}
            \] 
            (z --- координаты вектора в базисе, в котором $Q$ 
            принимает нормальный вид)

        \end{proof}
      
    \section*{Задача \textnumero 4}
        % текст задачи
        В евклидовом \dots
        \begin{proof}[Решение.]
            
        \end{proof}
    
    \section*{Задача \textnumero 5}
        В пространстве \dots
        \begin{proof}[Решение.] \ 

        \end{proof}

    \section*{Задача \textnumero 6}
        % текст задачи
        Прямая \dots
        \begin{proof}[Решение.]
            
        \end{proof}

\end{document}