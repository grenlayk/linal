\documentclass[a4paper]{article}
\usepackage{header}
\newcommand{\Mat}{\operatorname{Mat}}
\newcommand{\M}{\operatorname{M}}
\allowdisplaybreaks

%% Графика
\usepackage{graphicx}       
\graphicspath{{images/}}            
\usepackage{tikz}  
\usetikzlibrary{patterns}                 
\usepackage{pgfplots}              
\usepackage{circuitikz}
\usepackage{nicefrac}
\usepackage{bm}
            
\usepackage{titlesec}
\titlespacing*{\section}{0pt}{0pt}{0pt}

\usetikzlibrary{quotes,angles}
\usetikzlibrary{positioning,intersections}
\usetikzlibrary{through}

\textwidth=19.0cm \oddsidemargin=-1.3cm
\textheight=26cm \topmargin=-2.5cm

\newtheorem{task}{Задание}
\newtheorem*{task*}{Задание}

\theoremstyle{remark}

\newtheorem{remark}{Замечание}
\newtheorem*{remark*}{Замечание}
\newtheorem{commentarium}{Комментарий}
\newtheorem*{commentarium*}{Комментарий}

\usepackage{tikz-cd}

\newenvironment{sysmatrix}[1]
{
    \left(\begin{array}{@{}#1@{}}
}
{\end{array}\right)}
\newcommand{\smt}[2]{\begin{sysmatrix}{#1} #2\end{sysmatrix}}
\newcommand{\eq}[1]{\begin{cases} #1 \end{cases}}

\newcommand{\elon}[3]{%
  \ensuremath{\text{Э}_1(#1,\; #2,\; #3)}%
}
\newcommand{\eltw}[2]{%
  \ensuremath{\text{Э}_2(#1,\; #2)}%
}
\newcommand{\elth}[2]{%
  \ensuremath{\text{Э}_3(#1,\; #2)}%
}
 
\newcommand{\eqon}[3]{%
  \ensuremath{\overset{\text{Э}_1(#1,\; #2,\; #3)}{=}}%
}
\newcommand{\eqtw}[2]{%
  \ensuremath{\overset{\text{Э}_2(#1,\; #2)}{=}}%
}
\newcommand{\eqth}[2]{%
  \ensuremath{\overset{\text{Э}_3(#1,\; #2)}{=}}%
}

\newcommand{\arron}[3]{%
  \ensuremath{\xrightarrow{\text{Э}_1(#1,\; #2,\; #3)}}%
}
\newcommand{\arrtw}[2]{%
  \ensuremath{\xrightarrow{\text{Э}_2(#1,\; #2)}}%
}
\newcommand{\arrth}[2]{%
  \ensuremath{\xrightarrow{\text{Э}_3(#1,\; #2)}}%
}

\makeatletter
\renewcommand*\env@matrix[1][*\c@MaxMatrixCols c]{%
  \hskip -\arraycolsep
  \let\@ifnextchar\new@ifnextchar
  \array{#1}}
\makeatother

% tasks here https://www.dropbox.com/s/8yhwwpf3dvft1yd/LA_19-20_Stream2_Test2.pdf?dl=0
\title{Летний экзамен учебного года 2019/2020\\Вариант \textnumero 1}
\author{	
  % ОТКОММЕНТИРУЙ СЕБЯ
    % Алиса Вернигор       \\ \href{https://t.me/allisyonok}{Telegram} \and
	Сергей Лоптев        \\ \href{https://t.me/beast_sl}{Telegram} \and
	Оля Козлова        \\ \href{https://t.me/grenlayk}{Telegram}
}

\date{}

\begin{document}
	\maketitle
    % Оля
    \section*{Задача \textnumero 1}
        % текст задачи 
        Найдите базис пересечения двух подпространств $L_1, L_2 \subset \RR^4$,
        где $L_1$ состоит из всех решений уравнения $2x_1 - x_2 - x_3 + 2x_4 = 0$, 
        а $L_2$ есть линейная оболочка векторов $(3, 2, 3, -2), (2, 2, 3, 1), 
        (3, 1, 4, -3)$.
        \begin{proof}[Решение.] \ 
            Будем действовать по стандартному алгоритму.
            \begin{enumerate}
                \item Представим $L_2$ в виде ОСЛУ: 
                \begin{multline*}
                    \begin{pmatrix}
                        3 & 2 & 3 & -2 \\
                        2 & 2 & 3 & 1 \\
                        3 & 1 & 4 & -3
                    \end{pmatrix} 
                    \overset{\elon{2}{3}{-1}}{\arron{2}{1}{-1}}
                    \begin{pmatrix}
                        1 & 0 & 0 & -3 \\
                        2 & 2 & 3 & 1 \\
                        1 & -1 & 1 & -4
                    \end{pmatrix} \overset{\elon{1}{3}{-1}}{\arron{1}{2}{-2}}
                    \begin{pmatrix}
                        1 & 0 & 0 & -3 \\
                        0 & 2 & 3 & 7 \\
                        0 & -1 & 1 & -1
                    \end{pmatrix}
                    \overset{\eltw{2}{3}}{\arrth{3}{-1}}
                    \begin{pmatrix}
                        1 & 0 & 0 & -3 \\
                        0 & 1 & -1 & 1 \\
                        0 & 2 & 3 & 7 \\
                    \end{pmatrix} \\
                    \arron{2}{3}{-2}
                    \begin{pmatrix}
                        1 & 0 & 0 & -3 \\
                        0 & 1 & -1 & 1 \\
                        0 & 0 & 5 & 5 \\
                    \end{pmatrix}
                    \arrth{3}{\nicefrac15}
                    \begin{pmatrix}
                        1 & 0 & 0 & -3 \\
                        0 & 1 & -1 & 1 \\
                        0 & 0 & 1 & 1 \\
                    \end{pmatrix}
                    \arron{3}{2}{1}
                    \begin{pmatrix}
                        1 & 0 & 0 & -3 \\
                        0 & 1 & 0 & 2 \\
                        0 & 0 & 1 & 1 \\
                    \end{pmatrix}
                \end{multline*}
                Получаем ФСР: $
                \begin{pmatrix}
                    3\\
                    -2\\
                    -1\\
                    1
                \end{pmatrix}
                \Rightarrow L_2 \colon \eq{3x_1 - 2x_2 - x_3 + x_4 = 0}$
                \item Тогда базис $L_1 \cap L_2$ это ФСР 
                $\eq{
                    3x_1 - 2x_2 - x_3 + x_4 = 0 \\
                    2x_1 - x_2 - x_3 + 2x_4 = 0
                }$
                Построим ее. 
                \begin{multline*}
                    \begin{pmatrix}
                        3 & -2 & -1 & 1 \\
                        2 & -1 & -1 & 2 \\
                    \end{pmatrix} 
                    \arron{2}{1}{-1}
                    \begin{pmatrix}
                        1 & -1 & 0 & -1 \\
                        2 & -1 & -1 & 2 \\
                    \end{pmatrix} 
                    \arron{1}{2}{-2}
                    \begin{pmatrix}
                        1 & -1 & 0 & -1 \\
                        0 & 1 & -1 & 4 \\
                    \end{pmatrix} 
                    \arron{2}{1}{1}
                    \begin{pmatrix}
                        1 & 0 & -1 & 3 \\
                        0 & 1 & -1 & 4 \\
                    \end{pmatrix} 
                \end{multline*}
                ФСР: $
                \begin{pmatrix}
                    1\\
                    1\\
                    1\\
                    0
                \end{pmatrix}, 
                \begin{pmatrix}
                    -3\\
                    -4\\
                    0\\
                    1
                \end{pmatrix}
                $
            \end{enumerate}
            \textbf{Ответ:} базис $L_1 \cap L_2$: $
            \begin{pmatrix}
                1\\
                1\\
                1\\
                0
            \end{pmatrix}, 
            \begin{pmatrix}
                -3\\
                -4\\
                0\\
                1
            \end{pmatrix}
            $

        \end{proof}
    % Сережа
    \section*{Задача \textnumero 2}
        % текст задачи
        Найдите невырожденную замену координат (выражение старых координат через 
        новые), приводящие квадратичную форму $Q(x, y, z) = 16x^2 + 9y^2 + z^4 - 24xy + 8xz$
        к нормальному виду, и выпишите этот вид.
        \begin{proof}[Решение.] \ 
            Тут подойдет Лагранж или Симметричный Гаусс --- пользуйтесь 
            тем, что лучше знаете. Однако Гауссом будет немного быстрее, на мой взгляд.
            \begin{enumerate}
                \item Для начала выпишем матрицу билинейной формы: $B(Q, e) = 
                \begin{pmatrix}
                    16 & -12 & 4 \\
                    -12 & 9 & 0 \\
                    4 & 0 & 1 
                \end{pmatrix}
                $. 

                \item А дальше я буду применять Симметричного Гаусса. Помним, что матрица справа 
                от черты ``накапливает'' в себе элементарные преобразования строк 
                и что все преобразования к матрице билинейно формы должны быть симметричными.
                \begin{multline*}
                    \smt{rrr|rrr} {
                        16 & -12 & 4 & 1 & 0  & 0\\
                        -12 & 9 & 0 & 0 & 1  & 0\\
                        4 & 0 & 1 & 0 & 0 & 1
                    } \arron{3}{1}{-4} \\
                    \xrightarrow{\text{строки}} 
                    \smt{rrr|rrr} {
                        0 & -12 & 0 & 1 & 0  & -4\\
                        -12 & 9 & 0 & 0 & 1  & 0\\
                        4 & 0 & 1 & 0 & 0 & 1
                    } \xrightarrow{\text{столбцы}} 
                    \smt{rrr|rrr} {
                        0 & -12 & 0 & 1 & 0  & -4\\
                        -12 & 9 & 0 & 0 & 1  & 0\\
                        0 & 0 & 1 & 0 & 0 & 1
                    } 
                    \arrtw{2}{\nicefrac13}  \\
                    \xrightarrow{\text{строки}} 
                    \smt{rrr|rrr} {
                        0 & -12 & 0 & 1 & 0  & -4\\
                        -4 & 3 & 0 & 0 & \nicefrac13  & 0\\
                        0 & 0 & 1 & 0 & 0 & 1
                    } \xrightarrow{\text{столбцы}} 
                    \smt{rrr|rrr} {
                        0 & -4 & 0 & 1 & 0  & -4\\
                        -4 & 1 & 0 & 0 & \nicefrac13  & 0\\
                        0 & 0 & 1 & 0 & 0 & 1
                    } \arron{2}{1}{4} \\
                    \xrightarrow{\text{строки}} 
                    \smt{rrr|rrr} {
                        -16 & 0 & 0 & 1 & \nicefrac43  & -4\\
                        -4 & 1 & 0 & 0 & \nicefrac13  & 0\\
                        0 & 0 & 1 & 0 & 0 & 1
                    } \xrightarrow{\text{столбцы}} 
                    \smt{rrr|rrr} {
                        -16 & 0 & 0 & 1 & \nicefrac43  & -4\\
                        0 & 1 & 0 & 0 & \nicefrac13  & 0\\
                        0 & 0 & 1 & 0 & 0 & 1
                    } \arrtw{1}{\nicefrac14}  \\
                    \xrightarrow{\text{строки}} 
                    \smt{rrr|rrr} {
                        -4 & 0 & 0 & \nicefrac14 & \nicefrac13  & -1\\
                        0 & 1 & 0 & 0 & \nicefrac13  & 0\\
                        0 & 0 & 1 & 0 & 0 & 1
                    } \xrightarrow{\text{столбцы}} 
                    \smt{rrr|rrr} {
                        -1 & 0 & 0 & \nicefrac14 & \nicefrac13  & -1\\
                        0 & 1 & 0 & 0 & \nicefrac13  & 0\\
                        0 & 0 & 1 & 0 & 0 & 1
                    }
                \end{multline*}
                \item Получили, что $Q(e^\prime) = -{x^\prime}^2 + {y^\prime}^2 + {z^\prime}^2$. 
                Для матрицы перехода C верно равенство: 
                $C^T = \begin{pmatrix}
                    \nicefrac14 & \nicefrac13  & -1\\
                    0 & \nicefrac13  & 0\\
                    0 & 0 & 1
                \end{pmatrix}    
                $

                Отсюда $C = \begin{pmatrix}
                    \nicefrac14 & 0  & 0\\
                    \nicefrac13 & \nicefrac13  & 0\\
                    -1 & 0 & 1
                \end{pmatrix}    
                $. Тогда $
                \begin{pmatrix}
                    x\\
                    y\\
                    z \\
                \end{pmatrix} = C \cdot 
                \begin{pmatrix}
                    x^\prime\\
                    y^\prime\\
                    z^\prime \\
                \end{pmatrix} = 
                \begin{pmatrix}
                    \nicefrac14 & 0  & 0\\
                    \nicefrac13 & \nicefrac13  & 0\\
                    -1 & 0 & 1
                \end{pmatrix} \cdot 
                \begin{pmatrix}
                    x^\prime\\
                    y^\prime\\
                    z^\prime \\
                \end{pmatrix} = 
                \begin{pmatrix}
                    \nicefrac14 \cdot x^\prime\\
                    \nicefrac13 \cdot x^\prime + \nicefrac13 \cdot y^\prime\\
                    -x^\prime + z^\prime \\
                \end{pmatrix}
                $ 
                
                Проверочка, что все с нужной стороны умножается: было --- $e^\prime = e \cdot C$. 
                Выводим для координат: \begin{multline*}v = e \cdot x = e^\prime x^\prime = e \cdot C \cdot x^\prime 
                \Rightarrow x = C \cdot x^\prime \\\end{multline*}
            \end{enumerate}

            \textbf{Ответ:} нормальный вид: $Q(e^\prime) = -{x^\prime}^2 + {y^\prime}^2 + {z^\prime}^2$

            Невырожденная замена: 
            $\eq{
                x = \nicefrac14 \cdot x^\prime \\
                y = \nicefrac13 \cdot x^\prime + \nicefrac13 \cdot y^\prime \\
                z = -x^\prime + z^\prime
            }
            $

        \end{proof}

    % Оля
    \section*{Задача \textnumero 3}
        % текст задачи
        Найдите псевдорешение для следующей системы линейных уравнений: 
        $\eq{
            2x_1 - x_2 = 1\\
            -x_1 + 3x_2 = -2 \\
            x_1 + x_2 = 1 \\
            -2x_1 + 3x_2 = 2
        }$
        \begin{proof}[Решение.] \ 
            

        \end{proof}
    
    % Алиса
    \section*{Задача \textnumero 4}
        % текст задачи
        Приведите пример двух диагонализуемых линейных операторов $\varphi$ и 
        $\psi$ в $\RR^2$, для которых оператор $4\varphi + 7\psi$ недиагонализуем.
        \begin{proof}[Решение.] \ 
            

        \end{proof}
    
    % Сережа
    \section*{Задача \textnumero 5}
        % текст задачи
        Ортогональный линейный оператор $\varphi \colon \RR^3 \to \RR^3$ имеет в стандартном
        базисе матрицу 
        $$
        \begin{pmatrix}
            \nicefrac23 & -\nicefrac23 & \nicefrac13 \\
            \nicefrac23 & \nicefrac13 & \nicefrac23 \\
            \nicefrac13 & \nicefrac23 & -\nicefrac23 \\
        \end{pmatrix}
        $$
        Найдите ортонормированный базис, в котором матрица оператора $\varphi$ имеет 
        канонический вид и выпишите эту матрицу. Укажите ось и угол поворота, определяемого 
        оператором $\varphi$.
        \begin{proof}[Решение.] \ 
            Напомню, что канонический вид ортогонального оператора в $\RR^3$ --- $
            \begin{pmatrix}
                \text{П}(\alpha) & 0 \\
                0 & \pm 1
            \end{pmatrix} 
            $, где $\text{П}(\alpha) = 
            \begin{pmatrix}
                \cos{\alpha} & -\sin{\alpha} \\
                \sin{\alpha} & \cos{\alpha}
            \end{pmatrix}
            $.

            \begin{enumerate}
                \item $A \neq A^T \Rightarrow$ к диагональному виду привести не сможем.
                \item Найдем $f_3$ --- собственный вектор для $\pm 1$. 
                \begin{itemize}
                    \item Проверим 1.
                    \begin{multline*}
                        \begin{pmatrix}
                            -\nicefrac13 & -\nicefrac23 & \nicefrac13 \\
                            \nicefrac23 & -\nicefrac23 & \nicefrac23 \\
                            \nicefrac13 & \nicefrac23 & -\nicefrac53 \\
                        \end{pmatrix} 
                        \rightarrow
                        \begin{pmatrix}
                            -1 & -2 & 1 \\
                            2 & -2 & 2 \\
                            1 & 2 & -5 \\
                        \end{pmatrix} 
                        \rightarrow
                        \begin{pmatrix}
                            1 & 2 & 2 \\
                            0 & -6 & 0 \\
                            0 & 0 & -6 \\
                        \end{pmatrix} \Rightarrow \text{ФСР = $\{\varnothing$\}}\\
                    \end{multline*}
                    \item Проверим -1.
                    \begin{multline*}
                        \begin{pmatrix}
                            \nicefrac53 & -\nicefrac23 & \nicefrac13 \\
                            \nicefrac23 & \nicefrac43 & \nicefrac23 \\
                            \nicefrac13 & \nicefrac23 & \nicefrac13 \\
                        \end{pmatrix} 
                        \rightarrow
                        \begin{pmatrix}
                            5 & -2 & 1 \\
                            2 & 4 & 2 \\
                            1 & 2 & 1 \\
                        \end{pmatrix} 
                        \rightarrow
                        \begin{pmatrix}
                            1 & 2 & 1 \\
                            6 & 0 & 0 \\
                        \end{pmatrix} \Rightarrow \text{ФСР = } 
                        \begin{pmatrix}
                            0\\
                            1\\
                            -2
                        \end{pmatrix}, f_3 = 
                        \begin{pmatrix}
                            0\\
                            \nicefrac1{\sqrt{5}}\\
                            -\nicefrac2{\sqrt{5}}
                        \end{pmatrix}\\
                    \end{multline*}
                    $f_3$ --- ось поворота.
                \end{itemize}
                \item Дополним $f_3$ до ортонормированного базиса по ОСЛУ из пункта 2. 
                $f_1 = \begin{pmatrix}
                    0 \\
                    \nicefrac2{\sqrt{5}} \\
                    \nicefrac1{\sqrt{5}}\\
                \end{pmatrix}$, $f_2 = \begin{pmatrix}
                    1 \\
                    0 \\
                    0\\
                \end{pmatrix}$. Заметим, что $(f_1, f_2) = 0$, значит базис уже ортогонален 
                ($(f_1, f_3) = 0$ и $(f_2, f_3) = 0$ по построению).
                \item Найдем $\sin \alpha$ и $\cos \alpha$. $\sin \alpha = (\varphi(f_1), f_2), 
                \cos \alpha = (\varphi(f_1), f_1)$.
                \begin{itemize}
                    \item $\varphi(f_1) = Af_1 = 
                    \begin{pmatrix}
                        \nicefrac53 & -\nicefrac23 & \nicefrac13 \\
                        \nicefrac23 & \nicefrac43 & \nicefrac23 \\
                        \nicefrac13 & \nicefrac23 & \nicefrac13 \\
                    \end{pmatrix} 
                    \cdot 
                    \begin{pmatrix}
                        0 \\
                        \nicefrac2{\sqrt{5}} \\
                        \nicefrac1{\sqrt{5}}\\
                    \end{pmatrix} = 
                    \nicefrac{1}{3\sqrt{5}}
                    \begin{pmatrix}
                        -5 \\
                        4 \\
                        2\\
                    \end{pmatrix}$
                    \item $\sin \alpha = (\varphi(f_1), f_2) = -\nicefrac{5}{3\sqrt{5}} = -\nicefrac{\sqrt{5}}{3}$
                    \item $\cos \alpha = (\varphi(f_1), f_1) = \nicefrac{1}{15}(8 + 2) = \nicefrac23$
                \end{itemize}

                \textbf{Ответ:} $(f_1, f_2, f_3) = 
                \begin{pmatrix}
                    0 \\
                    \nicefrac2{\sqrt{5}} \\
                    \nicefrac1{\sqrt{5}}\\
                \end{pmatrix}$, $\begin{pmatrix}
                    1 \\
                    0 \\
                    0\\
                \end{pmatrix}$, 
                $
                \begin{pmatrix}
                    0\\
                    \nicefrac1{\sqrt{5}}\\
                    -\nicefrac2{\sqrt{5}}
                \end{pmatrix}, 
                $
                $A(\varphi, f) = 
                \begin{pmatrix}
                    \nicefrac23 & \nicefrac{\sqrt{5}}{3} & 0 \\
                    -\nicefrac{\sqrt{5}}{3} & \nicefrac23 & 0 \\
                    0 & 0 & -1 \\
                \end{pmatrix} 
                $

                Ось поворота - $f_3 = 
                \begin{pmatrix}
                    0\\
                    \nicefrac1{\sqrt{5}}\\
                    -\nicefrac2{\sqrt{5}}
                \end{pmatrix} 
                $, угол поворота $\alpha = - \arccos{\nicefrac{2}{3}}$
            \end{enumerate}


        \end{proof}

    % Оля
    \section*{Задача \textnumero 6}
        % текст задачи
        Приведите пример матрицы $A \in \Mat_{2\times 3}(\RR)$ ранга 2, для которой 
        ближайшей по норме Фробениуса матрицей ранга 1 будет матрица
        $$
        \begin{pmatrix}
            -6 & 3 & 3 \\
            3 & -1 & -1 \\
        \end{pmatrix}.
        $$
        \begin{proof}[Решение.] \ 
            

        \end{proof}

    % Оля
    \section*{Задача \textnumero 7}
        % текст задачи
        Найдите все значения параметра $a$, при которых уравнение 
        $$ 
        3y^2 + 2z^2 -4axz + 8x - 6y - 9
        $$
        определяет эллиптический параболоид в $\RR^3$. Для каждого найденного значения  $a$
        укажите прямоугольную декартову систему координат в $\RR^3$ (выражение старых координат 
        через новые), в которой данное уравнение принимает канонический вид. 
        \begin{proof}[Решение.] \ 
            

        \end{proof}
    
    % Сережа
    \section*{Задача \textnumero 8}
        % текст задачи
        Линейный оператор $\varphi \colon \RR^4 \to \RR^4$ имеет в стандартном базисе матрицу
        $$
        \begin{pmatrix}
            3 & 1 & 0 & -2 \\
            0 & 1 & 0 & 4 \\
            6 & 4 & 3 & -5 \\
            0 & -1 & 0 & 5 \\
        \end{pmatrix}.
        $$
        Найдите базис пространства $\RR^4$, в котором матрица оператора $\varphi$ 
        имеет жорданову форму и укажите эту жорданову форму.
        \begin{proof}[Решение.] \ 
            

        \end{proof}
      
\end{document}
