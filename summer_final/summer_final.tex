\documentclass[a4paper]{article}
\usepackage{header}
\newcommand{\Mat}{\operatorname{Mat}}
\newcommand{\M}{\operatorname{M}}
\allowdisplaybreaks

%% Графика
\usepackage{graphicx}       
\graphicspath{{images/}}            
\usepackage{tikz}  
\usetikzlibrary{patterns}                 
\usepackage{pgfplots}              
\usepackage{circuitikz}
\usepackage{nicefrac}
\usepackage{bm}
            
\usepackage{titlesec}
\titlespacing*{\section}{0pt}{0pt}{0pt}

\usetikzlibrary{quotes,angles}
\usetikzlibrary{positioning,intersections}
\usetikzlibrary{through}

\textwidth=19.0cm \oddsidemargin=-1.3cm
\textheight=26cm \topmargin=-2.5cm

\newtheorem{task}{Задание}
\newtheorem*{task*}{Задание}

\theoremstyle{remark}

\newtheorem{remark}{Замечание}
\newtheorem*{remark*}{Замечание}
\newtheorem{commentarium}{Комментарий}
\newtheorem*{commentarium*}{Комментарий}

\usepackage{tikz-cd}

\newenvironment{sysmatrix}[1]
{
    \left(\begin{array}{@{}#1@{}}
}
{\end{array}\right)}
\newcommand{\smt}[2]{\begin{sysmatrix}{#1} #2\end{sysmatrix}}
\newcommand{\eq}[1]{\begin{cases} #1 \end{cases}}

\newcommand{\elon}[3]{%
  \ensuremath{\text{Э}_1(#1,\; #2,\; #3)}%
}
\newcommand{\eltw}[2]{%
  \ensuremath{\text{Э}_2(#1,\; #2)}%
}
\newcommand{\elth}[2]{%
  \ensuremath{\text{Э}_3(#1,\; #2)}%
}
 
\newcommand{\eqon}[3]{%
  \ensuremath{\overset{\text{Э}_1(#1,\; #2,\; #3)}{=}}%
}
\newcommand{\eqtw}[2]{%
  \ensuremath{\overset{\text{Э}_2(#1,\; #2)}{=}}%
}
\newcommand{\eqth}[2]{%
  \ensuremath{\overset{\text{Э}_3(#1,\; #2)}{=}}%
}

\newcommand{\arron}[3]{%
  \ensuremath{\xrightarrow{\text{Э}_1(#1,\; #2,\; #3)}}%
}
\newcommand{\arrtw}[2]{%
  \ensuremath{\xrightarrow{\text{Э}_2(#1,\; #2)}}%
}
\newcommand{\arrth}[2]{%
  \ensuremath{\xrightarrow{\text{Э}_3(#1,\; #2)}}%
}

\makeatletter
\renewcommand*\env@matrix[1][*\c@MaxMatrixCols c]{%
  \hskip -\arraycolsep
  \let\@ifnextchar\new@ifnextchar
  \array{#1}}
\makeatother

% tasks here https://www.dropbox.com/s/8yhwwpf3dvft1yd/LA_19-20_Stream2_Test2.pdf?dl=0
\title{Летний экзамен учебного года 2019/2020\\Вариант \textnumero 1}
\author{	
  % ОТКОММЕНТИРУЙ СЕБЯ
    % Алиса Вернигор       \\ \href{https://t.me/allisyonok}{Telegram} \and
	Сергей Лоптев        \\ \href{https://t.me/beast_sl}{Telegram} \and
	Оля Козлова        \\ \href{https://t.me/grenlayk}{Telegram}
}

\date{}

\begin{document}
	\maketitle
    % Оля
    \section*{Задача \textnumero 1}
        % текст задачи 
        Найдите базис пересечения двух подпространств $L_1, L_2 \subset \RR^4$,
        где $L_1$ состоит из всех решений уравнения $2x_1 - x_2 - x_3 + 2x_4 = 0$, 
        а $L_2$ есть линейная оболочка векторов $(3, 2, 3, -2), (2, 2, 3, 1), 
        (3, 1, 4, -3)$.
        \begin{proof}[Решение.] \ 
            Будем действовать по стандартному алгоритму.
            \begin{enumerate}
                \item Представим $L_2$ в виде ОСЛУ: 
                \begin{multline*}
                    \begin{pmatrix}
                        3 & 2 & 3 & -2 \\
                        2 & 2 & 3 & 1 \\
                        3 & 1 & 4 & -3
                    \end{pmatrix} 
                    \overset{\elon{2}{3}{-1}}{\arron{2}{1}{-1}}
                    \begin{pmatrix}
                        1 & 0 & 0 & -3 \\
                        2 & 2 & 3 & 1 \\
                        1 & -1 & 1 & -4
                    \end{pmatrix} \overset{\elon{1}{3}{-1}}{\arron{1}{2}{-2}}
                    \begin{pmatrix}
                        1 & 0 & 0 & -3 \\
                        0 & 2 & 3 & 7 \\
                        0 & -1 & 1 & -1
                    \end{pmatrix}
                    \overset{\eltw{2}{3}}{\arrth{3}{-1}}
                    \begin{pmatrix}
                        1 & 0 & 0 & -3 \\
                        0 & 1 & -1 & 1 \\
                        0 & 2 & 3 & 7 \\
                    \end{pmatrix} \\
                    \arron{2}{3}{-2}
                    \begin{pmatrix}
                        1 & 0 & 0 & -3 \\
                        0 & 1 & -1 & 1 \\
                        0 & 0 & 5 & 5 \\
                    \end{pmatrix}
                    \arrth{3}{\nicefrac15}
                    \begin{pmatrix}
                        1 & 0 & 0 & -3 \\
                        0 & 1 & -1 & 1 \\
                        0 & 0 & 1 & 1 \\
                    \end{pmatrix}
                    \arron{3}{2}{1}
                    \begin{pmatrix}
                        1 & 0 & 0 & -3 \\
                        0 & 1 & 0 & 2 \\
                        0 & 0 & 1 & 1 \\
                    \end{pmatrix}
                \end{multline*}
                Получаем ФСР: $
                \begin{pmatrix}
                    3\\
                    -2\\
                    -1\\
                    1
                \end{pmatrix}
                \Rightarrow L_2 \colon \eq{3x_1 - 2x_2 - x_3 + x_4 = 0}$
                \item Тогда базис $L_1 \cap L_2$ это ФСР 
                $\eq{
                    3x_1 - 2x_2 - x_3 + x_4 = 0 \\
                    2x_1 - x_2 - x_3 + 2x_4 = 0
                }$
                Построим ее. 
                \begin{multline*}
                    \begin{pmatrix}
                        3 & -2 & -1 & 1 \\
                        2 & -1 & -1 & 2 \\
                    \end{pmatrix} 
                    \arron{2}{1}{-1}
                    \begin{pmatrix}
                        1 & -1 & 0 & -1 \\
                        2 & -1 & -1 & 2 \\
                    \end{pmatrix} 
                    \arron{1}{2}{-2}
                    \begin{pmatrix}
                        1 & -1 & 0 & -1 \\
                        0 & 1 & -1 & 4 \\
                    \end{pmatrix} 
                    \arron{2}{1}{1}
                    \begin{pmatrix}
                        1 & 0 & -1 & 3 \\
                        0 & 1 & -1 & 4 \\
                    \end{pmatrix} 
                \end{multline*}
                ФСР: $
                \begin{pmatrix}
                    1\\
                    1\\
                    1\\
                    0
                \end{pmatrix}, 
                \begin{pmatrix}
                    -3\\
                    -4\\
                    0\\
                    1
                \end{pmatrix}
                $
            \end{enumerate}
            \textbf{Ответ:} базис $L_1 \cap L_2$: $
            \begin{pmatrix}
                1\\
                1\\
                1\\
                0
            \end{pmatrix}, 
            \begin{pmatrix}
                -3\\
                -4\\
                0\\
                1
            \end{pmatrix}
            $

        \end{proof}
    % Оля
    \section*{Задача \textnumero 2}
        % текст задачи
        Найдите невырожденную замену координат (выражение старых координат через 
        новые), приводящие квадратичную форму $Q(x, y, z) = 16x^2 + 9y^2 + z^4 - 24xy + 8xz$
        к нормальному виду, и выпишите этот вид.
        \begin{proof}[Решение.] \ 
            Тут подойдет Лагранж или Симметричный Гаусс --- пользуйтесь 
            тем, что лучше знаете. Однако Гауссом будет немного быстрее, на мой взгляд.
            \begin{enumerate}
                \item Для начала выпишем матрицу билинейной формы: $B(Q, e) = 
                \begin{pmatrix}
                    16 & -12 & 4 \\
                    -12 & 9 & 0 \\
                    4 & 0 & 1 
                \end{pmatrix}
                $. 

                \item А дальше я буду применять Симметричного Гаусса. Помним, что матрица справа 
                от черты ``накапливает'' в себе элементарные преобразования строк 
                и что все преобразования к матрице билинейно формы должны быть симметричными.
                \begin{multline*}
                    \smt{rrr|rrr} {
                        16 & -12 & 4 & 1 & 0  & 0\\
                        -12 & 9 & 0 & 0 & 1  & 0\\
                        4 & 0 & 1 & 0 & 0 & 1
                    } \arron{3}{1}{-4} \\
                    \xrightarrow{\text{строки}} 
                    \smt{rrr|rrr} {
                        0 & -12 & 0 & 1 & 0  & -4\\
                        -12 & 9 & 0 & 0 & 1  & 0\\
                        4 & 0 & 1 & 0 & 0 & 1
                    } \xrightarrow{\text{столбцы}} 
                    \smt{rrr|rrr} {
                        0 & -12 & 0 & 1 & 0  & -4\\
                        -12 & 9 & 0 & 0 & 1  & 0\\
                        0 & 0 & 1 & 0 & 0 & 1
                    } 
                    \arrtw{2}{\nicefrac13}  \\
                    \xrightarrow{\text{строки}} 
                    \smt{rrr|rrr} {
                        0 & -12 & 0 & 1 & 0  & -4\\
                        -4 & 3 & 0 & 0 & \nicefrac13  & 0\\
                        0 & 0 & 1 & 0 & 0 & 1
                    } \xrightarrow{\text{столбцы}} 
                    \smt{rrr|rrr} {
                        0 & -4 & 0 & 1 & 0  & -4\\
                        -4 & 1 & 0 & 0 & \nicefrac13  & 0\\
                        0 & 0 & 1 & 0 & 0 & 1
                    } \arron{2}{1}{4} \\
                    \xrightarrow{\text{строки}} 
                    \smt{rrr|rrr} {
                        -16 & 0 & 0 & 1 & \nicefrac43  & -4\\
                        -4 & 1 & 0 & 0 & \nicefrac13  & 0\\
                        0 & 0 & 1 & 0 & 0 & 1
                    } \xrightarrow{\text{столбцы}} 
                    \smt{rrr|rrr} {
                        -16 & 0 & 0 & 1 & \nicefrac43  & -4\\
                        0 & 1 & 0 & 0 & \nicefrac13  & 0\\
                        0 & 0 & 1 & 0 & 0 & 1
                    } \arrtw{1}{\nicefrac14}  \\
                    \xrightarrow{\text{строки}} 
                    \smt{rrr|rrr} {
                        -4 & 0 & 0 & \nicefrac14 & \nicefrac13  & -1\\
                        0 & 1 & 0 & 0 & \nicefrac13  & 0\\
                        0 & 0 & 1 & 0 & 0 & 1
                    } \xrightarrow{\text{столбцы}} 
                    \smt{rrr|rrr} {
                        -1 & 0 & 0 & \nicefrac14 & \nicefrac13  & -1\\
                        0 & 1 & 0 & 0 & \nicefrac13  & 0\\
                        0 & 0 & 1 & 0 & 0 & 1
                    }
                \end{multline*}
                \item Получили, что $Q(e^\prime) = -{x^\prime}^2 + {y^\prime}^2 + {z^\prime}^2$. 
                Для матрицы перехода C верно равенство: 
                $C^T = \begin{pmatrix}
                    \nicefrac14 & \nicefrac13  & -1\\
                    0 & \nicefrac13  & 0\\
                    0 & 0 & 1
                \end{pmatrix}    
                $

                Отсюда $C = \begin{pmatrix}
                    \nicefrac14 & 0  & 0\\
                    \nicefrac13 & \nicefrac13  & 0\\
                    -1 & 0 & 1
                \end{pmatrix}    
                $. Тогда $
                \begin{pmatrix}
                    x\\
                    y\\
                    z \\
                \end{pmatrix} = C \cdot 
                \begin{pmatrix}
                    x^\prime\\
                    y^\prime\\
                    z^\prime \\
                \end{pmatrix} = 
                \begin{pmatrix}
                    \nicefrac14 & 0  & 0\\
                    \nicefrac13 & \nicefrac13  & 0\\
                    -1 & 0 & 1
                \end{pmatrix} \cdot 
                \begin{pmatrix}
                    x^\prime\\
                    y^\prime\\
                    z^\prime \\
                \end{pmatrix} = 
                \begin{pmatrix}
                    \nicefrac14 \cdot x^\prime\\
                    \nicefrac13 \cdot x^\prime + \nicefrac13 \cdot y^\prime\\
                    -x^\prime + z^\prime \\
                \end{pmatrix}
                $ 
                
                Проверочка, что все с нужной стороны умножается: было --- $e^\prime = e \cdot C$. 
                Выводим для координат: \begin{multline*}v = e \cdot x = e^\prime x^\prime = e \cdot C \cdot x^\prime 
                \Rightarrow x = C \cdot x^\prime \\\end{multline*}
            \end{enumerate}

            \textbf{Ответ:} нормальный вид: $Q(e^\prime) = -{x^\prime}^2 + {y^\prime}^2 + {z^\prime}^2$

            Невырожденная замена: 
            $\eq{
                x = \nicefrac14 \cdot x^\prime \\
                y = \nicefrac13 \cdot x^\prime + \nicefrac13 \cdot y^\prime \\
                z = -x^\prime + z^\prime
            }
            $

        \end{proof}

    % Серёжа
    \section*{Задача \textnumero 3}
        % текст задачи
        Найдите псевдорешение для следующей системы линейных уравнений: 
        $\eq{
            2x_1 - x_2 = 1\\
            -x_1 + 3x_2 = -2 \\
            x_1 + x_2 = 1 \\
            -2x_1 + 3x_2 = 2
        }$
        \begin{proof}[Решение.] \ 
        Систему из условия можно переписать как $Ax = b$, где:
        \begin{align*}
          A = \begin{pmatrix}
            2 & -1 \\
            -1 & 3 \\
            1 & 1 \\
            -2 & 3 
          \end{pmatrix}, \quad\quad 
          b = \begin{pmatrix}
            1 \\
            -2 \\
            1 \\
            2
          \end{pmatrix}
        \end{align*}
        Помним, что такая система может быть решена по формуле Атата:
        \begin{align*}
          x &= (A^TA)^{-1}A^Tb.
        \end{align*}
        Причём, в силу линейной независимости столбцов $A$, формула является корректной.

        Посчитаем по частям.
        \begin{align*}
          A^TA = \begin{pmatrix}
            2 & -1 & 1 & -2 \\
            -1 & 3 & 1 & 3
          \end{pmatrix} \cdot \begin{pmatrix}
            2 & -1 \\
            -1 & 3 \\
            1 & 1 \\
            -2 & 3
          \end{pmatrix} = \begin{pmatrix}
            10 & -10 \\
            -10 & 20
          \end{pmatrix}.
        \end{align*}
        Ищем обратную матрицу через формулу с алгебраическими дополнениями (обратите внимание, что матрица после $\frac{1}{\det A}$~--- это \textbf{транспонированная} матрица алгебраических дополнений):
        \begin{align*}
          (A^TA)^{-1} = 
          \frac{1}{\det A} \begin{pmatrix}
            A_{11} & A_{21} \\
            A_{12} & A_{22}  
          \end{pmatrix} = 
          \frac{1}{100} \begin{pmatrix}
            20 & 10 \\
            10 & 10
          \end{pmatrix}.
        \end{align*}
        Теперь $A^T b$:
        \begin{align*}
          A^T b = \begin{pmatrix}
            2 & -1 & 1 & -2 \\
            -1 & 3 & 1 & 3
          \end{pmatrix} \cdot \begin{pmatrix}
            1 \\
            -2 \\
            1 \\
            2
          \end{pmatrix} =
          \begin{pmatrix}
            1 \\
            0
          \end{pmatrix}
        \end{align*}
        Завершим решение:
        \begin{align*}
          \hat{x} = (A^TA)^{-1}A^Tb = 
          \frac{1}{100} \begin{pmatrix}
            20 & 10 \\
            10 & 10
          \end{pmatrix} \cdot \begin{pmatrix}
            1 \\
            0
          \end{pmatrix} = 
          \frac{\bm{1}}{\bm{10}}\begin{pmatrix}
            \bm{2} \\
            \bm{1}
          \end{pmatrix}
        \end{align*}
        \end{proof}
    
    % Серёжа
    \section*{Задача \textnumero 4}
        % текст задачи
        Приведите пример двух диагонализуемых линейных операторов $\varphi$ и 
        $\psi$ в $\RR^2$, для которых оператор $4\varphi + 7\psi$ недиагонализуем.
        \begin{proof}[Решение.] \ 
        Ну, тут есть два подхода. Рассмотрим оба.
        \begin{enumerate}[label=\arabic*)]
          \item Мы работаем в $\RR^2$. Чего мы хотим от недиагонализуемого оператора? Чтобы его характеристический многочлен (а он будет второй степени) не раскладывался на линейные множители. Например, подойдёт хар.многочлен $t^2 + c$, где $c > 0$. Таким образом, хотим получить следующую матрицу оператора $4 \varphi + 7 \psi$: 
          \begin{equation*}
            A(4\varphi + 7\psi, e) = \begin{pmatrix}
              0 & c \\
              -c & 0
            \end{pmatrix},
          \end{equation*}
          где $c \neq 0$.
          Раз наш оператор равен $4 \varphi + 7 \psi$, логично будет взять следующие матрицы операторов $\varphi$ и $\psi$:
          \begin{align*}
            A(\varphi, e) = \begin{pmatrix}
              7 & 1 \\
              -1 & -7
            \end{pmatrix}, \quad\quad A(\psi, e) = \begin{pmatrix}
              -4 & 1 \\
              -1 & 4
            \end{pmatrix}.
          \end{align*}
          Тогда матрица оператора $4\varphi + 7\psi$ будет иметь нужный нам вид, где $c = 11$.
          Проверим диагонализуемость операторов $\varphi$ и $\psi$. Найдём их хармногочлены:
          \begin{align*}
            &\begin{aligned}
              \chi_\varphi(t) = (7 - t)(-7 - t) + 1 = (t - 7)(t + 7) + 1 = t^2 - 49 + 1 = t^2 - 48.
            \end{aligned} \\
            &\begin{aligned}
              \chi_\psi(t) = (-4 - t)(4 - t) + 1 = (t - 4)(t + 4) + 1 = t^2 - 16 + 1 = t^2 - 15. 
            \end{aligned}
          \end{align*}
          Очевидно, оба хармногочлена раскладываются на линейные множители, причём алгебраическая кратность корней будет равна единице. Это значит, что линейные операторы $\varphi$ и $\psi$, заданные таким образом, диагонализуемы.
          \item Воспользуемся вот каким фактом. Матрица вида $\begin{pmatrix}
            a & b \\
            0 & c
          \end{pmatrix}$ диагонализуема либо когда $b = 0$ (в этом случае матрица изначально диагональна), либо когда $a \neq c$ (тогда у хармногочлена два разных корня). Иначе (то есть, если $a = c$ и $b \neq 0$, тогда у хармногочлена будет один корень, это собственное значение с алгебраической кратностью 2, но геометрической кратностью 1) оператор недиагонализуем. Это значит, что матрицу оператора $\varphi$ мы можем взять вида $\begin{pmatrix}
            a & 0 \\
            0 & a
          \end{pmatrix}$, где $a \neq 0$, а матрицу оператора $\psi$~-- вида $\begin{pmatrix}
            d & b \\
            0 & c
          \end{pmatrix}$, где $b \neq 0$ и $d \neq c$.
          Например, возьмём 
          \begin{align*}
            A(\varphi, e) = \begin{pmatrix}
              7 & 0 \\
              0 & 0
            \end{pmatrix}, \quad\quad A(\psi, e) = \begin{pmatrix}
              0 & 2 \\
              0 & 4
            \end{pmatrix}.
          \end{align*}
          Тогда:
          \begin{equation*}
            A(4\varphi + 7\psi, e) = \begin{pmatrix}
              28 & 14 \\
              0 & 28
            \end{pmatrix}
          \end{equation*}
          и $4\varphi + 7\psi$~--- недиагонализуемый оператор.
        \end{enumerate}

        \end{proof}
    
    % Оля
    \section*{Задача \textnumero 5}
        % текст задачи
        Ортогональный линейный оператор $\varphi \colon \RR^3 \to \RR^3$ имеет в стандартном
        базисе матрицу 
        $$
        \begin{pmatrix}
            \nicefrac23 & -\nicefrac23 & \nicefrac13 \\
            \nicefrac23 & \nicefrac13 & \nicefrac23 \\
            \nicefrac13 & \nicefrac23 & -\nicefrac23 \\
        \end{pmatrix}
        $$
        Найдите ортонормированный базис, в котором матрица оператора $\varphi$ имеет 
        канонический вид и выпишите эту матрицу. Укажите ось и угол поворота, определяемого 
        оператором $\varphi$.
        \begin{proof}[Решение.] \ 
            Напомню, что канонический вид ортогонального оператора в $\RR^3$ --- $
            \begin{pmatrix}
                \text{П}(\alpha) & 0 \\
                0 & \pm 1
            \end{pmatrix} 
            $, где $\text{П}(\alpha) = 
            \begin{pmatrix}
                \cos{\alpha} & -\sin{\alpha} \\
                \sin{\alpha} & \cos{\alpha}
            \end{pmatrix}
            $.

            \begin{enumerate}
                \item $A \neq A^T \Rightarrow$ к диагональному виду привести не сможем.
                \item Найдем $f_3$ --- собственный вектор для $\pm 1$. 
                \begin{itemize}
                    \item Проверим 1.
                    \begin{multline*}
                        \begin{pmatrix}
                            -\nicefrac13 & -\nicefrac23 & \nicefrac13 \\
                            \nicefrac23 & -\nicefrac23 & \nicefrac23 \\
                            \nicefrac13 & \nicefrac23 & -\nicefrac53 \\
                        \end{pmatrix} 
                        \rightarrow
                        \begin{pmatrix}
                            -1 & -2 & 1 \\
                            2 & -2 & 2 \\
                            1 & 2 & -5 \\
                        \end{pmatrix} 
                        \rightarrow
                        \begin{pmatrix}
                            1 & 2 & 2 \\
                            0 & -6 & 0 \\
                            0 & 0 & -6 \\
                        \end{pmatrix} \Rightarrow \text{ФСР = $\{\varnothing$\}}\\
                    \end{multline*}
                    \item Проверим -1.
                    \begin{multline*}
                        \begin{pmatrix}
                            \nicefrac53 & -\nicefrac23 & \nicefrac13 \\
                            \nicefrac23 & \nicefrac43 & \nicefrac23 \\
                            \nicefrac13 & \nicefrac23 & \nicefrac13 \\
                        \end{pmatrix} 
                        \rightarrow
                        \begin{pmatrix}
                            5 & -2 & 1 \\
                            2 & 4 & 2 \\
                            1 & 2 & 1 \\
                        \end{pmatrix} 
                        \rightarrow
                        \begin{pmatrix}
                            1 & 2 & 1 \\
                            6 & 0 & 0 \\
                        \end{pmatrix} \Rightarrow \text{ФСР = } 
                        \begin{pmatrix}
                            0\\
                            1\\
                            -2
                        \end{pmatrix}, f_3 = 
                        \begin{pmatrix}
                            0\\
                            \nicefrac1{\sqrt{5}}\\
                            -\nicefrac2{\sqrt{5}}
                        \end{pmatrix}\\
                    \end{multline*}
                    $f_3$ --- ось поворота.
                \end{itemize}
                \item Дополним $f_3$ до ортонормированного базиса по ОСЛУ из пункта 2. 
                $f_1 = \begin{pmatrix}
                    0 \\
                    \nicefrac2{\sqrt{5}} \\
                    \nicefrac1{\sqrt{5}}\\
                \end{pmatrix}$, $f_2 = \begin{pmatrix}
                    1 \\
                    0 \\
                    0\\
                \end{pmatrix}$. Заметим, что $(f_1, f_2) = 0$, значит базис уже ортогонален 
                ($(f_1, f_3) = 0$ и $(f_2, f_3) = 0$ по построению).
                \item Найдем $\sin \alpha$ и $\cos \alpha$. $\sin \alpha = (\varphi(f_1), f_2), 
                \cos \alpha = (\varphi(f_1), f_1)$.
                \begin{itemize}
                    \item $\varphi(f_1) = Af_1 = 
                    \begin{pmatrix}
                        \nicefrac53 & -\nicefrac23 & \nicefrac13 \\
                        \nicefrac23 & \nicefrac43 & \nicefrac23 \\
                        \nicefrac13 & \nicefrac23 & \nicefrac13 \\
                    \end{pmatrix} 
                    \cdot 
                    \begin{pmatrix}
                        0 \\
                        \nicefrac2{\sqrt{5}} \\
                        \nicefrac1{\sqrt{5}}\\
                    \end{pmatrix} = 
                    \nicefrac{1}{3\sqrt{5}}
                    \begin{pmatrix}
                        -5 \\
                        4 \\
                        2\\
                    \end{pmatrix}$
                    \item $\sin \alpha = (\varphi(f_1), f_2) = -\nicefrac{5}{3\sqrt{5}} = -\nicefrac{\sqrt{5}}{3}$
                    \item $\cos \alpha = (\varphi(f_1), f_1) = \nicefrac{1}{15}(8 + 2) = \nicefrac23$
                \end{itemize}

                \textbf{Ответ:} $(f_1, f_2, f_3) = 
                \begin{pmatrix}
                    0 \\
                    \nicefrac2{\sqrt{5}} \\
                    \nicefrac1{\sqrt{5}}\\
                \end{pmatrix}$, $\begin{pmatrix}
                    1 \\
                    0 \\
                    0\\
                \end{pmatrix}$, 
                $
                \begin{pmatrix}
                    0\\
                    \nicefrac1{\sqrt{5}}\\
                    -\nicefrac2{\sqrt{5}}
                \end{pmatrix}, 
                $
                $A(\varphi, f) = 
                \begin{pmatrix}
                    \nicefrac23 & \nicefrac{\sqrt{5}}{3} & 0 \\
                    -\nicefrac{\sqrt{5}}{3} & \nicefrac23 & 0 \\
                    0 & 0 & -1 \\
                \end{pmatrix} 
                $

                Ось поворота - $f_3 = 
                \begin{pmatrix}
                    0\\
                    \nicefrac1{\sqrt{5}}\\
                    -\nicefrac2{\sqrt{5}}
                \end{pmatrix} 
                $, угол поворота $\alpha = - \arccos{\nicefrac{2}{3}}$
            \end{enumerate}


        \end{proof}

    % Серёжа
    \section*{Задача \textnumero 6}
        % текст задачи
        Приведите пример матрицы $A \in \Mat_{2\times 3}(\RR)$ ранга 2, для которой 
        ближайшей по норме Фробениуса матрицей ранга 1 будет матрица
        $$
        \begin{pmatrix}
            -6 & 3 & 3 \\
            2 & -1 & -1 \\
        \end{pmatrix}.
        $$
        \begin{proof}[Решение.] \ 
          Для начала найдём полное SVD матрицы по стандартному \href{https://docviewer.yandex.ru/view/1203695933/?*=4ryhHsfuMOpXKV9Y7YnIVhDS7hd7InVybCI6InlhLWRpc2stcHVibGljOi8vVUo4QnpCMHExNnpuNlljdUJYRTViUFRmcFVUTm5QeitYYnhiWCtteWoxU0tjVzhlcUdaMG9MMXk2QXFEVWRZR3EvSjZicG1SeU9Kb25UM1ZvWG5EYWc9PTovU2VtaW5hcjMxX3NlbWluYXIucGRmIiwidGl0bGUiOiJTZW1pbmFyMzFfc2VtaW5hci5wZGYiLCJub2lmcmFtZSI6ZmFsc2UsInVpZCI6IjEyMDM2OTU5MzMiLCJ0cyI6MTYyNDI3NTI2NDM1MywieXUiOiIyNzM0OTIzMDkxNTkzMDEzMDIzIn0%3D}{алгоритму}. 
          \begin{enumerate}
            \item Составим матрицу $S = AA^T \in \Mat_2(\RR)$:
            \begin{equation}
              S = AA^T = 
              \begin{pmatrix}
                -6 & 3 & 3 \\
                2 & -1 & -1 \\
              \end{pmatrix} \cdot \begin{pmatrix}
                -6 & 2 \\
                3 & -1 \\
                3 & -1
              \end{pmatrix} =
              \begin{pmatrix}
                54 & -18 \\
                -18 & 6
              \end{pmatrix}.
            \end{equation}
            Тогда $S = U \Sigma \Sigma^T U^T$.
            \item Найдём собственное разложение для матрицы $S:$ $S = CDC^T$. Для этого найдём корни хармногочлена:
            \begin{align*}
              \chi_S = (54 - t)(6 - t) - 18^2 = (t - 54)(t - 6) - 324 = t^2 - 60t + 324 - 324 = t^2 - 60t = t(t - 60) 
            \end{align*}
            Хармногочлен разложился на неприводимые множители; его корни, они же собственные значения матрицы~--- $\lambda_1 = 60$ и $\lambda_2 = 0$ (обязательно пронумеровать по убыванию). Тогда
            \begin{equation*}
              D = \begin{pmatrix}
                60 & 0 \\
                0 & 0
              \end{pmatrix}.
            \end{equation*}
            Теперь найдём собственные векторы:
            \begin{itemize}
              \item для $\lambda_1 = 60$, найдём ФСР для матрицы $S - \lambda_1 E$:
              \begin{align*}
                \begin{pmatrix}
                  -6 & -18 \\
                  -18 & -54
                \end{pmatrix} \arron{2}{1}{-3} \begin{pmatrix}
                  -6 & -18 \\
                  0 & 0
                \end{pmatrix} \arrth{1}{-6} \begin{pmatrix}
                  1 & 3 \\
                  0 & 0
                \end{pmatrix}
              \end{align*}
              Тогда ФСР~--- один вектор $\begin{pmatrix}
                -3 & 1
              \end{pmatrix}$. Нормируем и получим вектор $\frac{1}{\sqrt{10}}\begin{pmatrix}
                -3 & 1
              \end{pmatrix}$.
              \item для $\lambda_2 = 60$, найдём ФСР для матрицы $S - \lambda_2 E$:
              \begin{align*}
                \begin{pmatrix}
                  54 & -18 \\
                  -18 & 6
                \end{pmatrix} \arron{1}{2}{3} \begin{pmatrix}
                  0 & 0 \\
                  -18 & 6
                \end{pmatrix} \arrtw{1}{2} \begin{pmatrix}
                  -18 & 6 \\
                  0 & 0
                \end{pmatrix} \arrth{1}{-\nicefrac{1}{3}} \begin{pmatrix}
                  3 & -1 \\
                  0 & 0
                \end{pmatrix}
              \end{align*}
              Тогда ФСР~--- один вектор $\begin{pmatrix}
                1 & 3
              \end{pmatrix}$. Нормируем и получим вектор $\frac{1}{\sqrt{10}}\begin{pmatrix}
                1 & 3
              \end{pmatrix}$.
            \end{itemize}
            Тогда матрица $C$ выглядит так:
            \begin{equation*}
              C = \frac{1}{\sqrt{10}} \begin{pmatrix}
                -3 & 1 \\
                1 & 3
              \end{pmatrix}.
            \end{equation*}
            \item Тогда $U = C$, а $\Sigma\Sigma^{-1} = D$. Значит, $\sigma_1 = \sqrt{\lambda_1} = \sqrt{60}$ и $\sigma_2 = \sqrt{\lambda_2} = 0$.
            \item Знаем из алгоритма, что $v_i = \frac{1}{\sigma_i}A^t u_i$, где $0 \leq i \leq s,\ s: \sigma_s > 0 \wedge \sigma_{s+1} = 0$. У нас $s = 1$, тогда посчитаем $v_1$:
            \begin{align*}
              v_1 = \frac{1}{\sigma_1} A^t u_1 = \frac{1}{\sqrt{60}} \begin{pmatrix}
                -6 & 2 \\
                3 & -1 \\
                3 & -1      
              \end{pmatrix} \cdot \frac{1}{\sqrt{10}} \begin{pmatrix}
                -3 \\
                1
              \end{pmatrix} = \frac{1}{10\sqrt{6}} \begin{pmatrix}
                20 \\
                -10 \\
                -10
              \end{pmatrix} = \frac{1}{\sqrt{6}} \begin{pmatrix}
                2 \\
                -1 \\
                -1
              \end{pmatrix}
            \end{align*}
            \item Осталось найти $v_2$ и $v_3$. Для этого найдём ФСР системы $Ax = 0$ и ортогонализуем его Грамом-Шмидтом:
            \begin{align*}
              \begin{pmatrix}
                -6 & 3 & 3 \\
                2 & -1 & -1
              \end{pmatrix} \arron{1}{2}{3}
              \begin{pmatrix}
                0 & 0 & 0 \\
                2 & -1 & -1
              \end{pmatrix} \arrtw{1}{2} 
              \begin{pmatrix}
                2 & -1 & -1 \\
                0 & 0 & 0 
              \end{pmatrix}
            \end{align*}
            Тогда ФСР~--- векторы $f_1 = \begin{pmatrix}
              1 & 2 & 0
            \end{pmatrix}$, $f_2 = \begin{pmatrix}
              1 & 0 & 2
            \end{pmatrix}$. Заметим, что эти векторы ортогональны изначальному $v_1$. Теперь ортогонализуем:
            \begin{align*}
              f'_1 &= f_1 = \begin{pmatrix}
                1 & 2 & 0
              \end{pmatrix} \\
              f'_2 &= f_2 - \frac{(f_2, f'_1)}{(f'_1, f'_1)}f'_1 = \begin{pmatrix}
                1 & 0 & 2
              \end{pmatrix} - \frac{1}{5}\begin{pmatrix}
                1 & 2 & 0
              \end{pmatrix} = \begin{pmatrix}
                \nicefrac{4}{5} & -\nicefrac{2}{5} & 2 
              \end{pmatrix} \to \begin{pmatrix}
                2 & -1 & 5
              \end{pmatrix}
            \end{align*}
            Нормируем:
            \begin{align*}
              v_2 &= \frac{f'_1}{|| f'_1 ||} = \frac{1}{\sqrt{5}}\begin{pmatrix}
                1 & 2 & 0
              \end{pmatrix} \\
              v_3 &= \frac{f'_2}{||f'_2||} = \frac{1}{\sqrt{30}}\begin{pmatrix}
                2 & -1 & 5
              \end{pmatrix}
            \end{align*}
          \end{enumerate}
          Ура, SVD найдено! На всякий случай запишем, что получилось:
          \begin{equation*}
            A = \frac{1}{\sqrt{10}} \begin{pmatrix}
              -3 & 1 \\
              1 & 3
            \end{pmatrix} \begin{pmatrix}
              \sqrt{60} & 0 & 0\\
              0 & 0 & 0
            \end{pmatrix} \begin{pmatrix}
              2/\sqrt{6} & 1/\sqrt{5} & 2/\sqrt{30} \\
              -1/\sqrt{6} & 2/\sqrt{5} &-1/\sqrt{30} \\
              -1/\sqrt{6} & 0 & 5/\sqrt{30}
            \end{pmatrix}
          \end{equation*}
          Теперь, чтобы получить какую-то матрицу, к которой $A$~--- ближайшая по норме Фробениуса ранга 1, надо поправить $\Sigma$, чтобы $0 < \sigma_2 < \sigma_1$. Например, подойдёт число $\sqrt{50}$. Замечательное число. Давайте тогда посчитаем  $A' = \sigma_1u_1^Tv_1 + \sigma_2 u_2^T v_2$:
          \begin{align*}
            \bm{A'} &\bm{=} \sqrt{60} \frac{1}{\sqrt{10}} \frac{1}{\sqrt{6}} \begin{pmatrix}
              -3 \\ 1
            \end{pmatrix} \begin{pmatrix}
              2 &
              -1 &
              -1
            \end{pmatrix} + \sqrt{50} \frac{1}{\sqrt{10}} \frac{1}{\sqrt{5}} \begin{pmatrix}
              1 \\
              3
            \end{pmatrix} \begin{pmatrix}
              1 & 2 & 0
            \end{pmatrix} = \begin{pmatrix}
              -6 & 3 & 3 \\
              2 & -1 & -1
            \end{pmatrix} + \begin{pmatrix}
              1 & 2 & 0 \\
              3 & 6 & 0
            \end{pmatrix} = \\ &= \begin{pmatrix}
              \bm{-5} & \bm{5} & \bm{3} \\
              \bm{5} & \bm{-5} & \bm{-1}
            \end{pmatrix}
          \end{align*} 
          Если что, таким способом на КР задачу решать не надо. Давайте оптимизируем. Во-первых, зачем нам считать полное SVD? Мы совсем нигде не воспользовались вектором $v_3$. Давайте лучше найдём компактное SVD, после чего дополним $u_1$ до ортонормированного базиса $\RR^2$, а для $v_1$ достаточно просто найти какой-то ортогональный вектор единичной длины из $\RR^3$, затем изменим $\sigma_2$ и тоже получим верный ответ.
          
          Алгоритм (из той же ссылки) очень похож, но во втором пункте мы забиваем на собственное значение $0$ и на собственный вектор, который ему отвечает. Получаем, что:
          \begin{align*}
            U = \frac{1}{\sqrt{10}}\begin{pmatrix}
              -3 \\
              1 
            \end{pmatrix}, \quad\quad 
            \Sigma = \begin{pmatrix}
              \sqrt{60}
            \end{pmatrix}.
          \end{align*}
          После этого мы по алгоритму получаем, что $v_1 = \frac{1}{\sigma_1} A^T u_1$, или (считали выше):
          \begin{align*}
            v_1 = \frac{1}{\sqrt{6}} \begin{pmatrix}
              2 \\
              -1 \\
              -1
            \end{pmatrix} = V.
          \end{align*}
          Осталось только дополнить векторы. Например, вектор $u_1$ дополняется вектором $u_2 = \frac{1}{\sqrt{10}} \begin{pmatrix}
            1 & 3
          \end{pmatrix}$, а вектор $v_1$~--- вектором $\frac{1}{\sqrt{3}} \begin{pmatrix}
            1 & 1 & 1
          \end{pmatrix}$. Тогда выберем $\sigma_2 = \sqrt{30}$ и получим:
          \begin{align*}
            \bm{A'} &\bm{=} \sigma_1u_1^Tv_1 + \sigma_2 u_2^T v_2 = \sqrt{60} \frac{1}{\sqrt{10}} \frac{1}{\sqrt{6}} \begin{pmatrix}
              -3 \\
              1
            \end{pmatrix} \begin{pmatrix}
              2 & -1 & -1
            \end{pmatrix} + \sqrt{30} \frac{1}{\sqrt{10}} \frac{1}{\sqrt{3}} \begin{pmatrix}
              1 \\
              3
            \end{pmatrix} \begin{pmatrix}
              1 & 1 & 1
            \end{pmatrix} = \\
            &= \begin{pmatrix}
              -6 & 3 & 3 \\
              2 & -1 & -1
            \end{pmatrix} + \begin{pmatrix}
              1 & 1 & 1 \\
              3 & 3  & 3
            \end{pmatrix} = \begin{pmatrix}
              \bm{-5} & \bm{4} & \bm{4} \\
              \bm{5} & \bm{2} & \bm{2}
            \end{pmatrix}
          \end{align*}
          Уже лучше. Но можно ещё лучше. Крутость нашей ситуации в том, что так как матрица одноранговая, найти векторы $u_1$ и $v_1$ для неё ищутся очень тривиально. Посмотрим на матрицу $A$:
          \begin{equation*}
            A = \begin{pmatrix}
              -6 & 3 & 3 \\
              2 & -1 & -1
            \end{pmatrix}
          \end{equation*}
          Очевидно, что первая строка~--- это вторая, умноженная на $-3$. Тогда ясно, что 
          \begin{equation*}
            A = \begin{pmatrix}
              -3 \\
              1
            \end{pmatrix} \begin{pmatrix}
              2 & -1 & -1
            \end{pmatrix}.
          \end{equation*}
          Тогда мы можем просто нормировать первый множитель и сказать, что это $u_1$, затем нормировать второй множитель и сказать, что это $v_1$, а затем взять произведение длин множителей и сказать что это $\sigma_1$. После этого алгоритм такой же, как в предыдущем пункте (про компактное SVD). Решили задачу, по пути разобрали два алгоритма и соптимизировали их. Отлично!
        \end{proof}

    % Серёжа
    \section*{Задача \textnumero 7}
        % текст задачи
        Найдите все значения параметра $a$, при которых уравнение 
        $$ 
        3y^2 + 2z^2 -4axz + 8x - 6y - 9
        $$
        определяет эллиптический параболоид в $\RR^3$. Для каждого найденного значения  $a$
        укажите прямоугольную декартову систему координат в $\RR^3$ (выражение старых координат 
        через новые), в которой данное уравнение принимает канонический вид. 
        \begin{proof}[Решение.] \ 
          Мы знаем, что уравнение эллиптического параболоида выглядит как:
          \begin{equation*}
            \frac{x^2}{a^2} + \frac{y^2}{b^2} = 2z.
          \end{equation*}
          Сначала разберёмся со слагаемыми второго порядка. Приведём часть $3y^2 + 2z^2 - 4axz$ к главным осям.
          Рассмотрим матрицу квадратичной формы:
        \begin{align*}
          A = \begin{pmatrix}
            0 & 0 & -2a \\
            0 & 3 & 0 \\
            -2a & 0 & 2
          \end{pmatrix}
        \end{align*}
        Она симметрична, следовательно, диагонализуема. Диагонализуем. Найдём характеристический многочлен:
        \begin{align*}
          \chi_A(t) = -\begin{vmatrix}
            -t & 0 & -2a \\
            0 & 3 - t & 0 \\
            -2a & 0 & 2 - t
          \end{vmatrix} = -(t - 3) \begin{vmatrix}
            -t & -2a \\
            -2a & 2 - t
          \end{vmatrix} = -(t - 3)((-t)(2 - t) - 4a^2) = - (t - 3)(t^2 - 2t - 4a^2).
        \end{align*}
        Найдём дискриминант правого множителя:
        \begin{equation*}
          D = 4 + 16a^2 = \left( 2\sqrt{1 + 4a^2} \right)^2.
        \end{equation*}
        Тогда корни правого множителя:
        \begin{equation*}
          t_{2,3} = \frac{2 \pm 2\sqrt{1 + 4a^2}}{2} = 1 \pm \sqrt{1 + 4a^2},\text{ или }t_2 = 1 + \sqrt{1 + 4a^2},\ t_3 = 1 - \sqrt{1 + 4a^2}.
        \end{equation*}
        То есть, нашли вид нашей квадратичной формы, уже приведённой к главным осям. Он такой:
        \begin{equation*}
          3x^2 + \left(1 + \sqrt{1 + 4a^2} \right)y^2 + \left(1 - \sqrt{1 + 4a^2} \right)z^2.
        \end{equation*}
        Вспомним канонический вид эллиптического параболоида (в начале решения). Тогда мы хотим, чтобы два слагаемых при квадратах были положительны, а третье равнялось нулю. Занулить мы можем только третье слагаемое, взяв $\bm{a = 0}$. Так и сделаем. Тогда получим такой итоговый вид квадратичной формы:
        \begin{equation*}
          3x^2 + 2y^2
        \end{equation*}
        Теперь найдём замену, которая приводит квадратичную форму в такой вид. Переберём собственные значения и найдём собственные векторы:
        \begin{itemize}
          \item $t = 3$. Ищем ФСР $A - 3E = 
          \begin{pmatrix}
              -3 & 0 & 0 \\
              0 & 0 & 0 \\
              0 & 0 & -1
          \end{pmatrix}$. ФСР: $v_1 = \begin{pmatrix}
            0 \\
            1 \\
            0
          \end{pmatrix}$.
          \item $t = 2$. Ищем ФСР $A - 2E = 
          \begin{pmatrix}
              -2 & 0 & 0 \\
              0 & 1 & 0 \\
              0 & 0 & 0
          \end{pmatrix}$. ФСР: $v_2 = \begin{pmatrix}
            0 \\
            0 \\
            1
          \end{pmatrix}$.
          \item $t = 0$. Ищем ФСР $A - 0E = 
          \begin{pmatrix}
              0 & 0 & 0 \\
              0 & 3 & 0 \\
              0 & 0 & 2
          \end{pmatrix}$. ФСР: $v_3 = \begin{pmatrix}
            1 \\
            0 \\
            0
          \end{pmatrix}$.
        \end{itemize}
        То есть, матрица перехода от старого базиса к новому:
        \begin{equation*}
          C = \begin{pmatrix}
            0 & 0 & 1 \\
            1 & 0 & 0 \\
            0 & 1 & 0
          \end{pmatrix},
        \end{equation*}
        и $x = Cx'$~--- выражение старых координат через промежуточные.
        Теперь мы знаем первоначальную замену и чему равно $a$. Применим их к исходному многочлену, получим:
        \begin{equation*}
          3x'^2 + 2y'^2 + 8z' - 6x' - 9 = 0
        \end{equation*}
        Такую штуку уже можно привести к эллиптическому параболоиду чем-то похожим на метод Лагранжа:
        \begin{align*}
          3x'^2 + 2y'^2 + 8z' - 6x' - 9 &= (3x'^2 - 6x' + 3) + 2y'^2 + 8z' - 12 = 3(x' - 1)^2 + 2y'^2 + 8z' - 12
        \end{align*}
        Но мы-то хотим к нулю приравнять:
        \begin{align*}
          3(x' - 1)^2 + 2y'^2 + 8z' - 12 &= 0 \\
          3(x' - 1)^2 + 2y'^2 &= 12 - 8z' \\
          3(x' - 1)^2 + 2y'^2 &= 2(6 - 4z')
        \end{align*}
        Тогда выражение итоговых координат через промежуточные:
        \begin{align*}
          \begin{cases}
            x'' = x' - 1 \\
            y'' = y' \\
            z'' = 6 - 4z'
          \end{cases}
        \end{align*}
        Выразим в обратную сторону:
        \begin{align*}
          \begin{cases}
            x' = x'' + 1 \\
            y' = y'' \\
            z' = 3/2 - (1/4)z''
          \end{cases}
        \end{align*}
        Но ещё мы помним выражение старых через промежуточные:
        \begin{align*}
          \begin{cases}
            z = y' \\
            y = x' \\
            x = z'
          \end{cases}
        \end{align*}
        Тогда выражение исходных через итоговые:
        \begin{align*}
          \begin{cases}
            \bm{x = 3/2 - (1/4)z''} \\
            \bm{y = x'' + 1} \\
            \bm{z = y''}
          \end{cases}
        \end{align*}        
        \end{proof}

    % Оля
    \section*{Задача \textnumero 8}
        % текст задачи
        Линейный оператор $\varphi \colon \RR^4 \to \RR^4$ имеет в стандартном базисе матрицу
        $$
        \begin{pmatrix}
            3 & 1 & 0 & -2 \\
            0 & 1 & 0 & 4 \\
            6 & 4 & 3 & -5 \\
            0 & -1 & 0 & 5 \\
        \end{pmatrix}.
        $$
        Найдите базис пространства $\RR^4$, в котором матрица оператора $\varphi$ 
        имеет жорданову форму и укажите эту жорданову форму.
        \begin{proof}[Решение.] \ 
            \begin{enumerate}
                \item Для начала определим собственные значения на диагонали, 
                посчитав характеристический многочлен

                \begin{multline*} 
                \chi_{A} = 
                 \begin{vmatrix}
                    3-t & 1 & 0 & -2 \\
                    0 & 1-t & 0 & -5\\
                    6 & 4 & 3-t & -5 \\
                    0 & -1 & 0 & 5-t \\
                 \end{vmatrix} = (3-t)
                 \begin{vmatrix}
                    3-t & 1 & -2 \\
                    0 & 1-t & -5\\
                    0 & -1  & 5-t \\
                 \end{vmatrix} =
                 (3 - t)^2
                 \begin{vmatrix}
                    1-t & -5\\
                    -1  & 5-t \\
                 \end{vmatrix} = \\ =
                 (3 - t)^2 ((1 - t)(5-t) + 4) = (3 - t)^2 (5 + t^2 -5t - t + 4) = (3 - t)^4
                \end{multline*}
                Получаем, что собственное значение только одно, $\lambda = 3$. 
                \item Найдем $V_\lambda(\varphi)$. Пусть $B = (A - \lambda E)$. Тогда $\dim V_\lambda = n - \text{rk}B$ 
                --- число жордановых клеток. Если вспомнить про обозначение $d_i = \dim Ker(\varphi - \lambda \cdot Id)^i$, 
                то сейчас мы в точности ищем $d_1$. Ну и заодно базис $\text{Ker} B$.
                \begin{multline*}
                    \begin{pmatrix}
                        0 & 1 & 0 & -2 \\
                        0 & -2 & 0 & -5\\
                        6 & 4 & 0 & -5 \\
                        0 & -1 & 0 & 2 \\
                    \end{pmatrix} \to 
                    \begin{pmatrix}
                        6 & 4 & 0 & -5 \\
                        0 & 1 & 0 & -2 \\
                    \end{pmatrix}
                    \to 
                    \begin{pmatrix}
                        6 & 0 & 0 & 3 \\
                        0 & 1 & 0 & -2 \\
                    \end{pmatrix}
                    \to 
                    \begin{pmatrix}
                        2 & 0 & 0 & 1 \\
                        0 & 1 & 0 & -2 \\
                    \end{pmatrix} 
                     \Rightarrow
                    \text{ФСР: }
                    v_1 = \begin{pmatrix}
                        -1\\
                        4\\
                        0\\
                        2
                    \end{pmatrix}, v_2 = 
                    \begin{pmatrix}
                        0\\
                        0\\
                        1\\
                        0
                    \end{pmatrix}\\
                \end{multline*}
                Получаем, что количество жордановых клеток равно 2.
                \item А еще знаем, что количество клеток размера k вычисляется по формуле 
                $2d_k - d_{k- 1} - d_{k + 1}$. Надо бы найти $d_2$, 
                чтобы понять сколько клеток размера 1 будет (тк надо решить --- клетки $2 \times 2$ и $2 \times 2$ или клетки 
                $1 \times 1$ и $3 \times 3$).
                Для этого посчитаем $B^2$. (Именно B, не A!)
                $$
                B^2 = 
                \begin{pmatrix}
                    0 & 1 & 0 & -2 \\
                    0 & -2 & 0 & -5\\
                    6 & 4 & 0 & -5 \\
                    0 & -1 & 0 & 2 \\
                \end{pmatrix}
                \cdot 
                \begin{pmatrix}
                    0 & 1 & 0 & -2 \\
                    0 & -2 & 0 & -5\\
                    6 & 4 & 0 & -5 \\
                    0 & -1 & 0 & 2 \\
                \end{pmatrix}
                = 
                \begin{pmatrix}
                    0 & 0 & 0 & 0 \\
                    0 & 0 & 0 & 0 \\
                    0 & 3 & 0 & -6 \\
                    0 & 0 & 0 & 0 \\
                \end{pmatrix} \to 
                \begin{pmatrix}
                    0 & 3 & 0 & -6 \\
                \end{pmatrix}
                \Rightarrow d_2 = 3$$
                Итого клеток размера 1: $2 \cdot 2 - 3 = 1$ ($d_0$ полагаем нулем). 
                $\Rightarrow$ ЖНФ: $A^\prime = 
                \begin{pmatrix}
                    3 & 1 & 0 & 0 \\
                    0 & 3 & 1 & 0 \\
                    0 & 0 & 3 & 0 \\
                    0 & 0 & 0 & 3 \\
                \end{pmatrix}$
                \item Теперь ищем базис f. Нам нужен вектор высоты 3. Вообще его надо было бы искать в 
                Ker $B^3$, но так как у нас корневое подпространство --- все пространство, но поищем 
                в стандартном базисе.
                Порадуемся моей интуиции и возьмем $f_3$ = $e_2 = 
                \begin{pmatrix}
                    0\\
                    1\\
                    0\\
                    0
                \end{pmatrix}$.
                Проверим, что высота и правда 3. 
                $$f_2 = Bf_3 = 
                \begin{pmatrix}
                    0 & 1 & 0 & -2 \\
                    0 & -2 & 0 & -5\\
                    6 & 4 & 0 & -5 \\
                    0 & -1 & 0 & 2 \\
                \end{pmatrix}
                \begin{pmatrix}
                    0\\
                    1\\
                    0\\
                    0
                \end{pmatrix} =
                \begin{pmatrix}
                    1\\
                    -2\\
                    4\\
                    -1
                \end{pmatrix}
                \text{  -- все ок, вектор не обнулился}
                $$
                $$f_1 = B^2f_3 = 
                \begin{pmatrix}
                    0 & 0 & 0 & 0 \\
                    0 & 0 & 0 & 0 \\
                    0 & 3 & 0 & -6 \\
                    0 & 0 & 0 & 0 \\
                \end{pmatrix}
                \begin{pmatrix}
                    0\\
                    1\\
                    0\\
                    0
                \end{pmatrix} =
                \begin{pmatrix}
                    0\\
                    0\\
                    3\\
                    0
                \end{pmatrix}
                \text{  -- все ок, вектор не обнулился}
                $$ 
                Получается, то высота $f_3$ и правда равна 3. (Если 
                ли бы у меня не вышло с первого раза подобрать я бы пошла 
                дальше проверять векторы из стандартного базиса / Ker $B^i$.)
                \item Осталось определить $f_4$. его надо выбрать так, чтобы $f_4 \in \text{Ker} B$ 
                и дополнял $f_1$ до базиса этого подпространства. 
                Можно сделать это строго по алгоритму дополнения вектора до базиса подпространства, 
                а можно просто посмотреть на базис Ker B, который мы нашли в пункте 2 увидеть, что 
                $f_1$ линейно выражается через $v_2 (f_1= 3v_2)$, а значит можно взять $f_4 = v_1 = 
                \begin{pmatrix}
                    -1\\
                    4\\
                    0\\
                    2
                \end{pmatrix}$
                \item Осталось проверить, что мы действительно получили базис. 
                (Сделайте этот сами, умоляю. \textbf{А если не сделаете, баллы снимут.})
            \end{enumerate}

            \textbf{Ответ:} ЖНФ: $A^\prime = 
            \begin{pmatrix}
                3 & 1 & 0 & 0 \\
                0 & 3 & 1 & 0 \\
                0 & 0 & 3 & 0 \\
                0 & 0 & 0 & 3 \\
            \end{pmatrix}$

            Базис: ($f_1, f_2, f_3, f_4$) = $
            \begin{pmatrix}
                0\\
                0\\
                3\\
                0
            \end{pmatrix}, 
            \begin{pmatrix}
                1\\
                -2\\
                4\\
                -1
            \end{pmatrix}, 
            \begin{pmatrix}
                0\\
                1\\
                0\\
                0
            \end{pmatrix}, 
            \begin{pmatrix}
                -1\\
                4\\
                0\\
                2
            \end{pmatrix}
            $
        \end{proof}
      
\end{document}
