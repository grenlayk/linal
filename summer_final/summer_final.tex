\documentclass[a4paper]{article}
\usepackage{header}
\newcommand{\Mat}{\operatorname{Mat}}
\newcommand{\M}{\operatorname{M}}
\allowdisplaybreaks

%% Графика
\usepackage{graphicx}       
\graphicspath{{images/}}            
\usepackage{tikz}  
\usetikzlibrary{patterns}                 
\usepackage{pgfplots}              
\usepackage{circuitikz}
\usepackage{nicefrac}
\usepackage{bm}
            
\usepackage{titlesec}
\titlespacing*{\section}{0pt}{0pt}{0pt}

\usetikzlibrary{quotes,angles}
\usetikzlibrary{positioning,intersections}
\usetikzlibrary{through}

\textwidth=19.0cm \oddsidemargin=-1.3cm
\textheight=26cm \topmargin=-2.5cm

\newtheorem{task}{Задание}
\newtheorem*{task*}{Задание}

\theoremstyle{remark}

\newtheorem{remark}{Замечание}
\newtheorem*{remark*}{Замечание}
\newtheorem{commentarium}{Комментарий}
\newtheorem*{commentarium*}{Комментарий}

\usepackage{tikz-cd}

\newenvironment{sysmatrix}[1]
{
    \left(\begin{array}{@{}#1@{}}
}
{\end{array}\right)}
\newcommand{\smt}[2]{\begin{sysmatrix}{#1} #2\end{sysmatrix}}
\newcommand{\eq}[1]{\begin{cases} #1 \end{cases}}

\newcommand{\elon}[3]{%
  \ensuremath{\text{Э}_1(#1,\; #2,\; #3)}%
}
\newcommand{\eltw}[2]{%
  \ensuremath{\text{Э}_2(#1,\; #2)}%
}
\newcommand{\elth}[2]{%
  \ensuremath{\text{Э}_3(#1,\; #2)}%
}
 
\newcommand{\eqon}[3]{%
  \ensuremath{\overset{\text{Э}_1(#1,\; #2,\; #3)}{=}}%
}
\newcommand{\eqtw}[2]{%
  \ensuremath{\overset{\text{Э}_2(#1,\; #2)}{=}}%
}
\newcommand{\eqth}[2]{%
  \ensuremath{\overset{\text{Э}_3(#1,\; #2)}{=}}%
}

\newcommand{\arron}[3]{%
  \ensuremath{\xrightarrow{\text{Э}_1(#1,\; #2,\; #3)}}%
}
\newcommand{\arrtw}[2]{%
  \ensuremath{\xrightarrow{\text{Э}_2(#1,\; #2)}}%
}
\newcommand{\arrth}[2]{%
  \ensuremath{\xrightarrow{\text{Э}_3(#1,\; #2)}}%
}

\makeatletter
\renewcommand*\env@matrix[1][*\c@MaxMatrixCols c]{%
  \hskip -\arraycolsep
  \let\@ifnextchar\new@ifnextchar
  \array{#1}}
\makeatother

% tasks here https://www.dropbox.com/s/8yhwwpf3dvft1yd/LA_19-20_Stream2_Test2.pdf?dl=0
\title{Летний экзамен учебного года 2019/2020\\Вариант \textnumero 1}
\author{	
  % ОТКОММЕНТИРУЙ СЕБЯ
    Алиса Вернигор       \\ \href{https://t.me/allisyonok}{Telegram} \and
	Сергей Лоптев        \\ \href{https://t.me/beast_sl}{Telegram} \and
	Оля Козлова        \\ \href{https://t.me/grenlayk}{Telegram}
}

\date{}

\begin{document}
	\maketitle
    % Алиса
    \section*{Задача \textnumero 1}
        % текст задачи
        Найдите базис пересечения двух подпространств $L_1, L_2 \subset \RR^4$,
        где $L_1$ состоит из всех решений уравнения $2x_1 - x_2 - x_3 + 2x_4 = 0$, 
        а $L_2$ есть линейная оболочка векторов $(3, 2, 3, -2), (2, 2, 3, 1), 
        (3, 1, 4, -3)$.
        \begin{proof}[Решение.] \ 
            

        \end{proof}
    % Сережа
    \section*{Задача \textnumero 2}
        % текст задачи
        Найдите невырожденную замену координат (выражение старых координат через 
        новые), приводящие квадратичную форму $Q(x, y, z) = 16x^2 + 9y^2 + z^4 - 24xy + 8xz$
        к нормальному виду, и выпишите этот вид.
        \begin{proof}[Решение.] \ 
            

        \end{proof}

    % Оля
    \section*{Задача \textnumero 3}
        % текст задачи
        Найдите псевдорешение для следующей системы линейных уравнений: 
        $\eq{
            2x_1 - x_2 = 1\\
            -x_1 + 3x_2 = -2 \\
            x_1 + x_2 = 1 \\
            -2x_1 + 3x_2 = 2
        }$
        \begin{proof}[Решение.] \ 
            

        \end{proof}
    
    % Алиса
    \section*{Задача \textnumero 4}
        % текст задачи
        Приведите пример двух диагонализуемых линейных операторов $\varphi$ и 
        $\psi$ в $\RR^2$, для которых оператор $4\varphi + 7\psi$ недиагонализуем.
        \begin{proof}[Решение.] \ 
            

        \end{proof}
    
    % Сережа
    \section*{Задача \textnumero 5}
        % текст задачи
        Ортогональный линейный оператор $\varphi \colon \RR^3 \to \RR^3$ имеет в стандартном
        базисе матрицу 
        $$
        \begin{pmatrix}
            \nicefrac23 & -\nicefrac23 & \nicefrac13 \\
            \nicefrac23 & \nicefrac13 & \nicefrac23 \\
            \nicefrac13 & \nicefrac23 & -\nicefrac23 \\
        \end{pmatrix}
        $$
        Найдите ортонормированный базис, в котором матрица оператора $\varphi$ имеет 
        канонический вид и выпишите эту матрицу. Укажите ось и угол поворота, определяемого 
        оператором $\varphi$.
        \begin{proof}[Решение.] \ 
            

        \end{proof}

    % Оля
    \section*{Задача \textnumero 6}
        % текст задачи
        Приведите пример матрицы $A \in \Mat_{2\times 3}(\RR)$ ранга 2, для которой 
        ближайшей по норме Фробениуса матрицей ранга 1 будет матрица
        $$
        \begin{pmatrix}
            -6 & 3 & 3 \\
            3 & -1 & -1 \\
        \end{pmatrix}.
        $$
        \begin{proof}[Решение.] \ 
            

        \end{proof}

    % Оля
    \section*{Задача \textnumero 7}
        % текст задачи
        Найдите все значения параметра $a$, при которых уравнение 
        $$ 
        3y^2 + 2z^2 -4axz + 8x - 6y - 9
        $$
        определяет эллиптический параболоид в $\RR^3$. Для каждого найденного значения  $a$
        укажите прямоугольную декартову систему координат в $\RR^3$ (выражение старых координат 
        через новые), в которой данное уравнение принимает канонический вид. 
        \begin{proof}[Решение.] \ 
            

        \end{proof}
    
    % Сережа
    \section*{Задача \textnumero 8}
        % текст задачи
        Линейный оператор $\varphi \colon \RR^4 \to \RR^4$ имеет в стандартном базисе матрицу
        $$
        \begin{pmatrix}
            3 & 1 & 0 & -2 \\
            0 & 1 & 0 & 4 \\
            6 & 4 & 3 & -5 \\
            0 & -1 & 0 & 5 \\
        \end{pmatrix}.
        $$
        Найдите базис пространства $\RR^4$, в котором матрица оператора $\varphi$ 
        имеет жорданову форму и укажите эту жорданову форму.
        \begin{proof}[Решение.] \ 
            

        \end{proof}
      
\end{document}
