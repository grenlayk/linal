\documentclass[a4paper]{article}
\usepackage{header}
\newcommand{\Mat}{\operatorname{Mat}}
\newcommand{\M}{\operatorname{M}}

%% Графика
\usepackage{graphicx}       
\graphicspath{{images/}}            
\usepackage{tikz}  
\usetikzlibrary{patterns}                 
\usepackage{pgfplots}              
\usepackage{circuitikz}
            

\usetikzlibrary{quotes,angles}
\usetikzlibrary{positioning,intersections}
\usetikzlibrary{through}

\textwidth=19.0cm \oddsidemargin=-1.3cm
\textheight=25cm \topmargin=-2.5cm

\newtheorem{task}{Задание}
\newtheorem*{task*}{Задание}

\theoremstyle{remark}

\newtheorem{remark}{Замечание}
\newtheorem*{remark*}{Замечание}
\newtheorem{commentarium}{Комментарий}
\newtheorem*{commentarium*}{Комментарий}

\usepackage{tikz-cd}

\newenvironment{sysmatrix}[1]
{
    \left(\begin{array}{@{}#1@{}}
}
{\end{array}\right)}
\newcommand{\smt}[2]{\begin{sysmatrix}{#1} #2\end{sysmatrix}}
\newcommand{\eq}[1]{\begin{cases} #1 \end{cases}}

\newcommand{\elon}[3]{%
  \ensuremath{\text{Э}_1(#1,\; #2,\; #3)}%
}
\newcommand{\eltw}[2]{%
  \ensuremath{\text{Э}_2(#1,\; #2)}%
}
\newcommand{\elth}[2]{%
  \ensuremath{\text{Э}_3(#1,\; #2)}%
}
 
\newcommand{\eqon}[3]{%
  \ensuremath{\overset{\text{Э}_1(#1,\; #2,\; #3)}{=}}%
}
\newcommand{\eqtw}[2]{%
  \ensuremath{\overset{\text{Э}_2(#1,\; #2)}{=}}%
}
\newcommand{\eqth}[2]{%
  \ensuremath{\overset{\text{Э}_3(#1,\; #2)}{=}}%
}

\newcommand{\arron}[3]{%
  \ensuremath{\xrightarrow{\text{Э}_1(#1,\; #2,\; #3)}}%
}
\newcommand{\arrtw}[2]{%
  \ensuremath{\xrightarrow{\text{Э}_2(#1,\; #2)}}%
}
\newcommand{\arrth}[2]{%
  \ensuremath{\xrightarrow{\text{Э}_3(#1,\; #2)}}%
}


\title{Зимний экзамен учебного года 2019/2020\\Вариант \textnumero 1}
\author{	
        % ОТКОММЕНТИРУЙ СЕБЯ
    % % Александр Богданов   \\ \href{https://t.me/SphericalPotatoInVacuum}{Telegram} \and
    Алиса Вернигор       \\ \href{https://t.me/allisyonok}{Telegram} \and
    % % Анастасия Григорьева \\ \href{https://t.me/weifoll}{Telegram} \and
    % Василий Шныпко       \\ \href{https://t.me/yourvash}{Telegram} \and
    % % Данил Казанцев       \\ \href{https://t.me/vserosbuybuy}{Telegram} \and
    % Денис Козлов         \\ \href{https://t.me/DKozl50}{Telegram} \and
    % Елизавета Орешонок   \\ \href{https://t.me/eaoresh}{Telegram} \and
    % % Иван Пешехонов       \\ \href{https://t.me/JohanDDC}{Telegram} \and
    % % Иван Добросовестнов  \\ \href{https://t.me/ivankot13}{Telegram} \and
    % % Настя Городилова     \\ \href{https://t.me/nastygorodi}{Telegram} \and
    % Никита Насонков      \\ \href{https://t.me/nnv_nick}{Telegram} \and
    % Даниэль Хайбулин      \\ \href{https://t.me/kiDaniel}{Telegram} \and
	Сергей Лоптев        \\ \href{https://t.me/beast_sl}{Telegram} \and
	Ольга Козлова        \\ \href{https://t.me/grenlayk}{Telegram}
	% Сабина Даянова        \\ \href{https://t.me/sabinadayanova}{Telegram} \and
}

\date{}

\begin{document}
	\maketitle

    \section*{Задача \textnumero 1}
        % текст задачи
        \begin{proof}[Решение.]
		
        \end{proof}	 
    
    \section*{Задача \textnumero 2}
        % текст задачи
	    \begin{proof}[Решение.]
		
        \end{proof}
    
    \section*{Задача \textnumero 3}
      Известно, что векторы  $v₁, v₂, v₃$ и $v₄$ некоторого векторного
      пространства над $\mathbb{R}$ линейно независимы. Могут ли
      векторы \\
      \[u₁ = 3v₁ + 2v₂ + v₄, u₂ = 2v₁ + v₂ + 3v₃ - v₄,
       u₃ = v₁ + v₂ - v₃ \] 
       быть линейно зависимыми? \textbf{Отет обоснуйте.}
      \begin{proof}[Решение.] \ \\
        \begin{itemize}
            \item  Мы знаем, что всякая линейно независимая система векторов
            является базисом своей линейной оболочки (этот факт упоминался
            на лекциях).
            \item Таким образом, векторы $v₁, v₂, v₃$ и $v₄$ являются
            базисом своей линейной оболочки -- назовём её $S$. 
            \item Заметим, что векторы $u₁, u₂, u₃ ∈ S$ (они представимы
            в виде линейной комбинации векторов $v₁, v₂, v₃$ и $v₄$, являющихся
            базисом $S$). 
            \item В базисе $v₁, v₂, v₃, v₄$ наши векторы $u₁, u₂, u₃$ 
            представимы следующим образом: \\
            $u₁ = \begin{pmatrix}
              3 \\
              2 \\ 
              0 \\ 
              1 
            \end{pmatrix}$ ,
            $u₂ = \begin{pmatrix}
              2 \\
              1 \\ 
              3 \\ 
              -1 
            \end{pmatrix}$ ,
            $u₃ = \begin{pmatrix}
              1 \\
              1 \\ 
              -1 \\ 
              0 
            \end{pmatrix}$
            \item Таким образом, мы свели нашу задачу к задаче о проверке 
            системы векторов на линейную независимость (просто в немного
            странном базисе).
            \item Вопрос о линейной зависимости конечного набора векторов
            сводится к составлению ОСЛУ и вопросу о наличии у него 
            ненулевого решения.
            \item Запишем векторы по столбцам и применим наш любимый метод
            Гаусса: \\
            $\begin{pmatrix}
              3 & 2 & 1\\
              2 & 1 & 1 \\
              0 & 3 & -1 \\
              1 & -1 & 0  
            \end{pmatrix}
            \rightsquigarrow
            \begin{pmatrix}
              1 & -1 & 0\\
              2 & 1 & 1 \\
              0 & 3 & -1 \\
              3 & 2 & 1
            \end{pmatrix}
            \rightsquigarrow
            \begin{pmatrix}
              1 & -1 & 0\\
              0 & 3 & 1 \\
              0 & 3 & -1 \\
              0 & 5 & 1
            \end{pmatrix}
            \rightsquigarrow
            \begin{pmatrix}
              1 & -1 & 0\\
              0 & 3 & 1 \\
              0 & 0 & 2 \\
              0 & 5 & 1
            \end{pmatrix}$
            \item Видим, что у нас 3 главных переменных (а неизвестных
            тоже 3) $⇒$ ОСЛУ имеет единственное решение -- нулевое $⇒$
            векторы линейно независимы.
        \end{itemize}   
        Внимательно смотрим на вопрос и пишем подходящий ответ: \\ 
        \textbf{Ответ:} Нет, не могут.
      \end{proof}
      \remark{
        Прийти к такой системе можно было и другим способом:
       \begin{itemize}
           \item Знаем, что векторы линейно зависимы тогда
           и только тогда, когда существует их нетривиальная (такая, что
           $∃i: \alpha_i ≠ 0$, ненулевая) линейная комбинации, равная 0
           \item То есть, тогда и только тогда, когда $∃a, b, c ∈ \mathbb{R}: 
           au₁ + bu₂ + cu₃ = 0$ и $(a ≠ 0) ∨ (b ≠ 0) ∨ (c ≠ 0)$
           \item Заметим, что: \\
           $u₁ = (v₁, v₂, v₃, v₄) ⋅ \begin{pmatrix}
             3 \\
             2 \\
             0 \\ 
             1
           \end{pmatrix}, u₂ = (v₁, v₂, v₃, v₄) ⋅ \begin{pmatrix}
            2 \\
            1 \\
            3 \\ 
            -1
          \end{pmatrix}, u₁ = (v₁, v₂, v₃, v₄) ⋅ \begin{pmatrix}
            1 \\
            1 \\
            -1 \\ 
            0
          \end{pmatrix}$ 
          \item Тогда имеем: \\
           $au₁ + bu₂ + cu₃ = a ⋅ (v₁, v₂, v₃, v₄) ⋅ \begin{pmatrix}
            3 \\
            2 \\
            0 \\ 
            1
          \end{pmatrix} + b ⋅ (v₁, v₂, v₃, v₄) ⋅ \begin{pmatrix}
           2 \\
           1 \\
           3 \\ 
           -1
         \end{pmatrix} + c ⋅ (v₁, v₂, v₃, v₄) ⋅ \begin{pmatrix}
           1 \\
           1 \\
           -1 \\ 
           0
         \end{pmatrix} = \\ = (v₁, v₂, v₃, v₄)\left(
          a  ⋅ \begin{pmatrix}
            3 \\
            2 \\
            0 \\ 
            1
          \end{pmatrix} + b  ⋅ \begin{pmatrix}
           2 \\
           1 \\
           3 \\ 
           -1
         \end{pmatrix} + c  ⋅ \begin{pmatrix}
           1 \\
           1 \\
           -1 \\ 
           0
         \end{pmatrix}
         \right)$
         \item Такое выражение равно 0, если:
         \begin{itemize}
             \item $(v₁, v₂, v₃, v₄) = \overline{0}$ -- сразу нет (у нас
             же линейно независимая система) 
             \item Второй множитель равен 0 -- вот это наша остановочка
         \end{itemize}
         \item Заметим, что: \\
         $ a  ⋅ \begin{pmatrix}
          3 \\
          2 \\
          0 \\ 
          1
        \end{pmatrix} + b  ⋅ \begin{pmatrix}
         2 \\
         1 \\
         3 \\ 
         -1
       \end{pmatrix} + c  ⋅ \begin{pmatrix}
         1 \\
         1 \\
         -1 \\ 
         0
       \end{pmatrix} =  \begin{pmatrix}
        3 & 2 & 1\\
        2 & 1 & 1 \\
        0 & 3 & -1 \\
        1 & -1 & 0  
      \end{pmatrix} ⋅ \begin{pmatrix}
        a \\
        b \\
        c  
      \end{pmatrix}$
      \item Отсюда получаем ту же ОСЛУ, что и в решении выше.  
       \end{itemize}
      }
    \section*{Задача \textnumero 4}
        % текст задачи
	    \begin{proof}[Решение.]
		
        \end{proof}
    
    \section*{Задача \textnumero 5}
      В пространстве $\mathbb{R}^5$ даны векторы 
      \[v₁ = (1, 2, 1, 3, 3), v₂ = (3, 7, 3, 5, 4),
      v₃ = (1, 1, 1, 7, 8), v₄ = (3, 4, 3, 8, 2).\]
      \begin{itemize}
          \item[(а)] Выберите среди данных векторов базис их линейной оболочки.
          \item[(б)] Дополните полученный в пункте (а) базис до базиса всего
          пространства $\mathbb{R}^5$. 
      \end{itemize}
	    \begin{proof}[Решение.] \ \\
        \begin{itemize}
            \item[(а)] Алгоритм решения такой:
            \begin{itemize}
                \item Укладываем векторы в столбцы
                \item Приводим матрицу к улучшенному ступеньчатому виду
                методом Гаусса (преобразованиями строк)
                \item Номера столбцов ведущих переменных и будут являться 
                номерами векторов, которые мы должны взять в базис
            \end{itemize} 
            $
              \begin{pmatrix}
                1 & 3 & 1 & 3 \\ 
                2 & 7 & 1 & 4 \\
                1 & 3 & 1 & 3 \\
                3 & 5 & 7 & 8 \\
                3 & 4 & 8 & 2
              \end{pmatrix} \rightsquigarrow
              \begin{pmatrix}
                1 & 3 & 1 & 3 \\ 
                0 & 1 & -1 & -2 \\
                0 & 0 & 0 & 0 \\
                0 & -4 & 4 & -1 \\
                0 & -5 & 5 & -7
              \end{pmatrix} \rightsquigarrow
              \begin{pmatrix}
                1 & 0 & 4 & 9 \\ 
                0 & 1 & -1 & -2 \\
                0 & 0 & 0 & 0 \\
                0 & 0 & 0 & -9 \\
                0 & 0 & 0 & -17
              \end{pmatrix} \rightsquigarrow
              \begin{pmatrix}
                1 & 0 & 4 & 0 \\ 
                0 & 1 & -1 & 0 \\
                0 & 0 & 0 & 1 \\
                0 & 0 & 0 & 0 \\
                0 & 0 & 0 & 0
              \end{pmatrix}
            ⇒$ \\ $⇒$в качестве базиса выбираем $v₁, v₂, v₄$
            \item[(б)] Алгоритм тут такой:
              \begin{itemize}
                  \item Укладываем полученный в пункте (а) базис по строкам
                  \item Приводим полученную матрицу к ступеньчатому виду
                  преобразованиями строк
                  \item Дополняем базис базисными единичками с номерами НЕГЛАВНЫХ
                  переменных нашей системы
              \end{itemize}  
              $
              \begin{pmatrix}
                1 & 2 & 1 & 3 & 3 \\ 
                3 & 7 & 3 & 5 & 4 \\
                3 & 4 & 3 & 8 & 2 
              \end{pmatrix}
              \rightsquigarrow
              \begin{pmatrix}
                1 & 2 & 1 & 3 & 3 \\ 
                0 & 1 & 0 & -4 & -5 \\
                0 & -2 & 0 & -1 & -7 
              \end{pmatrix}
              \rightsquigarrow
              \begin{pmatrix}
                1 & 0 & 1 & 11 & 13 \\ 
                0 & 1 & 0 & -4 & -5 \\
                0 & 0 & 0 & -9 & -17 
              \end{pmatrix}
              $ \\
              Свободные переменные имеют номера 3 и 5 $⇒$ матричными единичками
              с такими номерами и необходимо дополнить наш базис. \\ 
              Дополняющие векторы: \\
              \[
                e₃ = \begin{pmatrix}
                  0 \\ 
                  0\\
                  1 \\
                  0 \\ 
                  0
                \end{pmatrix}, 
                e₅ = \begin{pmatrix}
                  0 \\ 
                  0\\
                  0 \\
                  0 \\ 
                  1
                \end{pmatrix}
                \]  
        \end{itemize}
      \end{proof}

    \section*{Задача \textnumero 6}
        % текст задачи
	    \begin{proof}[Решение.]
		
        \end{proof}

    \section*{Задача \textnumero 7}
        % текст задачи
	    \begin{proof}[Решение.]
		
        \end{proof}

    \section*{Задача \textnumero 8}
        % текст задачи
	    \begin{proof}[Решение.]
		
        \end{proof}
 	
\end{document}