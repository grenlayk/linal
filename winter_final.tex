\documentclass[a4paper]{article}
\usepackage{header}
\newcommand{\Mat}{\operatorname{Mat}}
\newcommand{\M}{\operatorname{M}}

%% Графика
\usepackage{graphicx}       
\graphicspath{{images/}}            
\usepackage{tikz}  
\usetikzlibrary{patterns}                 
\usepackage{pgfplots}              
\usepackage{circuitikz}
\usepackage{bm}
            

\usetikzlibrary{quotes,angles}
\usetikzlibrary{positioning,intersections}
\usetikzlibrary{through}

\textwidth=19.0cm \oddsidemargin=-1.3cm
\textheight=25cm \topmargin=-2.5cm

\newtheorem{task}{Задание}
\newtheorem*{task*}{Задание}

\theoremstyle{remark}

\newtheorem{remark}{Замечание}
\newtheorem*{remark*}{Замечание}
\newtheorem{commentarium}{Комментарий}
\newtheorem*{commentarium*}{Комментарий}

\usepackage{tikz-cd}

\newenvironment{sysmatrix}[1]
{
    \left(\begin{array}{@{}#1@{}}
}
{\end{array}\right)}
\newcommand{\smt}[2]{\begin{sysmatrix}{#1} #2\end{sysmatrix}}
\newcommand{\eq}[1]{\begin{cases} #1 \end{cases}}

\newcommand{\elon}[3]{%
  \ensuremath{\text{Э}_1(#1,\; #2,\; #3)}%
}
\newcommand{\eltw}[2]{%
  \ensuremath{\text{Э}_2(#1,\; #2)}%
}
\newcommand{\elth}[2]{%
  \ensuremath{\text{Э}_3(#1,\; #2)}%
}
 
\newcommand{\eqon}[3]{%
  \ensuremath{\overset{\text{Э}_1(#1,\; #2,\; #3)}{=}}%
}
\newcommand{\eqtw}[2]{%
  \ensuremath{\overset{\text{Э}_2(#1,\; #2)}{=}}%
}
\newcommand{\eqth}[2]{%
  \ensuremath{\overset{\text{Э}_3(#1,\; #2)}{=}}%
}

\newcommand{\arron}[3]{%
  \ensuremath{\xrightarrow{\text{Э}_1(#1,\; #2,\; #3)}}%
}
\newcommand{\arrtw}[2]{%
  \ensuremath{\xrightarrow{\text{Э}_2(#1,\; #2)}}%
}
\newcommand{\arrth}[2]{%
  \ensuremath{\xrightarrow{\text{Э}_3(#1,\; #2)}}%
}

\makeatletter
\renewcommand*\env@matrix[1][*\c@MaxMatrixCols c]{%
  \hskip -\arraycolsep
  \let\@ifnextchar\new@ifnextchar
  \array{#1}}
\makeatother


\title{Зимний экзамен учебного года 2019/2020\\Вариант \textnumero 1}
\author{	
        % ОТКОММЕНТИРУЙ СЕБЯ
    % % Александр Богданов   \\ \href{https://t.me/SphericalPotatoInVacuum}{Telegram} \and
    Алиса Вернигор       \\ \href{https://t.me/allisyonok}{Telegram} \and
    % % Анастасия Григорьева \\ \href{https://t.me/weifoll}{Telegram} \and
    % Василий Шныпко       \\ \href{https://t.me/yourvash}{Telegram} \and
    % % Данил Казанцев       \\ \href{https://t.me/vserosbuybuy}{Telegram} \and
    % Денис Козлов         \\ \href{https://t.me/DKozl50}{Telegram} \and
    % Елизавета Орешонок   \\ \href{https://t.me/eaoresh}{Telegram} \and
    % % Иван Пешехонов       \\ \href{https://t.me/JohanDDC}{Telegram} \and
    % % Иван Добросовестнов  \\ \href{https://t.me/ivankot13}{Telegram} \and
    % % Настя Городилова     \\ \href{https://t.me/nastygorodi}{Telegram} \and
    % Никита Насонков      \\ \href{https://t.me/nnv_nick}{Telegram} \and
    % Даниэль Хайбулин      \\ \href{https://t.me/kiDaniel}{Telegram} \and
	Сергей Лоптев        \\ \href{https://t.me/beast_sl}{Telegram} \and
	Ольга Козлова        \\ \href{https://t.me/grenlayk}{Telegram}
	% Сабина Даянова        \\ \href{https://t.me/sabinadayanova}{Telegram} \and
}

\date{}

\begin{document}
	\maketitle
    \section*{Задача \textnumero 1}
      % текст задачи
      \begin{proof}[Решение.]
		
      \end{proof}	 
    
    \section*{Задача \textnumero 2}
      % текст задачи
      Найдите все комплексные решения уравнения $(2\sqrt{3} + i)z^4 = - 6\sqrt{3} + 10i$, у которых одиниз аргументов принадлежит интервалу $(3\pi,\, 7\pi/2)$.
      \begin{proof}[Решение.]
        Решим уравнение:
        \begin{align*}
          &\left( 2\sqrt{3} + i \right)z^4 = - 6\sqrt{3} + 10i \\
          &z^4 = \frac{-6\sqrt{3}+10i}{2\sqrt{3} + i} = \frac{\left(2\sqrt{3}\right)\left( -6\sqrt{3} \right) + 10}{\left( 2\sqrt{3} \right)^2 + 1} + \frac{\left( 10 \cdot 2\sqrt{3} \right) - \left( -6\sqrt{3} \right)}{\left( 2\sqrt{3} \right)^2 + 1}i = \frac{-26}{13} + \frac{26\sqrt{3}}{13}i = -2 + 2\sqrt{3}i \\
          &\big| z^4 \big| = \sqrt{(-2)^2 + \left( 2\sqrt{3} \right)^2} = \sqrt{16} = 4 \\
          &\cos \varphi = \frac{-2}{4} = -\frac{1}{2}\quad \text{--- косинус аргумента}\\
          &\sin \varphi = \frac{2\sqrt{3}}{4} = \frac{\sqrt{3}}{2} \quad \text{--- синус аргумента} \\
          &\varphi = \frac{2\pi}{3} \quad \text{--- аргумент}\\
          &z^4 = \big| z^4 \big| (\cos \varphi + i \sin \varphi) = 4\left( \cos \frac{2\pi}{3} + i \sin \frac{2\pi}{3} \right) \\
          &\begin{aligned}
            z = \sqrt[4]{4\left( \cos \frac{2\pi}{3} + i \sin \frac{2\pi}{3} \right)} = \sqrt[4]{4} \left( \cos \frac{2\pi/3 + 2\pi k}{4} + i \sin \frac{2\pi/3 + 2\pi k}{4} \right) = \sqrt{2} \left( \cos \left( \frac{\pi}{6} + \frac{\pi}{2} k \right) + i \sin \left( \frac{\pi}{6} + \frac{\pi}{2} k \right) \right), &\\
            k = 0,\, 1,\, 2,\, 3 &
          \end{aligned}
        \end{align*}
        Аргументам, принадлежащим интервалу $(3\pi,\, 7\pi/2)$, с отрезка $[0,\, 2\pi]$ соответствуют аргументы, принадлежащие объединению полуинтервалов $(\pi,\, 2\pi] \cup [0,\, \pi/2)$. Значит, нам подходят $k = 0, 2, 3$, так как $\frac{\pi}{6},\, \frac{7\pi}{6},\, \frac{10\pi}{6} \in (\pi,\, 2\pi] \cup [0,\, \pi/2)$ и $\frac{4\pi}{6} \notin (\pi,\, 2\pi] \cup [0,\, \pi/2)$.

        Тогда решения уравнения с аргументом, лежащим в интервале $(3\pi,\, 7\pi/2)$ ~--- комплексные числа $\sqrt{2} \left( \cos \frac{\pi}{6} + i \sin \frac{\pi}{6} \right) =\sqrt{2} \left( \frac{\sqrt{3}}{2} + i \frac{1}{2} \right) = \bm{\frac{\sqrt{3}}{\sqrt{2}} + i\frac{\sqrt{2}}{2}}$, $\sqrt{2} \left( \cos \frac{7\pi}{6} + i \sin \frac{7\pi}{6} \right) =\sqrt{2} \left( -\frac{\sqrt{3}}{2} - i \frac{1}{2} \right) = \bm{-\frac{\sqrt{3}}{\sqrt{2}} - i\frac{\sqrt{2}}{2}}$ и $\sqrt{2} \left( \cos \frac{10\pi}{6} + i \sin \frac{10\pi}{6} \right) =\sqrt{2} \left( \frac{1}{2} - i \frac{\sqrt{3}}{2} \right) = \bm{\frac{\sqrt{2}}{2} - i\frac{\sqrt{3}}{\sqrt{2}}}$.

      \end{proof}
    
    \section*{Задача \textnumero 3}
      % текст задачи
	    \begin{proof}[Решение.]
		
      \end{proof}
    
    \section*{Задача \textnumero 4}
      % текст задачи
      Пусть $A = \begin{pmatrix}
        1 & 2 & 0 \\
        0 & -1 & 3
      \end{pmatrix}$. Докажите, что множество всех матриц $X \in \Mat_{2 \times 3}(\RR)$, удовлетворяющих условию $XA^T + AX^T = 0$, является подпространством в пространстве $\Mat_{2 \times 3}(\RR)$; найдите базис и размерность этого подпространства.
	    \begin{proof}[Решение.]
        Заметим, что $XA^T = (AX^T)^T$, тогда уравнение можно переписать так:
        \begin{equation*}
          (AX^T)^T + AX^T = 0.
        \end{equation*}
        Проверим множества решений $U$ на условия подпространства:
        \begin{enumerate}
          \item $(A \cdot 0^T)^T + A \cdot 0^T = 0 \implies 0 \in U$
          \item $Y,\, Z \in U \implies (A(Y + Z)^T)^T + A(Y + Z)^T = \underbrace{(AY^T)^T + AY^T}_0 + \underbrace{(AZ^T)^T + AZ^T}_0 = 0 \implies Y+Z \in U$
          \item $Y \in U,\, \alpha \in \RR \implies (A(\alpha Y)^T)^T + A (\alpha Y)^T = \alpha (\underbrace{(AY^T)^T + AY^T}_0) = 0$
        \end{enumerate}
        Значит, $U$~--- подпространство в $\Mat_{2\times 3}(\RR)$.

        $AX^T$~--- это некоторая матрица размера $2 \times 2$, которая в сложении с собой транспонированной даёт ноль. У такой матрицы $[AX^T]_{ij} = -[AX^T]{ji}$, то есть диагональные элементы нулевые, а остальные противоположны с симметричным. Запишем это условие:
        \begin{equation*}
          AX^T = \begin{pmatrix}
            1 & 2 & 0 \\
            0 & -1 & 3
          \end{pmatrix} \begin{pmatrix}
            x_1 & x_4 \\
            x_2 & x_5 \\
            x_3 & x_6
          \end{pmatrix} = \begin{pmatrix}
            x_1 + 2x_2 & x_4 + 2x_5 \\
            -x_2 + 3x_3 & -x_5 + 3x_6
          \end{pmatrix} \implies \begin{dcases}
            x_1 + 2x_2 = 0\\
            x_4 + 2x_5 + x_2 - 3x_3 = 0\\
            -x_5 + 3x_6 = 0
          \end{dcases}
        \end{equation*}
        Решим ОСЛУ от переменных $x_1,\dots , x_6$:
        \begin{multline*}
          \smt{cccccc|c}{
            1 & 2 & 0 & 0 & 0 & 0 & 0 \\
            0 & 1 & -3 & 1 & 2 & 0 & 0 \\ 
            0 & 0 & 0 & 0 & -1 & 3 & 0
          } \overset{\elth{3}{-1}}{\arron{1}{2}{-2}} \smt{cccccc|c}{
            1 & 0 & 6 & -2 & -4 & 0 & 0 \\
            0 & 1 & -3 & 1 & 2 & 0 & 0 \\
            0 & 0 & 0 & 0 & 1 & -3 & 0
          } \overset{\elon{1}{3}{4}}{\arron{2}{3}{-2}} \\
          \overset{\elon{1}{3}{4}}{\arron{2}{3}{-2}} \smt{cccccc|c}{
            1 & 0 & 6 & -2 & 0 & -12 & 0 \\
            0 & 1 & -3 & 1 & 0 & 6 & 0 \\
            0 & 0 & 0 & 0 & 1 & -3 & 0
          }
        \end{multline*}
        Общее решение выводится несложно, но оно нам не нужно. Нам нужно ФСР. Оно будет содержать три вектора и выглядеть следующим образом: 
        \begin{equation*}
          v_1 = \begin{pmatrix}
            -6 \\
            3 \\
            1 \\
            0 \\
            0 \\
            0 \\
          \end{pmatrix},\ v_2 = \begin{pmatrix}
            2 \\
            -1 \\
            0 \\
            1 \\
            0 \\
            0
          \end{pmatrix},\ v_3 = \begin{pmatrix}
            12 \\
            -6 \\
            0 \\
            0 \\
            3 \\
            1
          \end{pmatrix}.
        \end{equation*}
        Тогда базис нашего пространства $U$~--- это матрицы 
        \begin{equation*}
          \bm{X_1} = \begin{pmatrix}
            \bm{-6} & \bm{3} & \bm{1} \\
            \bm{0} & \bm{0} & \bm{0}
          \end{pmatrix},\ \bm{X_2} = \begin{pmatrix}
            \bm{2} & \bm{-1} & \bm{0} \\
            \bm{1} & \bm{0} & \bm{0}
          \end{pmatrix},\ \bm{X_3} = \begin{pmatrix}
            \bm{12} & \bm{-6} & \bm{0} \\
            \bm{0} & \bm{3} & \bm{1}
          \end{pmatrix},
        \end{equation*}
        а его размерность (размер базиса)~--- $\bm{3}$.

      \end{proof}
    
    \section*{Задача \textnumero 5}
      % текст задачи
	    \begin{proof}[Решение.]
		
      \end{proof}

    \section*{Задача \textnumero 6}
      % текст задачи
      Существует ли однородная система линейных уравнений, для которой векторы $v_1 =(0,\, -1,\, -1,\, 1,\, 0)$, $v_2 = (1,\, 2,\, 2,\, 0,\, 0)$, $v_3 = (1,\, 0,\, 2,\, 0,\, 2)$ образуют фундаментальную систему решений? Если существует, то укажите её.
	    \begin{proof}[Решение.]
        Для начала проверим, что векторы $v_1,\, v_2$ и $v_3$ линейно независимы. Для этого запишем их в столбцы матрицы и приведём её к ступенчатому виду.
        \begin{multline*}
          \begin{pmatrix}
            0 & 1 & 1 \\
            -1 & -1 & -1 \\
            -1 & 2 & 2 \\
            1 & 0 & 0 \\
            0 & 0 & 2
          \end{pmatrix} \arrtw{1}{2} \begin{pmatrix}
            -1 & -1 & -1 \\
            0 & 1 & 1 \\
            -1 & 2 & 2 \\
            1 & 0 & 0 \\
            0 & 0 & 2
          \end{pmatrix} \overset{\elon{4}{1}{1}}{\arron{3}{1}{-1}} \begin{pmatrix}
            -1 & -1 & -1 \\
            0 & 1 & 1 \\
            0 & 3 & 3 \\
            0 & -1 & -1 \\
            0 & 0 & 2
          \end{pmatrix} \overset{\elon{4}{2}{1}}{\arron{3}{2}{-3}} \begin{pmatrix}
            -1 & -1 & -1 \\
            0 & 1 & 1 \\
            0 & 0 & 0 \\
            0 & 0 & 0 \\
            0 & 0 & 2
          \end{pmatrix} \arrtw{3}{5} \\
          \arrtw{3}{5} \left(
            \begin{tabular}{ccc}
            \multicolumn{1}{|c}{$-1$} & $-1$                     & $-1$ \\ \cline{1-1}
            \multicolumn{1}{c|}{$0$}  & $1$                      & $1$  \\ \cline{2-2}
            $0$                       & \multicolumn{1}{c|}{$0$} & $2$  \\ \cline{3-3} 
            $0$                       & $0$                      & $0$  \\
            $0$                       & $0$                      & $0$ 
            \end{tabular}\right)
        \end{multline*}  
        Действительно, эти векторы линейно независимы. В таком случае мы точно можем сказать, что существует ОСЛУ, для которой $v_1,\, v_2,\, v_3$ являются ФСР, так как у нас есть алгоритм, по которому эта ОСЛУ строится. Воспользуемся им.
        \begin{enumerate}
          \item Уложим векторы $v_1,\, v_2,\, v_3$ в строки матрицы $B \in \Mat_{3 \times 5}(\RR)$. Матрица $B$ будет выглядеть так:
          \begin{equation*}
            B = \begin{pmatrix}
              0 & -1 & -1 & 1 & 0 \\
              1 & 2 & 2 & 0 & 0 \\
              1 & 0 & 2 & 0 & 2
            \end{pmatrix}
          \end{equation*}
          \item Найдём ФСР системы $Bz = 0$:
          \begin{multline*}
            \smt{ccccc|c}{
              0 & -1 & -1 & 1 & 0 & 0 \\
              1 & 2 & 2 & 0 & 0 & 0 \\
              1 & 0 & 2 & 0 & 2 & 0
            } \arrtw{1}{2} \smt{ccccc|c}{
              1 & 2 & 2 & 0 & 0 & 0 \\
              0 & -1 & -1 & 1 & 0 & 0 \\
              1 & 0 & 2 & 0 & 2 & 0
            } \arron{3}{1}{-1} \smt{ccccc|c}{
              1 & 2 & 2 & 0 & 0 & 0 \\
              0 & -1 & -1 & 1 & 0 & 0 \\
              0 & -2 & 0 & 0 & 2 & 0
            } \arron{3}{2}{-2} \\ 
            \arron{3}{2}{-2} \smt{ccccc|c}{
              1 & 2 & 2 & 0 & 0 & 0 \\
              0 & -1 & -1 & 1 & 0 & 0 \\
              0 & 0 & 2 & -2 & 2 & 0
            } \overset{\elth{2}{-1}}{\arrth{3}{1/2}} \smt{ccccc|c}{
              1 & 2 & 2 & 0 & 0 & 0 \\
              0 & 1 & 1 & -1 & 0 & 0 \\
              0 & 0 & 1 & -1 & 1 & 0
            } \arron{1}{2}{-2} \smt{ccccc|c}{
              1 & 0 & 0 & 2 & 0 & 0 \\
              0 & 1 & 1 & -1 & 0 & 0 \\
              0 & 0 & 1 & -1 & 1 & 0
            } \to\\
            {\arron{2}{3}{-1}} \smt{ccccc|c}{
              1 & 0 & 0 & 2 & 0 & 0 \\
              0 & 1 & 0 & 0 & -1 & 0 \\
              0 & 0 & 1 & -1 & 1 & 0
            }
          \end{multline*}
          Дальше легко восстановить решение ОСЛУ, но нас интересует ФСР. Она будет выглядеть следующим образом:
          \begin{equation*}
            u_1 = \begin{pmatrix}
              -2 \\
              0 \\
              1 \\
              1 \\
              0
            \end{pmatrix},\ u_2 = \begin{pmatrix}
              0 \\
              1 \\
              -1 \\
              0 \\
              1
            \end{pmatrix}
          \end{equation*}
          \item Уложим ФСР в строки матрицы $A \in \Mat_{2 \times 5}(\RR)$. Матрица $A$ будет матрицей искомой ОСЛУ и будет выглядеть так:
          \begin{equation*}
            A = \begin{pmatrix}
              -2 & 0 & 1 & 1 & 0 \\
              0 & 1 & -1 & 0 & 1
            \end{pmatrix}
          \end{equation*}
        \end{enumerate}
        Не забываем, что нас просили найти не матрицу, а саму ОСЛУ. Запишем её:
        \begin{equation*}
          \begin{cases}
            \bm{-2x_1 + x_3 + x_4 = 0} \\
            \bm{x_2 - x_3 + x_5 = 0}
          \end{cases}
        \end{equation*} 
      \end{proof}

    \section*{Задача \textnumero 7}
      % текст задачи
	    \begin{proof}[Решение.]
		
      \end{proof}

    \section*{Задача \textnumero 8}
      % текст задачи
	    \begin{proof}[Решение.]
		
      \end{proof}
 	
\end{document}