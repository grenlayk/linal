\documentclass[a4paper]{article}
\usepackage{header}
\newcommand{\Mat}{\operatorname{Mat}}
\newcommand{\M}{\operatorname{M}}

%% Графика
\usepackage{graphicx}       
\graphicspath{{images/}}            
\usepackage{tikz}  
\usetikzlibrary{patterns}                 
\usepackage{pgfplots}              
\usepackage{circuitikz}
            

\usetikzlibrary{quotes,angles}
\usetikzlibrary{positioning,intersections}
\usetikzlibrary{through}

\textwidth=19.0cm \oddsidemargin=-1.3cm
\textheight=25cm \topmargin=-2.5cm

\newtheorem{task}{Задание}
\newtheorem*{task*}{Задание}

\theoremstyle{remark}

\newtheorem{remark}{Замечание}
\newtheorem*{remark*}{Замечание}
\newtheorem{commentarium}{Комментарий}
\newtheorem*{commentarium*}{Комментарий}

\usepackage{tikz-cd}

\newenvironment{sysmatrix}[1]
{
    \left(\begin{array}{@{}#1@{}}
}
{\end{array}\right)}
\newcommand{\smt}[2]{\begin{sysmatrix}{#1} #2\end{sysmatrix}}
\newcommand{\eq}[1]{\begin{cases} #1 \end{cases}}

\newcommand{\elon}[3]{%
  \ensuremath{\text{Э}_1(#1,\; #2,\; #3)}%
}
\newcommand{\eltw}[2]{%
  \ensuremath{\text{Э}_2(#1,\; #2)}%
}
\newcommand{\elth}[2]{%
  \ensuremath{\text{Э}_3(#1,\; #2)}%
}
 
\newcommand{\eqon}[3]{%
  \ensuremath{\overset{\text{Э}_1(#1,\; #2,\; #3)}{=}}%
}
\newcommand{\eqtw}[2]{%
  \ensuremath{\overset{\text{Э}_2(#1,\; #2)}{=}}%
}
\newcommand{\eqth}[2]{%
  \ensuremath{\overset{\text{Э}_3(#1,\; #2)}{=}}%
}

\newcommand{\arron}[3]{%
  \ensuremath{\xrightarrow{\text{Э}_1(#1,\; #2,\; #3)}}%
}
\newcommand{\arrtw}[2]{%
  \ensuremath{\xrightarrow{\text{Э}_2(#1,\; #2)}}%
}
\newcommand{\arrth}[2]{%
  \ensuremath{\xrightarrow{\text{Э}_3(#1,\; #2)}}%
}


\title{Зимний экзамен учебного года 2019/2020\\Вариант \textnumero 1}
\author{	
        % ОТКОММЕНТИРУЙ СЕБЯ
    % % Александр Богданов   \\ \href{https://t.me/SphericalPotatoInVacuum}{Telegram} \and
    Алиса Вернигор       \\ \href{https://t.me/allisyonok}{Telegram} \and
    % % Анастасия Григорьева \\ \href{https://t.me/weifoll}{Telegram} \and
    % Василий Шныпко       \\ \href{https://t.me/yourvash}{Telegram} \and
    % % Данил Казанцев       \\ \href{https://t.me/vserosbuybuy}{Telegram} \and
    % Денис Козлов         \\ \href{https://t.me/DKozl50}{Telegram} \and
    % Елизавета Орешонок   \\ \href{https://t.me/eaoresh}{Telegram} \and
    % % Иван Пешехонов       \\ \href{https://t.me/JohanDDC}{Telegram} \and
    % % Иван Добросовестнов  \\ \href{https://t.me/ivankot13}{Telegram} \and
    % % Настя Городилова     \\ \href{https://t.me/nastygorodi}{Telegram} \and
    % Никита Насонков      \\ \href{https://t.me/nnv_nick}{Telegram} \and
    % Даниэль Хайбулин      \\ \href{https://t.me/kiDaniel}{Telegram} \and
	Сергей Лоптев        \\ \href{https://t.me/beast_sl}{Telegram} \and
	Оля Козлова        \\ \href{https://t.me/grenlayk}{Telegram}
	% Сабина Даянова        \\ \href{https://t.me/sabinadayanova}{Telegram} \and
}

\date{}

\begin{document}
	\maketitle

    \section*{Задача \textnumero 1}
        % текст задачи
        \begin{proof}[Решение.]
		
        \end{proof}	 
    
    \section*{Задача \textnumero 2}
        % текст задачи
	    \begin{proof}[Решение.]
		
        \end{proof}
    
    \section*{Задача \textnumero 3}
        % текст задачи
	    \begin{proof}[Решение.]
		
        \end{proof}
    
    \section*{Задача \textnumero 4}
        % текст задачи
	    \begin{proof}[Решение.]
		
        \end{proof}
    
    \section*{Задача \textnumero 5}
        % текст задачи
	    \begin{proof}[Решение.]
		
        \end{proof}

    \section*{Задача \textnumero 6}
        % текст задачи
	    \begin{proof}[Решение.]
		
        \end{proof}

    \section*{Задача \textnumero 7}
        % текст задачи
        Найдите все значения параметра a, при которых матрица $A = 
        \begin{pmatrix}
            a & -1 & 0 & -2 \\ 
            3 & 3 & 2 & 0 \\ 
            -2 & 5 & 1 & 7 \\ 
        \end{pmatrix} 
        $
        представима в виде суммы двух матриц ранга 1, и для каждого 
        значения укажите такое представление.
	    \begin{proof}[Решение 1.]
            Знаем, что матрица ранга $r$ представима в виде суммы $r$ матриц ранга 1 и не представима
            в виде суммы меньшего числа таких матриц 
            (\href{https://www.youtube.com/watch?v=RgIt32Bc_gc&t=1h0m21s}{доказательство}). 
            Поэтому нам нужно такое значение параметра $a$, что rk $A = 2$. 

            Приведем ее к улучшенному ступенчатому виду элементарными преобразованиями строк: 
            %\xrightarrow{\text{Э}_2(1,3)}
            \begin{multline*}
                \begin{pmatrix}
                    a & -1 & 0 & -2 \\ 
                    3 & 3 & 2 & 0 \\ 
                    -2 & 5 & 1 & 7 \\ 
                \end{pmatrix} 
                \arron{2}{3}{-2}
                \begin{pmatrix}
                    a & -1 & 0 & -2 \\ 
                    7 & -7 & 0 & -14 \\ 
                    -2 & 5 & 1 & 7 \\ 
                \end{pmatrix}
                \arrtw{2}{-1}
                \begin{pmatrix}
                    a & -1 & 0 & -2 \\ 
                    -1 & 1 & 0 & 2 \\ 
                    -2 & 5 & 1 & 7 \\ 
                \end{pmatrix}
                \arron{1}{2}{1}\\
                \arron{1}{2}{1}
                \begin{pmatrix}
                    a-1 & 0 & 0 & 0 \\ 
                    -1 & 1 & 0 & 2 \\ 
                    -2 & 5 & 1 & 7 \\ 
                \end{pmatrix}
                \arron{3}{2}{-5}
                \begin{pmatrix}
                    a-1 & 0 & 0 & 0 \\ 
                    -1 & 1 & 0 & 2 \\ 
                    3 & 0 & 1 & -3 \\ 
                \end{pmatrix}
            \end{multline*}

            Получаем 2 ненулевые строки $\implies \text{rk}A \geqslant 2$. Однако нам нужно, 
            чтобы ранг был в точности 2 $\implies$ верхняя строка нулевая $\implies a = 1$ и это 
            единственное подходящее значение параметра.

            Улучшенный ступенчатый вид тогда: 
            $\begin{pmatrix}
                0 & 0 & 0 & 0 \\ 
                -1 & 1 & 0 & 2 \\ 
                3 & 0 & 1 & -3 \\ 
            \end{pmatrix}
            $

            Найдем по нему разложение. Выразим 1 и 4 столбец через 2 и 3: 

            $
            A = 
            \begin{pmatrix}
                -1A^{(2)} & 1A^{(2)} & 0 & 2A^{(2)} \\ 
            \end{pmatrix}
            + 
            \begin{pmatrix}
                3A^{(3)} & 0 & 1A^{(3)} & -3A^{(3)}
            \end{pmatrix} 
            = 
            \begin{pmatrix}
                1 & -1 & 0 & -2 \\ 
                -3 & 3 & 0 & 6 \\ 
                -5 & 5 & 0 & 10 \\ 
            \end{pmatrix}
            +
            \begin{pmatrix}
                0 & 0 & 0 & 0 \\ 
                6 & 0 & 2 & -6 \\ 
                3 & 0 & 1 & -3 \\ 
            \end{pmatrix}
            $

            (Общий алгоритм разложения матрицы в сумму матриц ранга 1 описан \href{https://drive.google.com/drive/folders/15svYwT_WAkuJ3TXS9_b5PeYdkD_t5gmj}{здесь} 
            в предпоследнем абзаце.) 

            Ответ: $a = 1$, $
            A = 
            \begin{pmatrix}
                1 & -1 & 0 & -2 \\ 
                -3 & 3 & 0 & 6 \\ 
                -5 & 5 & 0 & 10 \\ 
            \end{pmatrix}
            +
            \begin{pmatrix}
                0 & 0 & 0 & 0 \\ 
                6 & 0 & 2 & -6 \\ 
                3 & 0 & 1 & -3 \\ 
            \end{pmatrix}
            $
        \end{proof}

        \begin{proof}[Решение 2.] (Только идея, мне лень это техать)\

            Вспомним теорему о ранге матрицы (Теорема 14.1 в конспекте с hse-tex): 
            rk $A$ = наибольшему порядку ненулевого минора в $A$. 
            
            Отличие от предыдущего решения лишь в том, что мы по-другому будем искать подходящее 
            значение параметра $a$. Ненулевой минор ранга 2 точно найдется, 
            например
            $
            \begin{vmatrix}
                3 & 3 \\ 
                -2 & 5 \\ 
            \end{vmatrix}
            \implies \text{rk} A \geqslant 2$. Однако мы хотим чтобы ранг был в точности 2, поэтому 
            все миноры 3 порядка должны быть нулевыми. Осталось проверить все 4 варианта и оставить только те 
            значения параметра $a$, при которых везде получается 0. Далее ищем разложение как в решении 1.
            
            (Я реально это на экзе делала, прикиньте. 
            Кстати  проверку для минора, в который $a$ не входит, тоже надо выписать.)
        \end{proof}

    \section*{Задача \textnumero 8}
        % текст задачи
	    \begin{proof}[Решение.]
		
        \end{proof}
 	
\end{document}