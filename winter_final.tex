\documentclass[a4paper]{article}
\usepackage{header}
\newcommand{\Mat}{\operatorname{Mat}}
\newcommand{\M}{\operatorname{M}}
\allowdisplaybreaks

%% Графика
\usepackage{graphicx}       
\graphicspath{{images/}}            
\usepackage{tikz}  
\usetikzlibrary{patterns}                 
\usepackage{pgfplots}              
\usepackage{circuitikz}
\usepackage{bm}
            

\usetikzlibrary{quotes,angles}
\usetikzlibrary{positioning,intersections}
\usetikzlibrary{through}

\textwidth=19.0cm \oddsidemargin=-1.3cm
\textheight=25cm \topmargin=-2.5cm

\newtheorem{task}{Задание}
\newtheorem*{task*}{Задание}

\theoremstyle{remark}

\newtheorem{remark}{Замечание}
\newtheorem*{remark*}{Замечание}
\newtheorem{commentarium}{Комментарий}
\newtheorem*{commentarium*}{Комментарий}

\usepackage{tikz-cd}

\newenvironment{sysmatrix}[1]
{
    \left(\begin{array}{@{}#1@{}}
}
{\end{array}\right)}
\newcommand{\smt}[2]{\begin{sysmatrix}{#1} #2\end{sysmatrix}}
\newcommand{\eq}[1]{\begin{cases} #1 \end{cases}}

\newcommand{\elon}[3]{%
  \ensuremath{\text{Э}_1(#1,\; #2,\; #3)}%
}
\newcommand{\eltw}[2]{%
  \ensuremath{\text{Э}_2(#1,\; #2)}%
}
\newcommand{\elth}[2]{%
  \ensuremath{\text{Э}_3(#1,\; #2)}%
}
 
\newcommand{\eqon}[3]{%
  \ensuremath{\overset{\text{Э}_1(#1,\; #2,\; #3)}{=}}%
}
\newcommand{\eqtw}[2]{%
  \ensuremath{\overset{\text{Э}_2(#1,\; #2)}{=}}%
}
\newcommand{\eqth}[2]{%
  \ensuremath{\overset{\text{Э}_3(#1,\; #2)}{=}}%
}

\newcommand{\arron}[3]{%
  \ensuremath{\xrightarrow{\text{Э}_1(#1,\; #2,\; #3)}}%
}
\newcommand{\arrtw}[2]{%
  \ensuremath{\xrightarrow{\text{Э}_2(#1,\; #2)}}%
}
\newcommand{\arrth}[2]{%
  \ensuremath{\xrightarrow{\text{Э}_3(#1,\; #2)}}%
}

\makeatletter
\renewcommand*\env@matrix[1][*\c@MaxMatrixCols c]{%
  \hskip -\arraycolsep
  \let\@ifnextchar\new@ifnextchar
  \array{#1}}
\makeatother


\title{Зимний экзамен учебного года 2019/2020\\Вариант \textnumero 1}
\author{	
        % ОТКОММЕНТИРУЙ СЕБЯ
    % % Александр Богданов   \\ \href{https://t.me/SphericalPotatoInVacuum}{Telegram} \and
    Алиса Вернигор       \\ \href{https://t.me/allisyonok}{Telegram} \and
    % % Анастасия Григорьева \\ \href{https://t.me/weifoll}{Telegram} \and
    % Василий Шныпко       \\ \href{https://t.me/yourvash}{Telegram} \and
    % % Данил Казанцев       \\ \href{https://t.me/vserosbuybuy}{Telegram} \and
    % Денис Козлов         \\ \href{https://t.me/DKozl50}{Telegram} \and
    % Елизавета Орешонок   \\ \href{https://t.me/eaoresh}{Telegram} \and
    % % Иван Пешехонов       \\ \href{https://t.me/JohanDDC}{Telegram} \and
    % % Иван Добросовестнов  \\ \href{https://t.me/ivankot13}{Telegram} \and
    % % Настя Городилова     \\ \href{https://t.me/nastygorodi}{Telegram} \and
    % Никита Насонков      \\ \href{https://t.me/nnv_nick}{Telegram} \and
    % Даниэль Хайбулин      \\ \href{https://t.me/kiDaniel}{Telegram} \and
	Сергей Лоптев        \\ \href{https://t.me/beast_sl}{Telegram} \and
	Оля Козлова        \\ \href{https://t.me/grenlayk}{Telegram}
	% Сабина Даянова        \\ \href{https://t.me/sabinadayanova}{Telegram} \and
}

\date{}

\begin{document}
	\maketitle
    \section*{Задача \textnumero 1}

        % текст задачи
        Матрица $A \in \text{Mat}_{3 \times n}(\RR)$ такова, что система $Ax = b$ несовместна 
        при некотором $b \in \RR^{3}$, а система $A^Ty = c$ совместна для всех $c \in \RR^n$. 
        Каково наибольшее значение $n$, при котором такое возможно? Ответ обоснуйте. 
        \begin{proof}[Решение.]
            Столбцовый ранг матрицы $A$ равен строковому, а он в свою очередь $\leqslant 3$ 
            (тк всего 3 строки).

            \begin{itemize}
                \item Предположим, $n \geqslant 4$. Тогда улучшенный ступенчатый вид матрицы 
                $(A^T \mid c)$ выглядит так: 
                \[\smt{rrr|r} {
                    1 & 0 & 0 & c_1^{\prime} \\
                    0 & 1 & 0 & c_2^{\prime} \\
                    0 & 0 & 1 & c_3^{\prime} \\
                    0 & 0 & 0 & c_4^{\prime} \\
                    \vdots & \vdots & \vdots & \vdots\\
                    0 & 0 & 0 & c_n^{\prime}
                }\]
                так как ранг матрицы $A$ равен количеству строк / столбцов в улучшенном 
                ступенчатом виде (13.7 в конспекте с hse-tex) и так как это верно для любого $c$. 
                
                Но если $c_i^{\prime} \neq 0$ (где $i \geqslant 4$), то система несовместна. Такое нам не подходит. 
                $\implies n \leqslant 3$
                \item Предположим, $n = 3$ Тогда улучшенный ступенчатый вид матрицы 
                $(A^T \mid c)$ выглядит так:
                \[\smt{rrr|r} {
                    1 & 0 & 0 & c_1^{\prime} \\
                    0 & 1 & 0 & c_2^{\prime} \\
                    0 & 0 & 1 & c_3^{\prime} \\
                }\]
                Все ок, она совместна для любого $c$, так как нет строк вида $(0\ 0\ 0 \mid \Diamond)$, 
                где $\Diamond \neq 0$. Однако так как столбцовый ранг равен строковому, то 
                улучшенный ступенчатый вид матрицы 
                $(A \mid b)$ выглядит так:
                \[\smt{rrr|r} {
                    1 & 0 & 0 & b_1^{\prime} \\
                    0 & 1 & 0 & b_2^{\prime} \\
                    0 & 0 & 1 & b_3^{\prime} \\
                }\]
                Иии\dots она тоже совместна для любого $b$. Это нам не подходит :(

                \item Пусть теперь $n = 2$. Тогда улучшенный ступенчатый вид матрицы 
                $(A^T \mid c)$ выглядит так:
                \[\smt{rrr|r} {
                    1 & 0 & a_1 & c_1^{\prime} \\
                    0 & 1 & a_2 & c_2^{\prime} \\
                }\]
                И она совместна для любого $c$. Так как столбцовый ранг равен строковому, то 
                улучшенный ступенчатый вид матрицы 
                $(A \mid b)$ выглядит так:
                \[\smt{rr|r} {
                    1 & 0 & b_1^{\prime} \\
                    0 & 1 & b_2^{\prime} \\
                    a_1^\prime & a_2^\prime & b_3^{\prime} \\
                }\]
                и если $a_1^\prime = a_2^\prime = 0$, а $b_3^{\prime} \neq 0$, то 
                система несовместна. 

                Осталось лишь привести подходящий пример. (За отсутствие примера снижали!)

                Возьмем $A = 
                \begin{pmatrix}
                    1 & 0 \\ 
                    0 & 1  \\ 
                    0 & 0 \\ 
                \end{pmatrix} $
                
                Тогда $(A^T \mid c)$ всегда совместна --- подходит решение 
                $\begin{pmatrix}
                    c_1 \\ 
                    c_2 \\ 
                    0 \\ 
                \end{pmatrix}$, а система $(A \mid b)$ несовместна при $b = 
                \begin{pmatrix}
                    0 \\ 
                    0 \\ 
                    1 \\ 
                \end{pmatrix}$
            \end{itemize}

            Ответ: наибольшее возможное значение $n = 2$.
        \end{proof}	 
    
    \section*{Задача \textnumero 2}
      % текст задачи
      Найдите все комплексные решения уравнения $(2\sqrt{3} + i)z^4 = - 6\sqrt{3} + 10i$, у которых одиниз аргументов принадлежит интервалу $(3\pi,\, 7\pi/2)$.
      \begin{proof}[Решение.]
        Решим уравнение:
        \begin{align*}
          &\left( 2\sqrt{3} + i \right)z^4 = - 6\sqrt{3} + 10i \\
          &z^4 = \frac{-6\sqrt{3}+10i}{2\sqrt{3} + i} = \frac{\left(2\sqrt{3}\right)\left( -6\sqrt{3} \right) + 10}{\left( 2\sqrt{3} \right)^2 + 1} + \frac{\left( 10 \cdot 2\sqrt{3} \right) - \left( -6\sqrt{3} \right)}{\left( 2\sqrt{3} \right)^2 + 1}i = \frac{-26}{13} + \frac{26\sqrt{3}}{13}i = -2 + 2\sqrt{3}i \\
          &\big| z^4 \big| = \sqrt{(-2)^2 + \left( 2\sqrt{3} \right)^2} = \sqrt{16} = 4 \\
          &\cos \varphi = \frac{-2}{4} = -\frac{1}{2}\quad \text{--- косинус аргумента}\\
          &\sin \varphi = \frac{2\sqrt{3}}{4} = \frac{\sqrt{3}}{2} \quad \text{--- синус аргумента} \\
          &\varphi = \frac{2\pi}{3} \quad \text{--- аргумент}\\
          &z^4 = \big| z^4 \big| (\cos \varphi + i \sin \varphi) = 4\left( \cos \frac{2\pi}{3} + i \sin \frac{2\pi}{3} \right) \\
          &\begin{aligned}
            z = \sqrt[4]{4\left( \cos \frac{2\pi}{3} + i \sin \frac{2\pi}{3} \right)} = \sqrt[4]{4} \left( \cos \frac{2\pi/3 + 2\pi k}{4} + i \sin \frac{2\pi/3 + 2\pi k}{4} \right) = \sqrt{2} \left( \cos \left( \frac{\pi}{6} + \frac{\pi}{2} k \right) + i \sin \left( \frac{\pi}{6} + \frac{\pi}{2} k \right) \right), &\\
            k = 0,\, 1,\, 2,\, 3 &
          \end{aligned}
        \end{align*}
        Аргументам, принадлежащим интервалу $(3\pi,\, 7\pi/2)$, с отрезка $[0,\, 2\pi]$ соответствуют аргументы, принадлежащие интервалу $(\pi,\, 3\pi/2)$. Значит, нам подходит $k = 2$, так как $ \frac{7\pi}{6} \in (\pi,\, 3\pi/2)$ и $\frac{\pi}{6},\, \frac{4\pi}{6},\, \frac{10\pi}{6} \notin (\pi,\, 3\pi/2)$.

        Тогда решение уравнения с аргументом, лежащим в интервале $(3\pi,\, 7\pi/2)$ ~--- комплексное число
        \begin{equation*}
          \sqrt{2} \left( \cos \frac{7\pi}{6} + i \sin \frac{7\pi}{6} \right) =\sqrt{2} \left( -\frac{\sqrt{3}}{2} - i \frac{1}{2} \right) = \bm{-\frac{\sqrt{3}}{\sqrt{2}} - i\frac{\sqrt{2}}{2}}.
        \end{equation*}

      \end{proof}
    
    \section*{Задача \textnumero 3}
      Известно, что векторы  $v₁, v₂, v₃$ и $v₄$ некоторого векторного
      пространства над $\mathbb{R}$ линейно независимы. Могут ли
      векторы \\
      \[u₁ = 3v₁ + 2v₂ + v₄, u₂ = 2v₁ + v₂ + 3v₃ - v₄,
       u₃ = v₁ + v₂ - v₃ \] 
       быть линейно зависимыми? \textbf{Ответ обоснуйте.}
      \begin{proof}[Решение.] \ 
        \begin{itemize}
            \item  Мы знаем, что всякая линейно независимая система векторов
            является базисом своей линейной оболочки (этот факт упоминался
            на лекциях).
            \item Таким образом, векторы $v₁, v₂, v₃$ и $v₄$ являются
            базисом своей линейной оболочки -- назовём её $S$. 
            \item Заметим, что векторы $u₁, u₂, u₃ ∈ S$ (они представимы
            в виде линейной комбинации векторов $v₁, v₂, v₃$ и $v₄$, являющихся
            базисом $S$). 
            \item В базисе $v₁, v₂, v₃, v₄$ наши векторы $u₁, u₂, u₃$ 
            представимы следующим образом: \\
            $u₁ = \begin{pmatrix}
              3 \\
              2 \\ 
              0 \\ 
              1 
            \end{pmatrix}$ ,
            $u₂ = \begin{pmatrix}
              2 \\
              1 \\ 
              3 \\ 
              -1 
            \end{pmatrix}$ ,
            $u₃ = \begin{pmatrix}
              1 \\
              1 \\ 
              -1 \\ 
              0 
            \end{pmatrix}$
            \item Таким образом, мы свели нашу задачу к задаче о проверке 
            системы векторов на линейную независимость (просто в немного
            странном базисе).
            \item Вопрос о линейной зависимости конечного набора векторов
            сводится к составлению ОСЛУ и вопросу о наличии у него 
            ненулевого решения.
            \item Запишем векторы по столбцам и применим наш любимый метод
            Гаусса: \\
            $\begin{pmatrix}
              3 & 2 & 1\\
              2 & 1 & 1 \\
              0 & 3 & -1 \\
              1 & -1 & 0  
            \end{pmatrix}
            \rightsquigarrow
            \begin{pmatrix}
              1 & -1 & 0\\
              2 & 1 & 1 \\
              0 & 3 & -1 \\
              3 & 2 & 1
            \end{pmatrix}
            \rightsquigarrow
            \begin{pmatrix}
              1 & -1 & 0\\
              0 & 3 & 1 \\
              0 & 3 & -1 \\
              0 & 5 & 1
            \end{pmatrix}
            \rightsquigarrow
            \begin{pmatrix}
              1 & -1 & 0\\
              0 & 3 & 1 \\
              0 & 0 & 2 \\
              0 & 5 & 1
            \end{pmatrix}$
            \item Видим, что у нас 3 главных переменных (а неизвестных
            тоже 3) $⇒$ ОСЛУ имеет единственное решение -- нулевое $⇒$
            векторы линейно независимы.
        \end{itemize}   
        Внимательно смотрим на вопрос и пишем подходящий ответ: \\ 
        \textbf{Ответ:} Нет, не могут.
      \end{proof}
      \remark{
        Прийти к такой системе можно было и другим способом:
       \begin{itemize}
           \item Знаем, что векторы линейно зависимы тогда
           и только тогда, когда существует их нетривиальная (такая, что
           $∃i: \alpha_i ≠ 0$, ненулевая) линейная комбинации, равная 0
           \item То есть, тогда и только тогда, когда $∃a, b, c ∈ \mathbb{R}: 
           au₁ + bu₂ + cu₃ = 0$ и $(a ≠ 0) ∨ (b ≠ 0) ∨ (c ≠ 0)$
           \item Заметим, что: \\
           $u₁ = (v₁, v₂, v₃, v₄) ⋅ \begin{pmatrix}
             3 \\
             2 \\
             0 \\ 
             1
           \end{pmatrix}, u₂ = (v₁, v₂, v₃, v₄) ⋅ \begin{pmatrix}
            2 \\
            1 \\
            3 \\ 
            -1
          \end{pmatrix}, u₁ = (v₁, v₂, v₃, v₄) ⋅ \begin{pmatrix}
            1 \\
            1 \\
            -1 \\ 
            0
          \end{pmatrix}$ 
          \item Тогда имеем: \\
           $au₁ + bu₂ + cu₃ = a ⋅ (v₁, v₂, v₃, v₄) ⋅ \begin{pmatrix}
            3 \\
            2 \\
            0 \\ 
            1
          \end{pmatrix} + b ⋅ (v₁, v₂, v₃, v₄) ⋅ \begin{pmatrix}
           2 \\
           1 \\
           3 \\ 
           -1
         \end{pmatrix} + c ⋅ (v₁, v₂, v₃, v₄) ⋅ \begin{pmatrix}
           1 \\
           1 \\
           -1 \\ 
           0
         \end{pmatrix} = \\ = (v₁, v₂, v₃, v₄)\left(
          a  ⋅ \begin{pmatrix}
            3 \\
            2 \\
            0 \\ 
            1
          \end{pmatrix} + b  ⋅ \begin{pmatrix}
           2 \\
           1 \\
           3 \\ 
           -1
         \end{pmatrix} + c  ⋅ \begin{pmatrix}
           1 \\
           1 \\
           -1 \\ 
           0
         \end{pmatrix}
         \right)$
         \item Такое выражение равно 0, если:
         \begin{itemize}
             \item $(v₁, v₂, v₃, v₄) = \overline{0}$ -- сразу нет (у нас
             же линейно независимая система) 
             \item Второй множитель равен 0 -- вот это наша остановочка
         \end{itemize}
         \item Заметим, что: \\
         $ a  ⋅ \begin{pmatrix}
          3 \\
          2 \\
          0 \\ 
          1
        \end{pmatrix} + b  ⋅ \begin{pmatrix}
         2 \\
         1 \\
         3 \\ 
         -1
       \end{pmatrix} + c  ⋅ \begin{pmatrix}
         1 \\
         1 \\
         -1 \\ 
         0
       \end{pmatrix} =  \begin{pmatrix}
        3 & 2 & 1\\
        2 & 1 & 1 \\
        0 & 3 & -1 \\
        1 & -1 & 0  
      \end{pmatrix} ⋅ \begin{pmatrix}
        a \\
        b \\
        c  
      \end{pmatrix}$
      \item Отсюда получаем ту же ОСЛУ, что и в решении выше.  
       \end{itemize}
      }
    \section*{Задача \textnumero 4}
      % текст задачи
      Пусть $A = \begin{pmatrix}
        1 & 2 & 0 \\
        0 & -1 & 3
      \end{pmatrix}$. Докажите, что множество всех матриц $X \in \Mat_{2 \times 3}(\RR)$, удовлетворяющих условию $XA^T + AX^T = 0$, является подпространством в пространстве $\Mat_{2 \times 3}(\RR)$; найдите базис и размерность этого подпространства.
	    \begin{proof}[Решение.]
        Заметим, что $XA^T = (AX^T)^T$, тогда уравнение можно переписать так:
        \begin{equation*}
          (AX^T)^T + AX^T = 0.
        \end{equation*}
        Проверим множества решений $U$ на условия подпространства:
        \begin{enumerate}
          \item $(A \cdot 0^T)^T + A \cdot 0^T = 0 \implies 0 \in U$
          \item $Y,\, Z \in U \implies (A(Y + Z)^T)^T + A(Y + Z)^T = \underbrace{(AY^T)^T + AY^T}_0 + \underbrace{(AZ^T)^T + AZ^T}_0 = 0 \implies Y+Z \in U$
          \item $Y \in U,\, \alpha \in \RR \implies (A(\alpha Y)^T)^T + A (\alpha Y)^T = \alpha (\underbrace{(AY^T)^T + AY^T}_0) = 0$
        \end{enumerate}
        Значит, $U$~--- подпространство в $\Mat_{2\times 3}(\RR)$.

        $AX^T$~--- это некоторая матрица размера $2 \times 2$, которая в сложении с собой транспонированной даёт ноль. У такой матрицы $[AX^T]_{ij} = -[AX^T]_{ji}$, то есть диагональные элементы нулевые, а остальные противоположны с симметричными. Запишем это условие:
        \begin{equation*}
          AX^T = \begin{pmatrix}
            1 & 2 & 0 \\
            0 & -1 & 3
          \end{pmatrix} \begin{pmatrix}
            x_1 & x_4 \\
            x_2 & x_5 \\
            x_3 & x_6
          \end{pmatrix} = \begin{pmatrix}
            x_1 + 2x_2 & x_4 + 2x_5 \\
            -x_2 + 3x_3 & -x_5 + 3x_6
          \end{pmatrix} \implies \begin{dcases}
            x_1 + 2x_2 = 0\\
            x_4 + 2x_5 - x_2 + 3x_3 = 0\\
            -x_5 + 3x_6 = 0
          \end{dcases}
        \end{equation*}
        Решим ОСЛУ от переменных $x_1,\dots , x_6$:
        \begin{multline*}
          \smt{cccccc|c}{
            1 & 2 & 0 & 0 & 0 & 0 & 0 \\
            0 & -1 & 3 & 1 & 2 & 0 & 0 \\ 
            0 & 0 & 0 & 0 & -1 & 3 & 0
          } \overset{\elth{3}{-1}}{\arron{1}{2}{2}} \smt{cccccc|c}{
            1 & 0 & 6 & 2 & 4 & 0 & 0 \\
            0 & -1 & 3 & 1 & 2 & 0 & 0 \\
            0 & 0 & 0 & 0 & 1 & -3 & 0
          } \overset{\elon{1}{3}{-4}}{\arron{2}{3}{-2}} \\
          \overset{\elon{1}{3}{-4}}{\arron{2}{3}{-2}} \smt{cccccc|c}{
            1 & 0 & 6 & 2 & 0 & 12 & 0 \\
            0 & -1 & 3 & 1 & 0 & 6 & 0 \\
            0 & 0 & 0 & 0 & 1 & -3 & 0
          } \arrth{2}{-1} \smt{cccccc|c}{
            1 & 0 & 6 & 2 & 0 & 12 & 0 \\
            0 & 1 & -3 & -1 & 0 & -6 & 0 \\
            0 & 0 & 0 & 0 & 1 & -3 & 0
          }
        \end{multline*}
        Общее решение выводится несложно, но оно нам не нужно. Нам нужна ФСР. Она будет содержать три вектора и выглядеть следующим образом: 
        \begin{equation*}
          v_1 = \begin{pmatrix}
            -6 \\
            3 \\
            1 \\
            0 \\
            0 \\
            0 \\
          \end{pmatrix},\ v_2 = \begin{pmatrix}
            -2 \\
            1 \\
            0 \\
            1 \\
            0 \\
            0
          \end{pmatrix},\ v_3 = \begin{pmatrix}
            -12 \\
            6 \\
            0 \\
            0 \\
            3 \\
            1
          \end{pmatrix}.
        \end{equation*}
        Тогда базис нашего пространства $U$~--- это матрицы 
        \begin{equation*}
          \bm{X_1} = \begin{pmatrix}
            \bm{-6} & \bm{3} & \bm{1} \\
            \bm{0} & \bm{0} & \bm{0}
          \end{pmatrix},\ \bm{X_2} = \begin{pmatrix}
            \bm{-2} & \bm{1} & \bm{0} \\
            \bm{1} & \bm{0} & \bm{0}
          \end{pmatrix},\ \bm{X_3} = \begin{pmatrix}
            \bm{-12} & \bm{6} & \bm{0} \\
            \bm{0} & \bm{3} & \bm{1}
          \end{pmatrix},
        \end{equation*}
        а его размерность (размер базиса)~--- $\bm{3}$.

      \end{proof}
    
    \section*{Задача \textnumero 5}
      В пространстве $\mathbb{R}^5$ даны векторы 
      \[v₁ = (1, 2, 1, 3, 3), v₂ = (3, 7, 3, 5, 4),
      v₃ = (1, 1, 1, 7, 8), v₄ = (3, 4, 3, 8, 2).\]
      \begin{itemize}
          \item[(а)] Выберите среди данных векторов базис их линейной оболочки.
          \item[(б)] Дополните полученный в пункте (а) базис до базиса всего
          пространства $\mathbb{R}^5$. 
      \end{itemize}
	    \begin{proof}[Решение.] \ 
        \begin{itemize}
            \item[(а)] Алгоритм решения такой:
            \begin{itemize}
                \item Укладываем векторы в столбцы
                \item Приводим матрицу к улучшенному ступенчатому виду
                методом Гаусса (преобразованиями строк)
                \item Номера столбцов ведущих переменных и будут являться 
                номерами векторов, которые мы должны взять в базис
            \end{itemize} 
            $
              \begin{pmatrix}
                1 & 3 & 1 & 3 \\ 
                2 & 7 & 1 & 4 \\
                1 & 3 & 1 & 3 \\
                3 & 5 & 7 & 8 \\
                3 & 4 & 8 & 2
              \end{pmatrix} \rightsquigarrow
              \begin{pmatrix}
                1 & 3 & 1 & 3 \\ 
                0 & 1 & -1 & -2 \\
                0 & 0 & 0 & 0 \\
                0 & -4 & 4 & -1 \\
                0 & -5 & 5 & -7
              \end{pmatrix} \rightsquigarrow
              \begin{pmatrix}
                1 & 0 & 4 & 9 \\ 
                0 & 1 & -1 & -2 \\
                0 & 0 & 0 & 0 \\
                0 & 0 & 0 & -9 \\
                0 & 0 & 0 & -17
              \end{pmatrix} \rightsquigarrow
              \begin{pmatrix}
                1 & 0 & 4 & 0 \\ 
                0 & 1 & -1 & 0 \\
                0 & 0 & 0 & 1 \\
                0 & 0 & 0 & 0 \\
                0 & 0 & 0 & 0
              \end{pmatrix}
            ⇒$ \\ $⇒$в качестве базиса выбираем $v₁, v₂, v₄$
            \item[(б)] Алгоритм тут такой:
              \begin{itemize}
                  \item Укладываем полученный в пункте (а) базис по строкам
                  \item Приводим полученную матрицу к ступенчатому виду
                  преобразованиями строк
                  \item Дополняем базис базисными единичками с номерами НЕГЛАВНЫХ
                  переменных нашей системы
              \end{itemize}  
              $
              \begin{pmatrix}
                1 & 2 & 1 & 3 & 3 \\ 
                3 & 7 & 3 & 5 & 4 \\
                3 & 4 & 3 & 8 & 2 
              \end{pmatrix}
              \rightsquigarrow
              \begin{pmatrix}
                1 & 2 & 1 & 3 & 3 \\ 
                0 & 1 & 0 & -4 & -5 \\
                0 & -2 & 0 & -1 & -7 
              \end{pmatrix}
              \rightsquigarrow
              \begin{pmatrix}
                1 & 0 & 1 & 11 & 13 \\ 
                0 & 1 & 0 & -4 & -5 \\
                0 & 0 & 0 & -9 & -17 
              \end{pmatrix}
              $ \\
              Свободные переменные имеют номера 3 и 5 $⇒$ матричными единичками
              с такими номерами и необходимо дополнить наш базис. \\ 
              Дополняющие векторы: \\
              \[
                e₃ = \begin{pmatrix}
                  0 \\ 
                  0\\
                  1 \\
                  0 \\ 
                  0
                \end{pmatrix}, 
                e₅ = \begin{pmatrix}
                  0 \\ 
                  0\\
                  0 \\
                  0 \\ 
                  1
                \end{pmatrix}
                \]  
        \end{itemize}
        \begin{remark}
          Вспомним, что столбцовый ранг матрицы равен строковому рангу матрицы, поэтому
          мы могли не решать отдельно пункт (а), а сразу перейти к решению пункта (б): 
          линейно выразимый вектор (тот, который мы не взяли в базис), обнулился бы при
          преобразованиях.
        \end{remark}
      \end{proof}

    \section*{Задача \textnumero 6}
      % текст задачи
      Существует ли однородная система линейных уравнений, для которой векторы $v_1 =(0,\, -1,\, -1,\, 1,\, 0)$, $v_2 = (1,\, 2,\, 2,\, 0,\, 0)$, $v_3 = (1,\, 0,\, 2,\, 0,\, 2)$ образуют фундаментальную систему решений? Если существует, то укажите её.
	    \begin{proof}[Решение.]
        Для начала проверим, что векторы $v_1,\, v_2$ и $v_3$ линейно независимы. Для этого запишем их в столбцы матрицы и приведём её к ступенчатому виду.
        \begin{multline*}
          \begin{pmatrix}
            0 & 1 & 1 \\
            -1 & -1 & -1 \\
            -1 & 2 & 2 \\
            1 & 0 & 0 \\
            0 & 0 & 2
          \end{pmatrix} \arrtw{1}{2} \begin{pmatrix}
            -1 & -1 & -1 \\
            0 & 1 & 1 \\
            -1 & 2 & 2 \\
            1 & 0 & 0 \\
            0 & 0 & 2
          \end{pmatrix} \overset{\elon{4}{1}{1}}{\arron{3}{1}{-1}} \begin{pmatrix}
            -1 & -1 & -1 \\
            0 & 1 & 1 \\
            0 & 3 & 3 \\
            0 & -1 & -1 \\
            0 & 0 & 2
          \end{pmatrix} \overset{\elon{4}{2}{1}}{\arron{3}{2}{-3}} \begin{pmatrix}
            -1 & -1 & -1 \\
            0 & 1 & 1 \\
            0 & 0 & 0 \\
            0 & 0 & 0 \\
            0 & 0 & 2
          \end{pmatrix} \arrtw{3}{5} \\
          \arrtw{3}{5} \left(
            \begin{tabular}{ccc}
            \multicolumn{1}{|c}{$-1$} & $-1$                     & $-1$ \\ \cline{1-1}
            \multicolumn{1}{c|}{$0$}  & $1$                      & $1$  \\ \cline{2-2}
            $0$                       & \multicolumn{1}{c|}{$0$} & $2$  \\ \cline{3-3} 
            $0$                       & $0$                      & $0$  \\
            $0$                       & $0$                      & $0$ 
            \end{tabular}\right)
        \end{multline*}  
        Действительно, эти векторы линейно независимы. В таком случае мы точно можем сказать, что существует ОСЛУ, для которой $v_1,\, v_2,\, v_3$ являются ФСР, так как у нас есть алгоритм, по которому эта ОСЛУ строится. Воспользуемся им.
        \begin{enumerate}
          \item Уложим векторы $v_1,\, v_2,\, v_3$ в строки матрицы $B \in \Mat_{3 \times 5}(\RR)$. Матрица $B$ будет выглядеть так:
          \begin{equation*}
            B = \begin{pmatrix}
              0 & -1 & -1 & 1 & 0 \\
              1 & 2 & 2 & 0 & 0 \\
              1 & 0 & 2 & 0 & 2
            \end{pmatrix}
          \end{equation*}
          \item Найдём ФСР системы $Bz = 0$:
          \begin{multline*}
            \smt{ccccc|c}{
              0 & -1 & -1 & 1 & 0 & 0 \\
              1 & 2 & 2 & 0 & 0 & 0 \\
              1 & 0 & 2 & 0 & 2 & 0
            } \arrtw{1}{2} \smt{ccccc|c}{
              1 & 2 & 2 & 0 & 0 & 0 \\
              0 & -1 & -1 & 1 & 0 & 0 \\
              1 & 0 & 2 & 0 & 2 & 0
            } \arron{3}{1}{-1} \smt{ccccc|c}{
              1 & 2 & 2 & 0 & 0 & 0 \\
              0 & -1 & -1 & 1 & 0 & 0 \\
              0 & -2 & 0 & 0 & 2 & 0
            } \arron{3}{2}{-2} \\ 
            \arron{3}{2}{-2} \smt{ccccc|c}{
              1 & 2 & 2 & 0 & 0 & 0 \\
              0 & -1 & -1 & 1 & 0 & 0 \\
              0 & 0 & 2 & -2 & 2 & 0
            } \overset{\elth{2}{-1}}{\arrth{3}{1/2}} \smt{ccccc|c}{
              1 & 2 & 2 & 0 & 0 & 0 \\
              0 & 1 & 1 & -1 & 0 & 0 \\
              0 & 0 & 1 & -1 & 1 & 0
            } \arron{1}{2}{-2} \smt{ccccc|c}{
              1 & 0 & 0 & 2 & 0 & 0 \\
              0 & 1 & 1 & -1 & 0 & 0 \\
              0 & 0 & 1 & -1 & 1 & 0
            } \to\\
            {\arron{2}{3}{-1}} \smt{ccccc|c}{
              1 & 0 & 0 & 2 & 0 & 0 \\
              0 & 1 & 0 & 0 & -1 & 0 \\
              0 & 0 & 1 & -1 & 1 & 0
            }
          \end{multline*}
          Дальше легко восстановить решение ОСЛУ, но нас интересует ФСР. Она будет выглядеть следующим образом:
          \begin{equation*}
            u_1 = \begin{pmatrix}
              -2 \\
              0 \\
              1 \\
              1 \\
              0
            \end{pmatrix},\ u_2 = \begin{pmatrix}
              0 \\
              1 \\
              -1 \\
              0 \\
              1
            \end{pmatrix}
          \end{equation*}
          \item Уложим ФСР в строки матрицы $A \in \Mat_{2 \times 5}(\RR)$. Матрица $A$ будет матрицей искомой ОСЛУ и будет выглядеть так:
          \begin{equation*}
            A = \begin{pmatrix}
              -2 & 0 & 1 & 1 & 0 \\
              0 & 1 & -1 & 0 & 1
            \end{pmatrix}
          \end{equation*}
        \end{enumerate}
        Не забываем, что нас просили найти не матрицу, а саму ОСЛУ. Запишем её:
        \begin{equation*}
          \begin{cases}
            \bm{-2x_1 + x_3 + x_4 = 0} \\
            \bm{x_2 - x_3 + x_5 = 0}
          \end{cases}
        \end{equation*} 
      \end{proof}

    \section*{Задача \textnumero 7}
        Найдите все значения параметра $a$, при которых матрица $A = 
        \begin{pmatrix}
            a & -1 & 0 & -2 \\ 
            3 & 3 & 2 & 0 \\ 
            -2 & 5 & 1 & 7 \\ 
        \end{pmatrix} 
        $
        представима в виде суммы двух матриц ранга 1, и для каждого 
        значения укажите такое представление.
	    \begin{proof}[Решение 1.]
            Знаем, что матрица ранга $r$ представима в виде суммы $r$ матриц ранга 1 и не представима
            в виде суммы меньшего числа таких матриц 
            (\href{https://www.youtube.com/watch?v=RgIt32Bc_gc&t=1h0m21s}{доказательство}). 
            Поэтому нам нужно такое значение параметра $a$, что rk $A = 2$. 

            Применим к ней элементарные преобразования строк: 
            %\xrightarrow{\text{Э}_2(1,3)}
            \begin{multline*}
                \begin{pmatrix}
                    a & -1 & 0 & -2 \\ 
                    3 & 3 & 2 & 0 \\ 
                    -2 & 5 & 1 & 7 \\ 
                \end{pmatrix} 
                \arron{2}{3}{-2}
                \begin{pmatrix}
                    a & -1 & 0 & -2 \\ 
                    7 & -7 & 0 & -14 \\ 
                    -2 & 5 & 1 & 7 \\ 
                \end{pmatrix}
                \arrtw{2}{-1}
                \begin{pmatrix}
                    a & -1 & 0 & -2 \\ 
                    -1 & 1 & 0 & 2 \\ 
                    -2 & 5 & 1 & 7 \\ 
                \end{pmatrix}
                \arron{1}{2}{1}\\
                \arron{1}{2}{1}
                \begin{pmatrix}
                    a-1 & 0 & 0 & 0 \\ 
                    -1 & 1 & 0 & 2 \\ 
                    -2 & 5 & 1 & 7 \\ 
                \end{pmatrix}
                \arron{3}{2}{-5}
                \begin{pmatrix}
                    a-1 & 0 & 0 & 0 \\ 
                    -1 & 1 & 0 & 2 \\ 
                    3 & 0 & 1 & -3 \\ 
                \end{pmatrix}
            \end{multline*}

            Получаем 2 ненулевые строки $\implies \text{rk}A \geqslant 2$. Однако нам нужно, 
            чтобы ранг был в точности 2 $\implies$ верхняя строка нулевая $\implies a = 1$ и это 
            единственное подходящее значение параметра.

            Улучшенный ступенчатый вид тогда: 
            $\begin{pmatrix}
                0 & 0 & 0 & 0 \\ 
                -1 & 1 & 0 & 2 \\ 
                3 & 0 & 1 & -3 \\ 
            \end{pmatrix}
            $

            Найдем по нему разложение. Выразим 1 и 4 столбец через 2 и 3: 

            $
            A = 
            \begin{pmatrix}
                -1A^{(2)} & 1A^{(2)} & 0 & 2A^{(2)} \\ 
            \end{pmatrix}
            + 
            \begin{pmatrix}
                3A^{(3)} & 0 & 1A^{(3)} & -3A^{(3)}
            \end{pmatrix} 
            = 
            \begin{pmatrix}
                1 & -1 & 0 & -2 \\ 
                -3 & 3 & 0 & 6 \\ 
                -5 & 5 & 0 & 10 \\ 
            \end{pmatrix}
            +
            \begin{pmatrix}
                0 & 0 & 0 & 0 \\ 
                6 & 0 & 2 & -6 \\ 
                3 & 0 & 1 & -3 \\ 
            \end{pmatrix}
            $

            (Общий алгоритм разложения матрицы в сумму матриц ранга 1 описан \href{https://drive.google.com/drive/folders/15svYwT_WAkuJ3TXS9_b5PeYdkD_t5gmj}{здесь} 
            в предпоследнем абзаце.) 

            Ответ: $a = 1$, $
            A = 
            \begin{pmatrix}
                1 & -1 & 0 & -2 \\ 
                -3 & 3 & 0 & 6 \\ 
                -5 & 5 & 0 & 10 \\ 
            \end{pmatrix}
            +
            \begin{pmatrix}
                0 & 0 & 0 & 0 \\ 
                6 & 0 & 2 & -6 \\ 
                3 & 0 & 1 & -3 \\ 
            \end{pmatrix}
            $
        \end{proof}

        \begin{proof}[Решение 2.] (Только идея, мне лень это техать)\

            Вспомним теорему о ранге матрицы (теорема 14.1 в конспекте с hse-tex): 
            rk $A$ = наибольшему порядку ненулевого минора в $A$. 
            
            Отличие от предыдущего решения лишь в том, что мы по-другому будем искать подходящее 
            значение параметра $a$. Ненулевой минор ранга 2 точно найдется, 
            например
            $
            \begin{vmatrix}
                3 & 3 \\ 
                -2 & 5 \\ 
            \end{vmatrix}
            \implies \text{rk} A \geqslant 2$. Однако мы хотим чтобы ранг был в точности 2, поэтому 
            все миноры 3 порядка должны быть нулевыми. Осталось проверить все 4 варианта и оставить только те 
            значения параметра $a$, при которых везде получается 0. Далее ищем разложение как в решении 1.
            
            (Я реально это на экзе делала, прикиньте. 
            Кстати  проверку для минора, в который $a$ не входит, тоже надо выписать.)
        \end{proof}

    \section*{Задача \textnumero 8}
        % текст задачи
        Как изменится определитель матрицы $58 \times 58$, если с её столбцами одновременно 
        выполнить следующие действия 
        \begin{itemize}
            \item к каждому столбцу, кроме первого, прибавить предыдущий;
            \item из первого столбца вычесть последний?
        \end{itemize}
        Ответ обоснуйте.
	    \begin{proof}[Решение.]
            Пусть $A = 
            \begin{pmatrix}
                A_{(1)} & A_{(2)} & \ldots & A_{(57)} & A_{(58)} \\ 
            \end{pmatrix}$

            Тогда $A^{\prime} = 
            \begin{pmatrix}
                A_{(1)} - A_{(58)} & A_{(1)} + A_{(2)} & \ldots & A_{(56)} + A_{(57)} & A_{(57)} + A_{(58)} \\ 
            \end{pmatrix}$

            Запишем это следующим образом: 
            $A^{\prime} = A \cdot C$, где $C = 
            \begin{pmatrix}
                1 & 1 & 0 & \ldots & 0 & 0 \\ 
                0 & 1 & 1 & \ldots &   & 0 \\ 
                  & 0 & 1 &   &   &   \\ 
                \vdots &   &   & \ddots &   &  \vdots \\ 
                0 &  &  &   &   1 &   1\\ 
                -1 & 0 &  \ldots& \ldots &   0 &   1\\ 
            \end{pmatrix}
            $

            (Единицы на главное диагонали, единицы на диагонали над главной и -1 
            в $a_{58, 1}$, остальное нули. И да, это матрица перехода, но это просто для любопытных.)

            Вспомним теорему про определитель произведения матриц (теорема 7.1 в конспекте с hse-tex): 
            $\det AB = \det A \cdot \det B$.

            Тогда $\det A^{\prime} = \det A \cdot \det C$

            В определителе матрицы $C$ всего 2 ненулевых слагаемых, поэтому $\det C = (\text{sgn } id) \cdot 1 + (\text{sgn } \sigma) \cdot (-1)$, где 
            $\sigma = (1\ 2\ 3\ 4\ \ldots 57\ 58)$ (запись в виде цикла). $\text{sgn } \sigma = (-1)^{58 - 1} = -1$. Итого $\det C = 1 \cdot 1 + (-1)(-1) = 2$. 

            Тогда $\det A^{\prime} = 2\det A $

            Ответ: увеличится в 2 раза.
        \end{proof}
 	
\end{document}